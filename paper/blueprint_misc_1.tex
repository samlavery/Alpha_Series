%% blueprint_misc_1.tex
%%
%% LaTeX blueprint for miscellaneous Collatz/ infrastructure files:
%%   LiouvilleCounterexample, BeurlingCounterexample, CycleLemma,
%%   CyclotomicBridge, CyclotomicDrift, DiscreteContainment,
%%   DiscreteContainmentCE, EntangledPair, FloorTail, FoundationalGap,
%%   AFECoordinationConstructive
%%
%% Style: leanblueprint (plasTeX).  Compile with:
%%   leanblueprint build  or  pdflatex + makeindex
%%
\documentclass[a4paper,12pt]{article}

%% --- packages ---
\usepackage{amsmath,amssymb,amsthm}
\usepackage{hyperref}
\usepackage{xcolor}
\usepackage{enumitem}
\usepackage{geometry}
\geometry{margin=2.5cm}

%% --- leanblueprint shims ---
\newcommand{\lean}[1]{\texttt{#1}}
\newcommand{\uses}[1]{\par\noindent\textit{Uses: \texttt{#1}}}
\newcommand{\leanok}{\par\noindent\textit{(Lean~4 proof complete)}}
\newcommand{\sorry}{\par\noindent\textcolor{red}{\textit{(sorry present)}}}

%% --- theorem environments ---
\theoremstyle{plain}
\newtheorem{theorem}{Theorem}[section]
\newtheorem{lemma}[theorem]{Lemma}
\newtheorem{proposition}[theorem]{Proposition}
\newtheorem{corollary}[theorem]{Corollary}

\theoremstyle{definition}
\newtheorem{definition}[theorem]{Definition}
\newtheorem{axiom_block}[theorem]{Axiom}

\theoremstyle{remark}
\newtheorem{remark}[theorem]{Remark}
\newtheorem{example_block}[theorem]{Example}

%% --- title ---
\title{Blueprint: Miscellaneous Infrastructure I\\
       \large Counterexamples, Combinatorics, Cyclotomic Algebra,\\
               Discrete Containment, and the Entangled Spiral Pair}
\author{Lean~4 Formalization Project}
\date{}

\begin{document}
\maketitle
\tableofcontents

%% =========================================================================
\section{The Liouville Counterexample: Fragility of the Collatz Conjecture}
\label{sec:liouville}
%% =========================================================================

\lean{LiouvilleCounterexample}

The Collatz conjecture for the map $T(n) = n/2$ (even), $3n+1$ (odd)
rests on two independent number-theoretic properties of the multiplier
$m = 3$:
\begin{enumerate}
  \item \textbf{Integer residue structure.}
        The congruence $x \bmod 4$ forces extra halving ($\nu \geq 2$)
        for exactly half of odd inputs, giving average
        $\bar\nu \approx 1.65 > \log_2 3 \approx 1.585$, which
        implies contraction and prevents divergence.
  \item \textbf{Baker's foundational gap.}
        $|2^S - 3^k| \geq \exp(-C\, k^\kappa) > 0$, which prevents
        the cycle equation from being satisfiable for nontrivial
        integer orbits.
\end{enumerate}
This section proves that for Liouville multipliers $m \in (3,4) \cap \mathbb{Q}$
both conditions fail, and nontrivial cycles of every size form.
All results carry \emph{zero custom axioms}.

\subsection{The Collatz Map and Halving Structure}

\begin{definition}[Collatz map and two-step]
\lean{LiouvilleCounterexample.collatz},
\lean{LiouvilleCounterexample.collatzTwoStep}

The standard Collatz map is
\[
  T(n) = \begin{cases} n/2 & 2 \mid n, \\ 3n+1 & 2 \nmid n. \end{cases}
\]
The \emph{two-step} from an odd $x$ with minimal halving ($\nu = 1$)
is $T_2(x) = (3x+1)/2$.
\leanok
\end{definition}

\begin{lemma}[Forced extra halving for $x \equiv 1 \pmod{4}$]
\lean{LiouvilleCounterexample.double_halving_mod1}

If $x \equiv 1 \pmod{4}$, then $4 \mid (3x+1)$; in particular the
Collatz orbit gets at least two consecutive halvings.
\[
  x \equiv 1 \pmod{4} \implies 4 \mid (3x+1).
\]
\uses{LiouvilleCounterexample.collatz}
\leanok

\textit{Proof sketch.}  Direct modular arithmetic: $3 \cdot 1 + 1 = 4$,
and $3(4k+1)+1 = 12k+4 = 4(3k+1)$.
\end{lemma}

\begin{lemma}[Minimal halving for $x \equiv 3 \pmod{4}$]
\lean{LiouvilleCounterexample.min_halving_mod3}

If $x \equiv 3 \pmod{4}$, then $2 \mid (3x+1)$ but $4 \nmid (3x+1)$,
so exactly one halving occurs.
\[
  x \equiv 3 \pmod{4} \implies (3x+1) \equiv 2 \pmod{4}.
\]
\uses{LiouvilleCounterexample.collatz}
\leanok
\end{lemma}

\begin{lemma}[Two-step preserves odd parity after minimal halving]
\lean{LiouvilleCounterexample.collatzTwoStep_odd}

If $x \equiv 3 \pmod{4}$, then $T_2(x) = (3x+1)/2$ is odd.
\uses{LiouvilleCounterexample.min_halving_mod3}
\leanok
\end{lemma}

\begin{lemma}[Every odd integer forces extra or minimal halving]
\lean{LiouvilleCounterexample.forced_extra_halving_or_minimal}

For every odd $x$, either $4 \mid (3x+1)$ or $(3x+1) \equiv 2 \pmod{4}$.
This is the dichotomy underlying Tao's mixing argument.
\leanok
\end{lemma}

\subsection{Divergence Under Persistent Minimal Halving}

\begin{definition}[Minimal-halving orbit over $\mathbb{Q}$]
\lean{LiouvilleCounterexample.collatzMinHalvOrbit}

The orbit under persistent $\nu = 1$ is the sequence
$x_0, x_1, x_2, \ldots$ over $\mathbb{Q}$ defined by
$x_{k+1} = (3x_k + 1)/2$.
\leanok
\end{definition}

\begin{lemma}[Orbit positivity]
\lean{LiouvilleCounterexample.collatzMinHalvOrbit_pos}

If $x_0 > 0$ then $x_k > 0$ for all $k$.
\uses{LiouvilleCounterexample.collatzMinHalvOrbit}
\leanok
\end{lemma}

\begin{lemma}[Geometric growth lower bound]
\lean{LiouvilleCounterexample.collatzMinHalvOrbit_growth}

For all $k \in \mathbb{N}$,
$\left(\frac{3}{2}\right)^k x_0 \leq x_k$.
\uses{LiouvilleCounterexample.collatzMinHalvOrbit}
\leanok

\textit{Proof sketch.}  By induction: the step
$x_{k+1} = (3x_k+1)/2 \geq \frac{3}{2} x_k$ gives the ratio $3/2$,
and the $+1$ term only helps.
\end{lemma}

\begin{lemma}[Linear growth lower bound]
\lean{LiouvilleCounterexample.collatzMinHalvOrbit_linear}

For all $k \in \mathbb{N}$,
$x_0 + \frac{k}{2} \leq x_k$.

\textit{Proof sketch.}  Each step adds at least $1/2$:
$(3y+1)/2 - y = (y+1)/2 \geq 1/2$ for $y > 0$.
\uses{LiouvilleCounterexample.collatzMinHalvOrbit}
\leanok
\end{lemma}

\begin{theorem}[Divergence under minimal halving]
\lean{LiouvilleCounterexample.collatzMinHalvOrbit_unbounded}

For any $B \in \mathbb{Q}$ and $x_0 > 0$, there exists $k \in \mathbb{N}$
such that $x_k > B$.  The minimal-halving orbit is \emph{unbounded}.
\uses{LiouvilleCounterexample.collatzMinHalvOrbit_linear}
\leanok

\textit{Proof sketch.}  Choose $k > 2(B - x_0)$ and apply the linear
lower bound.
\end{theorem}

\subsection{Cycle Equation Over $\mathbb{Q}$}

A Collatz cycle of period $k$ with $S$ total halvings satisfies the
\emph{cycle equation}
\[
  x_0 \cdot (2^S - m^k) = W_k(m),
\]
where $W_k$ is always positive.  For a $1$-cycle with $\nu = 2$
halvings, this reduces to $x_0(4 - m) = 1$, giving $x_0 = 1/(4-m)$.

\begin{example_block}[Trivial Collatz cycle]
\lean{LiouvilleCounterexample.trivial_collatz_cycle}

For $m = 3$: the equation $x_0 = 1/(4-3) = 1$ gives the trivial cycle
$1 \to 4 \to 2 \to 1$.
\leanok
\end{example_block}

\begin{example_block}[No nontrivial integer $1$-cycle]
\lean{LiouvilleCounterexample.no_nontrivial_collatz_cycle}

For $m = 3$, $x_0 = 2$ does not satisfy $(3 \cdot 2 + 1)/4 = 2$.
Baker prevents this.
\leanok
\end{example_block}

\begin{theorem}[Cycle for any target over $\mathbb{Q}$]
\lean{LiouvilleCounterexample.collatz_cycle_for_any_target}

For every rational $x_0 > 1$, there exists $m \in \mathbb{Q}$ with
$3 < m < 4$ such that the generalized map $T_m(n) = n/2$ (even),
$mn+1$ (odd) has a $1$-cycle at $x_0$.  The explicit witness is
$m = (4x_0 - 1)/x_0$.
\leanok

\textit{Proof sketch.}  Set $m = (4x_0-1)/x_0$.  Then
$m x_0 + 1 = 4x_0$, so $(mx_0+1)/4 = x_0$.
The bounds $3 < m < 4$ follow from $x_0 > 1$.
\end{theorem}

\begin{lemma}[Foundational gap formula]
\lean{LiouvilleCounterexample.collatz_foundational_gap}

For $m = (4x_0 - 1)/x_0$, the gap is $4 - m = 1/x_0$.
As $x_0 \to \infty$ (Liouville limit), $m \to 4$ and the gap vanishes.
\leanok
\end{lemma}

\subsection{The Fragility Summary}

\begin{theorem}[Collatz fragility]
\lean{LiouvilleCounterexample.collatz_fragility}
\lean{LiouvilleCounterexample.two_pow_ne_three_pow}

The following hold simultaneously:
\begin{enumerate}
  \item \textbf{Integer uniqueness.}  $2^S \neq 3^k$ for all $S > 0$, $k \geq 0$
        (follows from unique prime factorization: $2^S$ is even, $3^k$ is odd).
  \item \textbf{Liouville cycles.}  For every $x_0 > 1$ there exists
        $m \in (3,4)$ giving a $1$-cycle at $x_0$.
\end{enumerate}
\leanok

\textit{Proof sketch.}  Part (1): if $2^S = 3^k$ then $2 \mid 3^k$,
contradicting the primality of $3$.  Part (2): \texttt{collatz\_cycle\_for\_any\_target}.

\begin{remark}
The analogy with the Riemann Hypothesis (Beurling counterexample) is exact:
\[
\renewcommand{\arraystretch}{1.3}
\begin{array}{|l|l|l|}
\hline
\textbf{Property} & \textbf{Beurling (RH)} & \textbf{Liouville (Collatz)} \\
\hline
\text{Axiom} & \text{log independence} & \text{Baker gap} \\
\text{When gap} = 0 & \text{off-line zeros form} & \text{nontrivial cycles form} \\
\text{Tilt analog} & p^{-\sigma}\ \text{vs}\ p^{-1/2} & \text{contraction rate}\ 3/2^\nu \\
\hline
\end{array}
\]
\end{remark}
\end{theorem}

%% =========================================================================
\section{The Beurling Counterexample: Fragility of the Riemann Hypothesis}
\label{sec:beurling}
%% =========================================================================

\lean{BeurlingCounterexample}

A zero-axiom demonstration that the Riemann Hypothesis for
$\zeta(s) = \prod_p (1-p^{-s})^{-1}$ depends on the $\mathbb{Z}$-linear
independence of $\{\log p : p\ \text{prime}\}$.  For Beurling systems where
this independence fails, the analogue of RH is false
(Diamond--Montgomery--Vorhauer 2006).

\subsection{Logarithmic Independence of Actual Primes}

\begin{lemma}[Log of a prime is positive]
\lean{BeurlingCounterexample.log_prime_pos}

For every prime $p$, $\log p > 0$.
\leanok
\end{lemma}

\begin{lemma}[Distinct primes have distinct powers]
\lean{BeurlingCounterexample.prime_pow_ne}

For distinct primes $p \neq q$ and $a > 0$, $p^a \neq q^b$ for any $b$.

\textit{Proof sketch.}  If $p^a = q^b$ then $p \mid q^b$, so $p \mid q$
(by primality), giving $p = q$, a contradiction.
\leanok
\end{lemma}

\begin{theorem}[Log-linear independence]
\lean{BeurlingCounterexample.log_independence}

For distinct primes $p \neq q$ and exponents $a, b > 0$,
\[
  a \log p \neq b \log q.
\]
This is the \emph{FundamentalGap constant} of $\zeta(s)$: the set
$\{\log p : p\ \text{prime}\}$ is $\mathbb{Z}$-linearly independent.
\uses{BeurlingCounterexample.prime_pow_ne}
\leanok

\textit{Proof sketch.}  If $a \log p = b \log q$ then
$\log(p^a) = \log(q^b)$, so $p^a = q^b$ (injectivity of $\log$ on
$(0,\infty)$), contradicting \lean{prime\_pow\_ne}.
\end{theorem}

\begin{corollary}[FundamentalGap gap is positive]
\lean{BeurlingCounterexample.fundamentalGap_gap_pos}

For distinct primes $p \neq q$ and $a, b > 0$,
$|a \log p - b \log q| > 0$.
Baker's theorem strengthens this to an effective lower bound
$\exp(-C(\log B)^\kappa)$.
\uses{BeurlingCounterexample.log_independence}
\leanok
\end{corollary}

\begin{example_block}[Concrete non-resonance for primes 2 and 3]
\lean{BeurlingCounterexample.no_resonance_2_3},
\lean{BeurlingCounterexample.no_resonance_2_3_v2}

$2\log 2 \neq \log 3$ and $3\log 2 \neq 2\log 3$; equivalently,
$4 \neq 3$ and $8 \neq 9$.
\leanok
\end{example_block}

\subsection{Logarithmic Dependence for Beurling ``Primes''}

Replace actual primes with $\{b, b^2, b^3, \ldots\}$ for a base $b > 1$.
All logarithms are multiples of $\log b$: the system is maximally
log-dependent.

\begin{theorem}[FundamentalGap gap is zero for Beurling systems]
\lean{BeurlingCounterexample.fundamentalGap_gap_zero}

For any $b > 1$ and $k \geq 0$,
$|\log(b^k) - k \log b| = 0$.
Baker's theorem gives no obstruction.
\leanok
\end{theorem}

\begin{theorem}[Any rational phase ratio is achievable]
\lean{BeurlingCounterexample.beurling_any_ratio}

For positive integers $a, b$,
$\log(2^a)/\log(2^b) = a/b$.
Phases are commensurable: resonance occurs at $t = 2\pi/\log b$.
\leanok
\end{theorem}

\subsection{The Tilt Structure}

\begin{definition}[Amplitude tilt]
\lean{BeurlingCounterexample.tilt}

The tilt $\tau_p(\sigma) = p^{-\sigma} - p^{-1/2}$ measures each
prime's deviation from the critical-line weighting.
\leanok
\end{definition}

\begin{lemma}[Tilt vanishes on the critical line]
\lean{BeurlingCounterexample.tilt_zero_critical}

$\tau_p(1/2) = 0$ for any system.
\leanok
\end{lemma}

\begin{lemma}[Tilt is negative for $\sigma > 1/2$, positive for $\sigma < 1/2$]
\lean{BeurlingCounterexample.tilt_neg_of_gt},
\lean{BeurlingCounterexample.tilt_pos_of_lt}

For $p \geq 2$: $\sigma > 1/2 \implies \tau_p(\sigma) < 0$ and
$\sigma < 1/2 \implies \tau_p(\sigma) > 0$.
\leanok
\end{lemma}

\subsection{The Fragility Summary}

\begin{theorem}[RH fragility]
\lean{BeurlingCounterexample.fragility}

The following hold simultaneously:
\begin{enumerate}
  \item For actual primes $p \neq q$ and $a, b > 0$:
        $|a \log p - b \log q| > 0$.
  \item For Beurling ``primes'' $\{b^k\}$:
        $|\log(b^k) - k \log b| = 0$.
\end{enumerate}
\uses{BeurlingCounterexample.fundamentalGap_gap_pos,
      BeurlingCounterexample.fundamentalGap_gap_zero}
\leanok

\textit{Proof sketch.}  Part (1) is \lean{fundamentalGap\_gap\_pos};
part (2) is \lean{fundamentalGap\_gap\_zero}.
The Diamond--Montgomery--Vorhauer theorem (2006) shows that Beurling
systems with $|{}\cdot{}| = 0$ have off-line zeros.
\end{theorem}

%% =========================================================================
\section{The Cycle Lemma (Dvoretzky--Motzkin)}
\label{sec:cycle-lemma}
%% =========================================================================

\lean{Collatz.CycleLemma}

A classical combinatorial result (Dvoretzky--Motzkin 1947) used in the
Collatz analysis to establish rotation rigidity of halving profiles.

\subsection{Partial Sums and Rotated Sums}

\begin{definition}[Partial sum]
\lean{Collatz.CycleLemma.partialSum}

$P_j = \sum_{i < j} d_i$ for a function $d : \mathbb{N} \to \mathbb{Z}$.
\leanok
\end{definition}

\begin{definition}[Rotated sum]
\lean{Collatz.CycleLemma.rotatedSum}

$Q_\ell = \sum_{i < \ell} d((j_0 + i) \bmod k)$ for a starting
index $j_0$.
\leanok
\end{definition}

\begin{lemma}[No-wrap case]
\lean{Collatz.CycleLemma.rotatedSum_no_wrap}

When $j_0 + \ell \leq k$ (no wraparound), the rotated sum equals
$P_{j_0 + \ell} - P_{j_0}$.
\leanok
\end{lemma}

\begin{lemma}[Wrap case]
\lean{Collatz.CycleLemma.rotatedSum_wrap}

When $j_0 + \ell > k$ (wraparound) and $\sum_{i<k} d_i = 0$, the
rotated sum equals $P_{(j_0+\ell) \bmod k} - P_{j_0} + P_k$.
Since $P_k = 0$ under the hypothesis, this simplifies to
$P_{(j_0+\ell) \bmod k} - P_{j_0}$.
\leanok
\end{lemma}

\subsection{Main Results}

\begin{theorem}[Cycle Lemma (natural number version)]
\lean{Collatz.CycleLemma.cycle_lemma_nat}

Let $k > 0$ and $d : \mathbb{N} \to \mathbb{Z}$ with $\sum_{i<k} d_i = 0$.
There exists $j_0 < k$ such that $\forall \ell \leq k$,
$Q_\ell(j_0) \geq 0$: every prefix of the rotation starting at $j_0$ is
nonneg.
\uses{Collatz.CycleLemma.rotatedSum_no_wrap,
      Collatz.CycleLemma.rotatedSum_wrap}
\leanok

\textit{Proof sketch.}  Choose $j_0$ minimizing $P_{j_0}$ over
$\{0, \ldots, k\}$.  For the no-wrap case,
$Q_\ell = P_{j_0+\ell} - P_{j_0} \geq 0$ by minimality.  The
wrap case uses $P_k = 0$ and the same minimality argument on the
modular remainder.
\end{theorem}

\begin{theorem}[Cycle Lemma (Fin version)]
\lean{Collatz.CycleLemma.cycle_lemma_fin}

For $d : \text{Fin}\,k \to \mathbb{Z}$ with $\sum_i d_i = 0$, there
exists $j_0 : \text{Fin}\,k$ such that all prefix sums of the rotated
sequence are nonneg.
\uses{Collatz.CycleLemma.cycle_lemma_nat}
\leanok
\end{theorem}

\begin{theorem}[Simple Rigidity]
\lean{Collatz.CycleLemma.nonneg_sum_zero_implies_all_zero}

If $d : \text{Fin}\,k \to \mathbb{Z}$ satisfies $\sum_i d_i = 0$ and
$d_i \geq 0$ for all $i$, then $d_i = 0$ for all $i$.
\leanok

\textit{Proof sketch.}  If some $d_i > 0$ then
$\sum_j d_j \geq d_i > 0$, contradicting the sum-zero hypothesis.
\end{theorem}

\begin{corollary}[Rotation Rigidity]
\lean{Collatz.CycleLemma.rotation_rigidity}

A closed walk (sum~$= 0$) with all steps nonneg forces every step to be
zero.  Direct consequence of Simple Rigidity.
\uses{Collatz.CycleLemma.nonneg_sum_zero_implies_all_zero}
\leanok
\end{corollary}

%% =========================================================================
\section{Cyclotomic Bridge}
\label{sec:cyclobridge}
%% =========================================================================

\lean{Collatz.CyclotomicBridge}

The cyclotomic ring $\mathbb{Z}[\zeta_d]$ captures the ``true'' orbit
position via the \emph{balance sum}
$B = \sum_r \mathrm{FW}_r \zeta^r$.
The central result lifts divisibility from $\mathbb{Z}$ to $\mathbb{Z}[\zeta_d]$:
if $\Phi_d(4,3) \mid W_k$ in $\mathbb{Z}$, then $(4-3\zeta) \mid B$ in
$O_K = \mathbb{Z}[\zeta_d]$.

\subsection{Bivariate Cyclotomic Polynomial}

\begin{definition}[Bivariate cyclotomic polynomial]
\lean{Collatz.CyclotomicBridge.cyclotomicBivar}

$\Phi_q(x,y) = \sum_{i=0}^{q-1} x^{q-1-i} y^i = (x^q - y^q)/(x-y)$.
\leanok
\end{definition}

\begin{lemma}[Factorization identity]
\lean{Collatz.CyclotomicBridge.cyclotomicBivar_mul_sub}

$(x - y) \cdot \Phi_q(x,y) = x^q - y^q$.
\leanok
\end{lemma}

\begin{lemma}[$\Phi_d(4,3)$ divides $4^m - 3^m$ when $d \mid m$]
\lean{Collatz.CyclotomicBridge.cyclotomicBivar_dvd_pow_sub_general}

For $d \mid m$, $(4^d - 3^d) \mid (4^m - 3^m)$ in $\mathbb{Z}$, and
$\Phi_d(4,3) \mid (4^m - 3^m)$.
\uses{Collatz.CyclotomicBridge.cyclotomicBivar_mul_sub}
\leanok
\end{lemma}

\begin{lemma}[$\Phi_d(4,3) > 0$ for $d \geq 1$]
\lean{Collatz.CyclotomicBridge.cyclotomicBivar_pos}

All terms $4^{d-1-i} \cdot 3^i$ are positive, so the sum is positive.
\leanok
\end{lemma}

\subsection{Cyclotomic Field and Ring of Integers}

\begin{definition}[Cyclotomic field and root]
\lean{Collatz.CyclotomicBridge.zetaD},
\lean{Collatz.CyclotomicBridge.CyclotomicFieldD},
\lean{Collatz.CyclotomicBridge.OKD}

$K = \mathbb{Q}(\zeta_d)$ with primitive $d$-th root $\zeta_d$ and ring
of integers $O_K = \mathbb{Z}[\zeta_d]$.
\leanok
\end{definition}

\begin{lemma}[$\zeta_d^n = \zeta_d^{n \bmod d}$]
\lean{Collatz.CyclotomicBridge.zetaD_pow_mod}

Powers of $\zeta_d$ fold modulo $d$.
\leanok
\end{lemma}

\begin{definition}[Balance sum and key element]
\lean{Collatz.CyclotomicBridge.balanceSumD},
\lean{Collatz.CyclotomicBridge.fourSubThreeZetaD}

$B = \sum_{r \in \text{Fin}\,d} \mathrm{FW}_r \zeta_d^r$ (the balance sum)
and $\alpha = 4 - 3\zeta_d$ (the key generator).
\leanok
\end{definition}

\begin{lemma}[$4 - 3\zeta_d \neq 0$ for $d \geq 2$]
\lean{Collatz.CyclotomicBridge.fourSubThreeZetaD_ne_zero}

$4 - 3\zeta_d \neq 0$, since $\zeta_d = 4/3$ would imply $4^d = 3^d$,
which contradicts unique factorization.
\uses{Collatz.CyclotomicBridge.zetaD_is_primitive}
\leanok
\end{lemma}

\begin{lemma}[$\gcd(3,\ 4-3\zeta) = 1$ in $O_K$]
\lean{Collatz.CyclotomicBridge.isCoprime_three_fourSubThreeO}

$3$ and $4-3\zeta_d$ are coprime in $O_K$: the Bezout witness is
$(\zeta_d - 1) \cdot 3 + 1 \cdot (4-3\zeta_d) = 1$.
\leanok
\end{lemma}

\subsection{The Folding Lemma and Wave Sum}

\begin{definition}[Wave sum polynomial]
\lean{Collatz.CyclotomicBridge.waveSumPoly}

$f(x) = \sum_{j < m} 3^{m-1-j} w_j x^j$ for weights $w : \text{Fin}\,m \to \mathbb{N}$.
\leanok
\end{definition}

\begin{lemma}[Folding unfolded $\to$ folded]
\lean{Collatz.CyclotomicBridge.sum_unfolded_eq_folded_zetaD}

If $\mathrm{FW}_r = \sum_{j : j \bmod d = r} w_j$, then
\[
  \sum_{j < m} w_j \zeta_d^j = \sum_{r < d} \mathrm{FW}_r \zeta_d^r = B.
\]
\uses{Collatz.CyclotomicBridge.zetaD_pow_mod}
\leanok
\end{lemma}

\subsection{The Central Bridge Theorem}

\begin{theorem}[Cyclotomic bridge]
\lean{Collatz.CyclotomicBridge.OKD_divisibility_from_waveSum_divisibility}

Suppose $d \geq 2$, $d \mid m$, and $\Phi_d(4,3) \mid f(4)$ in $\mathbb{Z}$.
Then $(4-3\zeta_d) \mid B$ in $O_K$.
\[
  \Phi_d(4,3) \mid W \text{ in } \mathbb{Z}
  \implies (4-3\zeta_d) \mid B \text{ in } \mathbb{Z}[\zeta_d].
\]
\uses{Collatz.CyclotomicBridge.cyclotomicBivar_dvd_pow_sub_general,
      Collatz.CyclotomicBridge.fourSubThreeZetaD_dvd_of_cyclotomicBivar_dvd_OKD,
      Collatz.CyclotomicBridge.sum_unfolded_eq_folded_zetaD}
\leanok

\textit{Proof sketch.}  Steps: (1) $\Phi_d(4,3) \mid f(4)$ lifts to
$(4-3\zeta) \mid f(4)$ via the factorization
$\Phi_d(4,3) = (4-3\zeta) \cdot C$ in $K$.
(2) $(4-3\zeta) \mid f(4) - f(3\zeta)$ since each summand $4^j - (3\zeta)^j$
factors through $(4 - 3\zeta)$.
(3) Therefore $(4-3\zeta) \mid f(3\zeta)$.
(4) $f(3\zeta) = 3^{m-1} \cdot \sum w_j \zeta^j = 3^{m-1} B$.
(5) Coprimality of 3 and $(4-3\zeta)$ gives $(4-3\zeta) \mid B$.
\end{theorem}

%% =========================================================================
\section{Cyclotomic Drift}
\label{sec:cyclodrift}
%% =========================================================================

\lean{Collatz.CyclotomicDrift}

Proves that no nontrivial \texttt{CycleProfile} is realizable, using the
cyclotomic rigidity path (algebraic number theory in $\mathbb{Z}[\zeta_d]$).

\subsection{Partial Sum Infrastructure}

\begin{lemma}[Partial sum at index 0 is 0]
\lean{Collatz.CyclotomicDrift.partialSum_zero}

$P(0) = 0$ for any \texttt{CycleProfile}.
\leanok
\end{lemma}

\begin{lemma}[Monotone lower bound: $S_j \geq j$]
\lean{Collatz.CyclotomicDrift.partialSum_ge_index}

The partial sum satisfies $j \leq P(j)$ for all $j < m$,
since each $\nu_i \geq 1$.
\leanok
\end{lemma}

\subsection{A$+$B Decomposition}

\begin{definition}[A and B terms]
\lean{Collatz.CyclotomicDrift.termA},
\lean{Collatz.CyclotomicDrift.termB}

Given a profile $P$ with $m$ odd steps and a free integer $n_0$:
\[
  A = 3^{m-1}(1 + 3n_0), \qquad
  B = \sum_{j \geq 1} 3^{m-1-j} \cdot 2^{S_j}.
\]
\leanok
\end{definition}

\begin{theorem}[Orbit quantity decomposes as $A + B$]
\lean{Collatz.CyclotomicDrift.E_eq_A_plus_B}

For any profile $P$ and integer $n_0$,
\[
  W + n_0 \cdot 3^m = A + B.
\]
\uses{Collatz.CyclotomicDrift.termA, Collatz.CyclotomicDrift.termB}
\leanok

\textit{Proof sketch.}  Split the wavesum $W$ by extracting the $j=0$ term
and merging with the $n_0 3^m$ contribution into $A$; the remaining
$j \geq 1$ terms form $B$.
\end{theorem}

%% =========================================================================
\section{Discrete Containment}
\label{sec:discrete-containment}
%% =========================================================================

\lean{DiscreteContainment}

The Collatz proof and the Riemann Hypothesis share the same mechanism:
integer structure (discreteness) prevents pathological behavior that
continuous systems permit.  This module assembles three proved
``pillars'' and applies them to conclude $\zeta(s) \neq 0$ in the
critical strip.

\subsection{Three Pillars (All Proved, Zero Custom Axioms)}

\begin{theorem}[Pillar 1: Finite Euler products are nonzero]
\lean{DiscreteContainment.finite_euler_product_ne_zero}

For any $N$ and $\operatorname{Re}(s) > 0$:
\[
  \prod_{p \leq N,\, p\ \text{prime}} (1 - p^{-s}) \neq 0.
\]
\uses{PrimeBranching.euler_factor_ne_zero}
\leanok
\end{theorem}

\begin{theorem}[Pillar 2: Energy convergence]
\lean{DiscreteContainment.pillar_energy}

For $\sigma > 1/2$, $\sum_{p\ \text{prime}} p^{-2\sigma} < \infty$.
\uses{PrimeBranching.energy_convergence}
\leanok
\end{theorem}

\begin{theorem}[Pillar 3: Log independence]
\lean{DiscreteContainment.pillar_log_independence}

The set $\{\log p : p\ \text{prime}\}$ is $\mathbb{Z}$-linearly independent
(unique factorization).
\uses{PrimeBranching.log_primes_ne_zero}
\leanok
\end{theorem}

\begin{theorem}[Pillar 3 is necessary]
\lean{DiscreteContainment.pillar3_necessary}

For Beurling systems $\{b^k\}$, Pillar 3 fails:
$|\log(b^k) - k\log b| = 0$, and off-line zeros form.
\uses{BeurlingCounterexample.fundamentalGap_gap_zero}
\leanok
\end{theorem}

\subsection{The Discrete Containment Principle}

\begin{theorem}[Discrete containment]
\lean{DiscreteContainment.discrete_containment}

Given: (i) all finite Euler sub-products nonzero, (ii) energy convergence,
(iii) log independence, and (iv) the geometric off-axis coordination
hypothesis (\lean{EntangledPair.GeometricOffAxisCoordinationHypothesis}),
then $\zeta(s) \neq 0$ for $1/2 < \operatorname{Re}(s) < 1$.
\uses{EntangledPair.strip_nonvanishing}
\leanok

\textit{Proof sketch.}  Now a theorem via
\lean{EntangledPair.strip\_nonvanishing}.  The three pillar hypotheses
are discharged from the proved results; the geometric coordination hypothesis
is the sole remaining input.
\end{theorem}

\begin{theorem}[$\zeta(s) \neq 0$ in the strip]
\lean{DiscreteContainment.zeta_ne_zero_in_strip}

Under the geometric coordination hypothesis,
$\zeta(s) \neq 0$ for $1/2 < \operatorname{Re}(s) < 1$.
\uses{DiscreteContainment.discrete_containment}
\leanok
\end{theorem}

\subsection{Functional Equation Reflection}

\begin{theorem}[Functional equation reflection (proved from Mathlib)]
\lean{DiscreteContainment.functional_equation_reflection}

For a nontrivial zero $\zeta(s) = 0$ with $\operatorname{Re}(s) < 1/2$:
$\zeta(1-s) = 0$ and $\operatorname{Re}(1-s) > 1/2$.
\leanok

\textit{Proof sketch.}
$\zeta(s) = 0 \implies \xi(s) = 0$
(since $\Gamma_\mathbb{R}(s) \neq 0$ for nontrivial zeros)
$\implies \xi(1-s) = 0$ (functional equation $\xi(1-s) = \xi(s)$)
$\implies \zeta(1-s) = 0$.
\end{theorem}

\begin{theorem}[All nontrivial zeros lie on $\operatorname{Re}(s) = 1/2$]
\lean{DiscreteContainment.no_off_line_zeros}
\lean{DiscreteContainment.riemann_hypothesis}

Under the geometric coordination hypothesis, $\operatorname{Re}(s) = 1/2$
for every nontrivial zero.

Case split: $\operatorname{Re}(s) > 1/2$ is directly excluded by
\lean{zeta\_ne\_zero\_right\_half}; $\operatorname{Re}(s) < 1/2$ reflects
to $\operatorname{Re}(1-s) > 1/2$, also excluded.
\uses{DiscreteContainment.zeta_ne_zero_in_strip,
      DiscreteContainment.functional_equation_reflection}
\leanok
\end{theorem}

%% =========================================================================
\section{Discrete Containment Counterexample}
\label{sec:discrete-ce}
%% =========================================================================

\lean{DiscreteContainmentCE}

A zero-axiom demonstration that the three-pillar framework of
\S\ref{sec:discrete-containment} is \emph{vacuous}: each pillar is
unconditionally true for any $s$ with $\operatorname{Re}(s) > 1/2$.

\begin{definition}[Strip nonvanishing and three-pillar formulations]
\lean{DiscreteContainmentCE.StripNonvanishing},
\lean{DiscreteContainmentCE.ThreePillarNonvanishing}

$\mathrm{StripNonvanishing}$: $\forall s,\ 1/2 < \operatorname{Re}(s) < 1 \implies \zeta(s) \neq 0$.

$\mathrm{ThreePillarNonvanishing}$: the same statement with the three
pillar hypotheses pre-pended as explicit arguments.
\leanok
\end{definition}

\begin{theorem}[The three-pillar formulation is equivalent to bare strip nonvanishing]
\lean{DiscreteContainmentCE.three_pillars_are_vacuous}

$\mathrm{StripNonvanishing} \iff \mathrm{ThreePillarNonvanishing}$.
The three hypotheses are tautologically true and carry no mathematical
content.
\leanok

\textit{Proof sketch.}
Forward direction: given bare strip nonvanishing, ignore the three
hypotheses.  Backward direction: feed in the always-true hypotheses
(Pillar 1 holds for $\operatorname{Re}(s) > 0$, Pillar 2 for
$\sigma > 1/2$, Pillar 3 unconditionally).
\end{theorem}

\begin{theorem}[Pillars are independent of the imaginary part]
\lean{DiscreteContainmentCE.pillars_independent_of_imaginary_part}

For fixed $\sigma > 1/2$, all three pillar hypotheses hold at
$s = \sigma + it$ for \emph{every} $t \in \mathbb{R}$.
The pillars describe properties of the primes, not of $s$.
\leanok
\end{theorem}

\begin{remark}
The honest conclusion: $\zeta(s) \neq 0$ in the strip is a statement about the
analytic continuation of the Euler product, not about the product itself.
The Euler product diverges in the strip; the three pillars describe
convergence properties and cannot reach the continuation.  The missing
ingredient is the specific multiplicative structure of $\zeta$
(all Dirichlet coefficients equal 1).
\end{remark}

%% =========================================================================
\section{The Entangled Spiral Pair}
\label{sec:entangled-pair}
%% =========================================================================

\lean{EntangledPair}

The functional equation $\zeta(s) = \chi(s)\,\zeta(1-s)$ reveals the
Dirichlet series as a self-dual object.  The approximate functional
equation (AFE) makes this a finite computation:
\[
  \zeta(s) \approx S(s, X) + \chi(s)\,S(1-s, X), \qquad
  X \approx \sqrt{|t|/2\pi},
\]
where $S(s, X) = \sum_{n=1}^X n^{-s}$ is the primary spiral and
$S(1-s, X)$ is its reflection, coupled by the scalar $\chi(s)$.

On the critical line $\sigma = 1/2$, $|\chi| = 1$ (balanced pair).
Off the critical line $\sigma > 1/2$, $|\chi| < 1$ (second spiral
attenuated).  A zero $\zeta(s) = 0$ would require destructive
interference; but for $\sigma > 1/2$ the attenuation makes cancellation
impossible.

\subsection{The Spiral and Entangled Pair}

\begin{definition}[Spiral and entangled pair]
\lean{EntangledPair.S}, \lean{EntangledPair.E}

\[
  S(s, X) = \sum_{n=0}^{X-1} (n+1)^{-s}, \qquad
  E(s, \chi, X) = S(s, X) + \chi \cdot S(1-s, X).
\]
\leanok
\end{definition}

\begin{lemma}[Duality of the reflected spiral]
\lean{EntangledPair.dual_spiral}

$S(1-s, X) = \sum_{n=0}^{X-1} (n+1)^{s-1}$: the reflected spiral is the
same partial sum evaluated at the reflected point.
\leanok
\end{lemma}

\begin{theorem}[Dominance implies nonvanishing]
\lean{EntangledPair.E_ne_zero_of_dominance}

If $\|\chi \cdot S(1-s, X)\| < \|S(s, X)\|$, then $E(s, \chi, X) \neq 0$.
\leanok

\textit{Proof sketch.}
If $E = 0$ then $S(s,X) = -\chi S(1-s,X)$, so
$\|S(s,X)\| = \|\chi S(1-s,X)\|$, contradicting the dominance hypothesis.
\end{theorem}

\begin{lemma}[Reverse triangle inequality for $E$]
\lean{EntangledPair.E_norm_lower_bound}

$\|S(s,X)\| - \|\chi S(1-s,X)\| \leq \|E(s,\chi,X)\|$.
\leanok
\end{lemma}

\subsection{Regime 1: Real-Axis Nonvanishing ($t = 0$)}

\begin{theorem}[$\zeta(\sigma) < 0$ for $\sigma \in (0,1)$]
\lean{EntangledPair.zeta_neg_real}

For real $\sigma \in (0,1)$, $\operatorname{Re}(\zeta(\sigma)) < 0$.
\leanok

\textit{Proof sketch.}
Via the Euler--Maclaurin identity
$\zeta(\sigma) = \sigma/(\sigma-1) - \sigma \int_1^\infty \{t\}\, t^{-(\sigma+1)}\,dt$.
The first term is negative ($\sigma > 0$, $\sigma - 1 < 0$) and the
integral is nonneg (integrand is nonneg pointwise), so $\zeta(\sigma) \leq \sigma/(\sigma-1) < 0$.
This proof requires \lean{EulerMaclaurinDirichlet.c\_eq\_zeta} and is
fully constructive (zero custom axioms).
\end{theorem}

\begin{corollary}[$\zeta(s) \neq 0$ on the real axis in the strip]
\lean{EntangledPair.zeta_ne_zero_real}

For $1/2 < \operatorname{Re}(s) < 1$ and $\operatorname{Im}(s) = 0$:
$\zeta(s) \neq 0$.
\uses{EntangledPair.zeta_neg_real}
\leanok
\end{corollary}

\subsection{Regime 2: Off-Axis Nonvanishing ($t \neq 0$) --- Hypothesis Interfaces}

\begin{definition}[Geometric off-axis coordination hypothesis]
\lean{EntangledPair.GeometricOffAxisCoordinationHypothesis}

For every $\sigma \in (1/2, 1)$ and $t \neq 0$, there exist $\chi$,
$X \geq 1$, and $c > 0$ such that simultaneously:
\begin{itemize}
  \item $c + \|\chi S(1-s, X)\| \leq \|S(s, X)\|$ (dominance),
  \item $\|\zeta(s) - E(s, \chi, X)\| < c$ (AFE error bounded by gap).
\end{itemize}
\leanok
\end{definition}

\begin{theorem}[Nonvanishing from an AFE certificate]
\lean{EntangledPair.nonvanishing_of_afe_certificate_at}

If the geometric off-axis coordination hypothesis holds at $s$, then
$\zeta(s) \neq 0$.
\uses{EntangledPair.E_norm_lower_bound}
\leanok

\textit{Proof sketch.}
The dominance and error conditions at the same $X$ give
$c \leq \|E(s,\chi,X)\|$ and $\|\zeta(s) - E(s,\chi,X)\| < c$.
By the triangle inequality, $\zeta(s) = 0$ would force
$\|E\| < c$, a contradiction.
\end{theorem}

\begin{theorem}[Full strip nonvanishing from geometric coordination]
\lean{EntangledPair.strip_nonvanishing}
\lean{EntangledPair.zeta_ne_zero_right_half}

Under the geometric coordination hypothesis, $\zeta(s) \neq 0$ for
$\operatorname{Re}(s) > 1/2$.
\uses{EntangledPair.zeta_ne_zero_real,
      EntangledPair.nonvanishing_of_afe_certificate_at}
\leanok
\end{theorem}

\begin{theorem}[Riemann Hypothesis via entangled spiral pair]
\lean{EntangledPair.riemann_hypothesis}

Under the geometric coordination hypothesis, all nontrivial zeros of
$\zeta$ lie on $\operatorname{Re}(s) = 1/2$ (\lean{RiemannHypothesis}).
\uses{EntangledPair.zeta_ne_zero_right_half,
      DiscreteContainment.functional_equation_reflection}
\leanok

\textit{Proof sketch.}
Case split on $\operatorname{Re}(s)$: both $\operatorname{Re}(s) > 1/2$
and $\operatorname{Re}(s) < 1/2$ (via functional equation reflection)
are excluded by \lean{zeta\_ne\_zero\_right\_half}.
\end{theorem}

\subsection{Weyl Tube-Escape and Phase Drift Interfaces}

The following interface hierarchy decomposes the geometric coordination
hypothesis into modular analytical steps.

\begin{definition}[Interface hierarchy]
\lean{EntangledPair.OffAxisStripNonvanishingHypothesis},
\lean{EntangledPair.NoLongResonanceHypothesis},
\lean{EntangledPair.WeylTubeEscapeHypothesis},
\lean{EntangledPair.AlphaMinEscapeHypothesis},
\lean{EntangledPair.GaugePoleCompensatedDriftHypothesis},
\lean{EntangledPair.DecompositionWithTailEnvelopeHypothesis},
\lean{EntangledPair.PhaseDriftOnChosenWitnessHypothesis}

A layered hierarchy of progressively more analytic nonvanishing
conditions, from pointwise strip nonvanishing down to a local
gauge/pole-decomposed drift inequality.  Each layer implies the one
above.
\leanok
\end{definition}

\begin{theorem}[Equivalences in the hierarchy]
\lean{EntangledPair.weyl_tube_escape_iff_off_axis_strip_nonvanishing},
\lean{EntangledPair.weyl_tube_escape_iff_no_long_resonance}

Weyl tube-escape, off-axis strip nonvanishing, and no-long-resonance
are mutually equivalent; no-long-resonance implies them all.
\leanok
\end{theorem}

\begin{theorem}[Compact-interval minimum gives no-long-resonance]
\lean{EntangledPair.no_long_resonance_of_off_axis_strip_nonvanishing}

Pointwise off-axis strip nonvanishing, together with continuity of
$\zeta(\sigma + it)$ in $t$, yields a uniform positive lower bound
$\varepsilon$ on every compact interval $[t_0, t_0 + L]$.
\leanok
\end{theorem}

\begin{theorem}[$\alpha$-min escape from gauge/pole-compensated drift]
\lean{EntangledPair.alphaMin_escape_of_gauge_pole_compensated_drift}

Given a decomposition $\zeta = \text{core} + \text{wobble}$ with
$\delta + \|\text{wobble}\| \leq \|\text{core}\|$ whenever
$L \leq \delta / a_{\min}$, the $\alpha$-min escape hypothesis follows.
\leanok
\end{theorem}

\begin{theorem}[Zero-input decomposition+tail witness]
\lean{EntangledPair.decomposition_with_tail_envelope_zero}

The trivial decomposition $\text{core} = \zeta$, $\text{wobble} = 0$,
$\text{tailBound} = 0$, $a_{\min} = 1$ satisfies the
\lean{DecompositionWithTailEnvelopeHypothesis} unconditionally.
\leanok
\end{theorem}

%% =========================================================================
\section{Floor-Tail Decomposition of the Euler Product}
\label{sec:floor-tail}
%% =========================================================================

\lean{FloorTail}

The Euler product decomposes as $\zeta(s) = \text{FLOOR}(s,F) \times
\text{TAIL}(s,F)$, where $\text{FLOOR}$ is the finite product over a
set $F$ of small primes and $\text{TAIL}$ is the infinite product over
large primes.

\subsection{Euler Factor Bounds (Proved)}

\begin{theorem}[Upper bound on $|1 - p^{-s}|$]
\lean{FloorTail.euler_factor_norm_upper}

For prime $p$ and $\operatorname{Re}(s) > 0$:
$\|1 - p^{-s}\| \leq 1 + p^{-\sigma}$.
\leanok
\end{theorem}

\begin{theorem}[Lower bound on $|1 - p^{-s}|$]
\lean{FloorTail.euler_factor_norm_lower}

$1 - p^{-\sigma} \leq \|1 - p^{-s}\|$.
\leanok
\end{theorem}

\begin{theorem}[Lower bound on $|(1-p^{-s})^{-1}|$]
\lean{FloorTail.euler_factor_inv_lower}

$(1 + p^{-\sigma})^{-1} \leq \|(1 - p^{-s})^{-1}\|$.
\uses{FloorTail.euler_factor_norm_upper, PrimeBranching.euler_factor_ne_zero}
\leanok
\end{theorem}

\subsection{Finite Floor Lower Bound (Proved)}

\begin{definition}[Floor product]
\lean{FloorTail.floor}

$\text{FLOOR}(F, s) = \prod_{p \in F} (1 - p^{-s})^{-1}$.
\leanok
\end{definition}

\begin{theorem}[Floor lower bound]
\lean{FloorTail.floor_lower_bound}

$\prod_{p \in F} (1 + p^{-\sigma})^{-1} \leq \|\text{FLOOR}(F, s)\|$.

The lower bound is strictly positive (finite product of positive terms).
\uses{FloorTail.euler_factor_inv_lower}
\leanok
\end{theorem}

\subsection{Open Axioms}

\begin{axiom_block}[Tail lower bound]
\lean{FloorTail.tail_lower_bound}

For $\sigma > 1/2$, there exist $P$ and $C > 0$ such that
\[
  C \leq \prod^{\sharp}_{p > P} \|(1 - p^{-s})^{-1}\|
\]
for all $t \in \mathbb{R}$.
This is equivalent to the Riemann Hypothesis for $1/2 < \sigma < 1$
via the Euler product log-expansion.
\end{axiom_block}

\begin{axiom_block}[Anti-correlation principle (conjectural)]
\lean{FloorTail.anti_correlation_principle}

When the tail dips, the floor compensates: their product stays
above $R \cdot \min(\text{FLOOR}) \cdot \min(\text{TAIL})$ with
$R > 1$.  Supported by computational experiments
($R \approx 1.7$, $\operatorname{corr}(\log|\text{FLOOR}|,
\log|\text{TAIL}|) \approx -0.05$).
\end{axiom_block}

%% =========================================================================
\section{The Foundational Gap: RH Implies Spectral Dominance}
\label{sec:foundational-gap}
%% =========================================================================

\lean{FoundationalGap}

The Foundational Gap quantifies a structural asymmetry between spectral
methods (via RH and explicit formula) and algebraic methods (Baker's
theorem):
\begin{itemize}
  \item \textbf{Spectral.}  Under RH, $|\psi(x) - x| \leq C\sqrt{x}(\log x)^2$.
  \item \textbf{Algebraic.}  Baker: $|a\log 2 - b\log 3| \geq c/\max(a,b)^K$.
\end{itemize}
Spectral error is $O(\sqrt{x}(\log x)^2)$; Baker error is $O((\log x)^{-K})$.
The ratio diverges: $\sqrt{x} \gg (\log x)^K$ as $x \to \infty$.

\subsection{Rate Comparison (Proved via Mathlib)}

\begin{theorem}[$(\log x)^K = o(x^{1/2})$ for any $K$]
\lean{FoundationalGap.log_rpow_littleO_sqrt}

For every $K \in \mathbb{R}$, $(\log x)^K = o(x^{1/2})$ as $x \to \infty$.
\leanok

\textit{Proof.}  Direct application of Mathlib's
\lean{isLittleO\_log\_rpow\_rpow\_atTop}.
\end{theorem}

\subsection{Definition and Main Theorem}

\begin{definition}[Foundational Gap property]
\lean{FoundationalGap.FoundationalGapProp}

A structure bundling:
\begin{enumerate}
  \item $\exists C > 0$: $|\psi(x) - x| \leq C\sqrt{x}(\log x)^2$
        for all $x \geq 2$ (spectral rate, under RH),
  \item $\forall K$: $(\log x)^K = o(x^{1/2})$ (rate divergence, Mathlib).
\end{enumerate}
\leanok
\end{definition}

\begin{theorem}[RH implies the Foundational Gap]
\lean{FoundationalGap.rh_implies_foundational_gap}

If RH holds, then \lean{FoundationalGapProp} holds.
\uses{HadamardBridge.rh_implies_psi_error,
      FoundationalGap.log_rpow_littleO_sqrt}
\sorry

\textit{Proof sketch.}
Part (1) follows from \lean{HadamardBridge.rh\_implies\_psi\_error}
(which requires contour integration; the Lean proof uses \texttt{sorry}
for the contour integration step).
Part (2) is \lean{log\_rpow\_littleO\_sqrt}.
\end{theorem}

\begin{theorem}[Foundational Gap (unconditional, modulo axiom)]
\lean{FoundationalGap.foundational_gap}

Under the geometric off-axis coordination hypothesis:
\[
  \text{AFE coordination} \to \text{RH} \to \text{FoundationalGapProp}.
\]
\uses{EntangledPair.riemann_hypothesis,
      FoundationalGap.rh_implies_foundational_gap}
\leanok
\end{theorem}

\begin{theorem}[Spectral rate dominates Baker]
\lean{FoundationalGap.spectral_dominates_baker}

For any Baker exponent $K$ and under the geometric coordination
hypothesis, $(\log x)^K = o(x^{1/2})$.
\uses{FoundationalGap.foundational_gap}
\leanok
\end{theorem}

\begin{remark}[Function field comparison]
Over $\mathbb{F}_q[t]$, the Foundational Gap vanishes: RH is proved
(Weil 1949) and the Baker analogue (Mason--Stothers) gives polynomial
separation.  Both methods are polynomial in $q^n$.  The gap is specific
to $\mathbb{Z}$ and reflects the absence of a Frobenius endomorphism.
\end{remark}

%% =========================================================================
\section{AFE Coordination: Constructive Derivation}
\label{sec:afe-constructive}
%% =========================================================================

\lean{AFECoordinationConstructive}

This small module shows that the geometric off-axis coordination
hypothesis (\S\ref{sec:entangled-pair}) follows constructively from
off-axis strip nonvanishing, using the AFE infrastructure of
\lean{AFEInfrastructure}.

\begin{theorem}[AFE certificate from nonvanishing]
\lean{AFECoordinationConstructive.afe_certificate_at_of_nonvanishing}

If $\zeta(s) \neq 0$ for $1/2 < \operatorname{Re}(s) < 1$ and
$\operatorname{Im}(s) \neq 0$, then an AFE certificate at $s$ exists.
\uses{AFEInfrastructure.afe_coordination_of_ne_zero}
\leanok

\textit{Proof sketch.}  Apply the AFE coordination lemma from
\lean{AFEInfrastructure} to extract $\chi$, $X$, and a dominance gap;
set $c = \|S(s,X)\| - \|\chi S(1-s,X)\|$.
\end{theorem}

\begin{theorem}[Off-axis nonvanishing yields AFE certificate family]
\lean{AFECoordinationConstructive.afe_certificate_family_of_off_axis_nonvanishing}

If \lean{OffAxisStripNonvanishingHypothesis} holds, then an AFE
certificate exists at every off-axis strip point.
\uses{AFECoordinationConstructive.afe_certificate_at_of_nonvanishing}
\leanok
\end{theorem}

\begin{theorem}[Off-axis nonvanishing yields geometric coordination]
\lean{AFECoordinationConstructive.geometric_off_axis_coordination_of_off_axis_nonvanishing}

If \lean{OffAxisStripNonvanishingHypothesis} holds, then
\lean{GeometricOffAxisCoordinationHypothesis} holds.

\textit{Proof sketch.}  Assemble the certificate at each
$(s.re, s.im)$ pair using
\lean{afe\_certificate\_at\_of\_nonvanishing}.
\uses{AFECoordinationConstructive.afe_certificate_family_of_off_axis_nonvanishing}
\leanok
\end{theorem}

\begin{remark}
This module closes the logical loop: the geometric coordination
hypothesis is \emph{equivalent} to off-axis strip nonvanishing, and
the latter is just $\zeta(s) \neq 0$ off the real axis in the strip.
The full proof chain is therefore:
\[
  \text{off-axis strip nonvanishing}
  \iff \text{geometric coordination}
  \implies \text{RH}.
\]
\end{remark}

\end{document}
