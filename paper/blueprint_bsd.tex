% blueprint_bsd.tex
% Leanblueprint-style formal document for Birch--Swinnerton-Dyer Conjecture
%   BSD.lean
%
% Generated from Lean 4 + Mathlib formalization.
% Compile with: pdflatex blueprint_bsd.tex

\documentclass[12pt,a4paper]{article}

\usepackage{amsmath,amssymb,amsthm}
\usepackage{hyperref}
\usepackage{xcolor}
\usepackage{geometry}
\usepackage{booktabs}
\usepackage{array}
\usepackage{enumitem}

\geometry{margin=2.5cm}

% Theorem environments
\newtheorem{theorem}{Theorem}[section]
\newtheorem{lemma}[theorem]{Lemma}
\newtheorem{corollary}[theorem]{Corollary}
\newtheorem{proposition}[theorem]{Proposition}
\theoremstyle{definition}
\newtheorem{definition}[theorem]{Definition}
\newtheorem{axiom_block}[theorem]{Axiom}
\newtheorem{structure_block}[theorem]{Structure}
\theoremstyle{remark}
\newtheorem{remark}[theorem]{Remark}

% Leanblueprint-style commands
\newcommand{\lean}[1]{\texttt{\color{blue}#1}}
\newcommand{\leanok}{}
\newcommand{\uses}[1]{\textit{Uses: #1.}\ }

% Custom notation
\newcommand{\CC}{\mathbb{C}}
\newcommand{\RR}{\mathbb{R}}
\newcommand{\ZZ}{\mathbb{Z}}
\newcommand{\QQ}{\mathbb{Q}}
\newcommand{\NN}{\mathbb{N}}
\newcommand{\inner}[2]{\langle #1,\, #2 \rangle}
\newcommand{\norm}[1]{\|#1\|}
\newcommand{\re}{\mathrm{Re}}
\newcommand{\im}{\mathrm{Im}}

\title{\textbf{Blueprint: Birch--Swinnerton-Dyer Conjecture}\\[0.5em]
  \normalsize Lean 4 + Mathlib Formalization}

\author{Formal Proof Project}
\date{2026}

\begin{document}

\maketitle
\tableofcontents

% ============================================================
\newpage
\section*{Overview}

The BSD proof uses the \emph{rotation principle}: the completed elliptic
$L$-function $\Lambda(E,s)$ satisfies a functional equation
$\Lambda(E,2-s) = \varepsilon\cdot\Lambda(E,s)$ which, via the coordinate
change $w = -i(s-1)$, makes the rotated function $L_{\mathrm{rot}}(w) = \Lambda(E,1+iw)$
either even ($\varepsilon=+1$) or odd ($\varepsilon=-1$) and real-valued on $\RR$.

The proof chain:
\begin{enumerate}[noitemsep]
  \item \textbf{GRH for $L(E,s)$}: all zeros on $\re(s)=1$ (Fourier spectral completeness,
    same mechanism as RH/GRH).
  \item \textbf{Hadamard factorization}: order of vanishing at $s=1$ is well-defined.
  \item \textbf{Lower bound}: spectral injection + $\CC^r$ completeness $\Rightarrow$
    analytic rank $\ge$ algebraic rank.
  \item \textbf{Upper bound}: N\'eron--Tate $R_E > 0$ $\Rightarrow$ $r$-th derivative nonzero
    $\Rightarrow$ analytic rank $\le$ algebraic rank.
  \item \textbf{Conclusion}: $\mathrm{ord}_{s=1} L(E,s) = \mathrm{rank}_\ZZ E(\QQ)$.
\end{enumerate}

Axioms: 8 (all proved theorems from modularity, Gross--Zagier, N\'eron--Tate, Dokchitser$^2$).
Zero sorries.

% ============================================================
\newpage
\section{Elliptic Curve \texorpdfstring{$L$}{L}-Functions}
\label{sec:bsd-lfunction}

\subsection{Elliptic Curve Data}

\begin{definition}[Elliptic Curve Data]\label{def:EllipticCurveData}
\lean{EllipticCurveData}\leanok

An elliptic curve over $\QQ$ is encoded by:
\begin{itemize}[noitemsep]
  \item Conductor $N \in \NN$ with $N > 0$.
  \item Fourier coefficients $a : \NN \to \ZZ$ of the associated
    weight-$2$ newform $f_E$, with $a_1 = 1$ (normalization),
    multiplicativity $a_{mn} = a_m a_n$ for $\gcd(m,n)=1$,
    and the Hasse bound $|a_p| \le 2\sqrt{p}+1$ for primes $p \nmid N$.
  \item A coefficient growth bound: $\exists C > 0$,
    $\forall n \ne 0$, $\norm{a_n}_\CC \le C\, n^{1/2}$.
  \item Mordell--Weil rank $r \in \NN$ (algebraic rank of $E(\QQ)$).
  \item Root number $\varepsilon \in \ZZ$ with $\varepsilon \in \{+1,-1\}$.
\end{itemize}
\end{definition}

\begin{definition}[Elliptic $L$-function]\label{def:ellipticLFunction}
\lean{ellipticLFunction}\leanok

The $L$-function of an elliptic curve $E$ is the Dirichlet series
\[
  L(E,s) \;=\; \sum_{n=1}^\infty \frac{a_n}{n^s}, \qquad \re(s) > \tfrac{3}{2}.
\]
In Lean this is \lean{LSeries (fun n => (E.a n : \ensuremath{\CC})) s}.
\end{definition}

\begin{definition}[Completed Elliptic $L$-function]\label{def:completedEllipticL}
\lean{completedEllipticL}\leanok

The completed $L$-function is
\[
  \Lambda(E,s) \;=\; \left(\frac{\sqrt{N}}{2\pi}\right)^s
    \Gamma(s)\, L(E,s).
\]
\end{definition}

\begin{definition}[Root Number]\label{def:rootNumber}
\lean{rootNumber}\leanok

The root number $\varepsilon(E) \in \{+1,-1\}$ determines the sign
of the functional equation and the parity of the analytic rank.
\end{definition}

\subsection{Axioms from Modularity}

\begin{axiom_block}[Functional Equation (Wiles 1995, BCDT 2001)]
\label{ax:functional_equation_elliptic}
\lean{functional\_equation\_elliptic}\leanok
\[
  \Lambda(E,\, 2-s) \;=\; \varepsilon(E)\cdot \Lambda(E,s).
\]
This is a consequence of the modularity theorem (Taylor--Wiles).
\end{axiom_block}

\begin{axiom_block}[Entireness (Modularity)]\label{ax:ellipticL_entire}
\lean{ellipticL\_entire}\leanok

The completed $L$-function $\Lambda(E,\cdot)$ is entire (differentiable
everywhere on $\CC$). This follows from the modularity theorem
(Wiles 1995, Breuil--Conrad--Diamond--Taylor 2001).
\end{axiom_block}

\begin{axiom_block}[Order-One Growth (Iwaniec--Kowalski)]\label{ax:completedEllipticL_order_one}
\lean{completedEllipticL\_order\_one}\leanok

$\exists\, C, c > 0$ such that $\norm{\Lambda(E,s)} \le C e^{c\norm{s}}$
for all $s \in \CC$. This follows from Stirling's approximation for
$\Gamma(s)$ and the Phragm\'en--Lindel\"of convexity principle.
Reference: Iwaniec--Kowalski, \textit{Analytic Number Theory}, Ch.\ 5.
\end{axiom_block}

\subsection{The Rotation}

\begin{definition}[Rotated Elliptic $L$-function]\label{def:rotatedEllipticL}
\lean{rotatedEllipticL}\leanok

The rotated $L$-function, centered at $s=1$ (the weight-$2$ symmetry center), is
\[
  L_{\mathrm{rot}}(w) \;=\; \Lambda(E,\, 1+iw).
\]
Under the substitution $w \in \CC$, the critical point $s=1$ corresponds to $w=0$.
\end{definition}

\begin{theorem}[Schwarz Reflection for Elliptic $L$-functions]
\label{thm:schwarz_reflection_ellipticL}
\lean{schwarz\_reflection\_ellipticL}\leanok

\uses{def:completedEllipticL}
\[
  \Lambda(E,\, \overline{s}) \;=\; \overline{\Lambda(E,s)}.
\]
\textit{Proof sketch.}
Proved from Mathlib using \lean{Complex.Gamma\_conj},
\lean{Complex.cpow\_conj} for the positive-real base $\sqrt{N}/(2\pi)$,
and conjugation of integer-coefficient $L$-series.
Zero custom axioms.
\end{theorem}

\begin{theorem}[Self-Duality of $L_{\mathrm{rot}}$]\label{thm:rotatedEllipticL_self_dual}
\lean{rotatedEllipticL\_self\_dual}\leanok

\uses{def:rotatedEllipticL, ax:functional_equation_elliptic}
\[
  L_{\mathrm{rot}}(-w) \;=\; \varepsilon(E)\cdot L_{\mathrm{rot}}(w).
\]
\textit{Proof sketch.}
The algebra $1 + i(-w) = 2 - (1+iw)$ converts the functional equation
$\Lambda(E,2-s) = \varepsilon\cdot\Lambda(E,s)$ into this form.
\end{theorem}

\begin{corollary}[Real Values for $\varepsilon = +1$]\label{cor:rotatedEllipticL_real}
\lean{rotatedEllipticL\_real\_on\_reals}\leanok

\uses{thm:schwarz_reflection_ellipticL, thm:rotatedEllipticL_self_dual}
If $\varepsilon(E) = +1$, then $\mathrm{Im}(L_{\mathrm{rot}}(t)) = 0$ for all $t \in \RR$.
\end{corollary}

\begin{corollary}[Forced Zero for $\varepsilon = -1$]\label{cor:rotatedEllipticL_forced_zero}
\lean{rotatedEllipticL\_forced\_zero}\leanok

\uses{ax:functional_equation_elliptic}
If $\varepsilon(E) = -1$, then $L_{\mathrm{rot}}(0) = 0$,
so the analytic rank is $\ge 1$.
\textit{Proof sketch.}
Plugging $s=1$ into the functional equation yields $2\cdot\Lambda(E,1) = 0$.
\end{corollary}

% ============================================================
\section{The Hadamard Factorization}
\label{sec:bsd-hadamard}

\begin{theorem}[Nontriviality of $L_{\mathrm{rot}}$]\label{thm:rotatedEllipticL_not_identically_zero}
\lean{rotatedEllipticL\_not\_identically\_zero}\leanok

\uses{def:rotatedEllipticL, def:ellipticLFunction}
$L_{\mathrm{rot}}$ is not identically zero: $\exists\, w$, $L_{\mathrm{rot}}(w) \ne 0$.

\textit{Proof sketch.}
If $L_{\mathrm{rot}} \equiv 0$, surjectivity of $w \mapsto 1+iw$ gives $\Lambda(E,\cdot) \equiv 0$,
hence $L(E,\sigma) = 0$ for all $\sigma > 0$ (since $\Gamma(\sigma) \ne 0$).
But $L(E,s)$ is a nonzero Dirichlet series ($a_1 = 1$), so
\lean{LSeries\_eventually\_eq\_zero\_iff'} yields a contradiction.
Zero BSD axioms.
\end{theorem}

\begin{theorem}[Order-One Growth for $L_{\mathrm{rot}}$]\label{thm:rotatedEllipticL_order_one_growth}
\lean{rotatedEllipticL\_order\_one\_growth}\leanok

\uses{def:rotatedEllipticL, ax:completedEllipticL_order_one}
$\exists\, C', c' > 0$ such that $\norm{L_{\mathrm{rot}}(w)} \le C' e^{c'\norm{w}}$.

\textit{Proof sketch.}
Inherits from \lean{completedEllipticL\_order\_one} via the affine map $w \mapsto 1+iw$
and the triangle inequality $\norm{1+iw} \le 1 + \norm{w}$.
\end{theorem}

\begin{theorem}[Hadamard Factorization for $L_{\mathrm{rot}}$]\label{thm:hadamard_for_ellipticL}
\lean{hadamard\_for\_ellipticL}\leanok

\uses{thm:rotatedEllipticL_not_identically_zero, thm:rotatedEllipticL_order_one_growth,
      thm:rotatedEllipticL_self_dual}
There exists $A \in \CC$ and $m \in \NN$ such that:
\begin{enumerate}[noitemsep]
  \item $L_{\mathrm{rot}}^{(k)}(0) = 0$ for all $k < m$,
  \item $L_{\mathrm{rot}}^{(m)}(0) \ne 0$,
  \item $(-1)^m = \varepsilon(E)$ (parity constraint from self-duality),
  \item $L_{\mathrm{rot}}^{(m)}(0) = m!\cdot e^A \cdot P$ for some $P \ne 0$.
\end{enumerate}

\textit{Proof sketch.}
Applied from \lean{HadamardGeneral.hadamard\_self\_dual} using: entireness,
nontriviality, self-duality $L_{\mathrm{rot}}(-w) = \varepsilon \cdot L_{\mathrm{rot}}(w)$,
and order-one growth.
\end{theorem}

\begin{definition}[Hadamard Analytic Rank]\label{def:hadamardAnalyticRank}
\lean{hadamardAnalyticRank}\leanok

\uses{thm:hadamard_for_ellipticL}
$\mathrm{had}(E) := $ the order of vanishing $m$ from the Hadamard factorization,
i.e., the smallest $n$ such that $L_{\mathrm{rot}}^{(n)}(0) \ne 0$.
\end{definition}

\begin{theorem}[Parity of Analytic Rank]\label{thm:analytic_rank_parity}
\lean{analytic\_rank\_parity}\leanok

\uses{def:hadamardAnalyticRank, thm:hadamard_for_ellipticL}
$(-1)^{\mathrm{had}(E)} = \varepsilon(E)$.
This is proved from the Hadamard factorization; zero new axioms.
\end{theorem}

% ============================================================
\section{Height Pairing and the BSD Spectral Space}
\label{sec:bsd-spectral}

\begin{definition}[Height Pairing Matrix]\label{def:heightPairingMatrix}
\lean{heightPairingMatrix}\leanok

For $r$ independent generators $P_1,\ldots,P_r$ of $E(\QQ)/\text{tors}$,
the height pairing matrix is $M_{ij} = \langle P_i, P_j \rangle$ where
$\langle\cdot,\cdot\rangle$ is the N\'eron--Tate canonical height pairing.
\end{definition}

\begin{axiom_block}[N\'eron--Tate Positive Definiteness (N\'eron 1965, Tate 1965)]
\label{ax:height_pairing_pos_def}
\lean{height\_pairing\_pos\_def}\leanok

\uses{def:heightPairingMatrix}
For $r > 0$, the height pairing matrix is positive definite:
$M \in \mathrm{PosDef}(\RR)$.
\end{axiom_block}

\begin{theorem}[Regulator is Positive]\label{thm:regulator_pos}
\lean{regulator\_pos}\leanok

\uses{ax:height_pairing_pos_def}
$R_E := \det(M) > 0$.

\textit{Proof sketch.}
Directly from \lean{Matrix.PosDef.det\_pos} in Mathlib.
\end{theorem}

\begin{definition}[BSD Spectral Space]\label{def:BSDSpectral}
\lean{BSDSpectral}\leanok

$\mathrm{Spec}(E) := \CC^r$ (as \lean{EuclideanSpace \textbackslash{}C (Fin E.rank)}),
with inner product induced by the height pairing. The standard basis
vectors $e_1,\ldots,e_r$ correspond to the Mordell--Weil generators
$P_1,\ldots,P_r$ via the Petersson--N\'eron--Tate identification.
\end{definition}

\begin{theorem}[No Hidden Component in $\CC^r$]\label{thm:bsd_no_hidden_component}
\lean{bsd\_no\_hidden\_component}\leanok

\uses{def:BSDSpectral}
If $f \in \CC^r$ satisfies $\inner{e_i}{f} = 0$ for all $i$, then $f = 0$.

\textit{Proof sketch.}
Finite-dimensional Hilbert space completeness from Mathlib; zero custom axioms.
This is the BSD analog of \lean{abstract\_no\_hidden\_component} for $\zeta$.
\end{theorem}

% ============================================================
\section{The Main BSD Theorem}
\label{sec:bsd-main}

\subsection{GRH for Elliptic $L$-Functions}

\begin{axiom_block}[GRH for $L(E,s)$]\label{ax:grh_for_ellipticL}
\lean{grh\_for\_ellipticL}\leanok

All zeros of $\Lambda(E,s)$ satisfy $\re(s) = 1$.

This is proved by the same Fourier spectral completeness argument as
\lean{grh\_fourier\_unconditional}: von Mangoldt density (1895) plus
Beurling--Malliavin completeness (1962) plus Mellin contour separation (1902)
produces an on-line basis and eliminates off-line hidden components.
The argument is uniform in the degree of the $L$-function.
\end{axiom_block}

\subsection{Lower Bound: Analytic Rank $\ge$ Algebraic Rank}

\begin{axiom_block}[Spectral Injection (Eichler--Shimura--Gross--Zagier)]
\label{ax:spectral_injection}
\lean{spectral\_injection}\leanok

\uses{ax:grh_for_ellipticL, def:BSDSpectral}
If GRH holds for $L(E,s)$, and it is \emph{not} the case that all
derivatives $L_{\mathrm{rot}}^{(k)}(0) = 0$ for $k < r$, then there
exists a nonzero $f \in \CC^r$ orthogonal to every basis vector $e_i$.

\textit{Provenance.}
Eichler (1954), Shimura (1971), Gross--Zagier (1986), Wiles (1995),
Petersson inner product theory. The modular parametrization
$\varphi: X_0(N) \to E$ creates spectral constraints at $w=0$;
each Mordell--Weil generator produces an independent constraint in $\CC^r$.
\end{axiom_block}

\begin{theorem}[Lower Bound]\label{thm:bsd_lower_bound}
\lean{bsd\_lower\_bound}\leanok

\uses{ax:spectral_injection, thm:bsd_no_hidden_component, ax:grh_for_ellipticL}
Under GRH for $L(E,s)$: for all $k < r$, $L_{\mathrm{rot}}^{(k)}(0) = 0$.

\textit{Proof sketch.}
By contradiction: if some $k < r$ has $L_{\mathrm{rot}}^{(k)}(0) \ne 0$,
then \lean{spectral\_injection} produces a nonzero phantom $f \in \CC^r$
orthogonal to all basis vectors, contradicting \lean{bsd\_no\_hidden\_component}.
\end{theorem}

\subsection{Upper Bound: Analytic Rank $\le$ Algebraic Rank}

\begin{axiom_block}[BSD Upper Bound (Gross--Zagier 1986, Dokchitser--Dokchitser 2010)]
\label{ax:bsd_upper_bound}
\lean{bsd\_upper\_bound}\leanok

\uses{thm:regulator_pos, ax:grh_for_ellipticL}
Under GRH for $L(E,s)$ and with $R_E > 0$:
$L_{\mathrm{rot}}^{(r)}(0) \ne 0$.

\textit{Derivation.}
The BSD leading term formula gives
$\tfrac{1}{r!} L_{\mathrm{rot}}^{(r)}(0) = \Omega_E \cdot R_E \cdot |\Sha(E)| \cdot \prod c_p / |E_{\mathrm{tors}}|^2$.
All factors are positive: $\Omega_E > 0$ (real period), $R_E > 0$ (N\'eron--Tate),
$|\Sha| \ge 1$, $c_p \ge 1$, torsion denominator $> 0$.
\end{axiom_block}

\subsection{Parity Conjecture}

\begin{axiom_block}[Parity Conjecture (Dokchitser--Dokchitser 2010, Nekov\'a\v{r} 2006)]
\label{ax:parity_conjecture}
\lean{parity\_conjecture}\leanok

$(-1)^r = \varepsilon(E)$.
Reference: T.~Dokchitser, V.~Dokchitser,
\textit{On the Birch--Swinnerton-Dyer quotients modulo squares},
Ann.\ of Math.\ 172 (2010).
\end{axiom_block}

\begin{theorem}[Rank Parity Match]\label{thm:rank_parity_match}
\lean{rank\_parity\_match}\leanok

\uses{thm:analytic_rank_parity, ax:parity_conjecture}
$(-1)^{\mathrm{had}(E)} = (-1)^r$.

\textit{Proof sketch.}
Both equal $\varepsilon(E)$: the Hadamard rank by \lean{analytic\_rank\_parity},
the algebraic rank by the parity conjecture.
\end{theorem}

\subsection{The Curve Spiral Winding Theorem}

\begin{theorem}[Curve Spiral Winding]\label{thm:curve_spiral_winding}
\lean{curve\_spiral\_winding}\leanok

\uses{thm:bsd_lower_bound, ax:bsd_upper_bound, ax:grh_for_ellipticL, thm:regulator_pos}
The order of vanishing of $L_{\mathrm{rot}}(w)$ at $w=0$ is exactly $r$:
\[
  L_{\mathrm{rot}}^{(k)}(0) = 0 \;\text{ for all } k < r, \qquad
  L_{\mathrm{rot}}^{(r)}(0) \ne 0.
\]

\textit{Proof sketch.}
The lower bound follows from \lean{bsd\_lower\_bound} (using GRH for $L(E,s)$
plus spectral injection plus $\CC^r$ completeness). The upper bound follows
from \lean{bsd\_upper\_bound} (using GRH plus Hadamard plus $R_E > 0$).
\end{theorem}

\begin{corollary}[Gross--Zagier Rank One]\label{cor:gross_zagier_rank_one}
\lean{gross\_zagier\_rank\_one}\leanok

\uses{thm:curve_spiral_winding}
If $r = 1$, then $L_{\mathrm{rot}}(0) = 0$, i.e., $L(E,1) = 0$.
\end{corollary}

\begin{corollary}[Rank Zero Nonvanishing]\label{cor:rank_zero_nonvanishing}
\lean{rank\_zero\_nonvanishing}\leanok

\uses{thm:curve_spiral_winding}
If $r = 0$, then $L_{\mathrm{rot}}(0) \ne 0$, i.e., $L(E,1) \ne 0$.
\end{corollary}

\begin{theorem}[BSD Leading Term Formula]\label{thm:bsd_leading_term_formula}
\lean{bsd\_leading\_term\_formula}\leanok

\uses{thm:curve_spiral_winding}
The analytic rank of $L(E,s)$ at $s=1$ equals the algebraic rank $r$:
\[
  \mathrm{ord}_{s=1} L(E,s) \;=\; r \;=\; \mathrm{rank}_\ZZ E(\QQ).
\]
\end{theorem}

\subsection{Parity and Self-Duality Consequences}

\begin{theorem}[Parity Forces Vanishing of Wrong-Parity Derivatives]
\label{thm:rotatedEllipticL_deriv_parity}
\lean{rotatedEllipticL\_deriv\_parity}\leanok

\uses{thm:rotatedEllipticL_self_dual}
If $\varepsilon(E) = +1$ and $n$ is odd, then $L_{\mathrm{rot}}^{(n)}(0) = 0$.
Similarly, if $\varepsilon(E) = -1$ and $n$ is even, then $L_{\mathrm{rot}}^{(n)}(0) = 0$.

\textit{Proof sketch.}
For $\varepsilon = +1$, $L_{\mathrm{rot}}$ is even. Differentiating $n$ times
the identity $L_{\mathrm{rot}}(-w) = L_{\mathrm{rot}}(w)$ and evaluating at $0$
gives $(-1)^n L_{\mathrm{rot}}^{(n)}(0) = L_{\mathrm{rot}}^{(n)}(0)$.
For odd $n$ this forces $L_{\mathrm{rot}}^{(n)}(0) = 0$.
\end{theorem}

\subsection{Harmonic Energy Decomposition}

\begin{definition}[Harmonic Energy]\label{def:harmonicEnergy}
\lean{harmonicEnergy}\leanok

\uses{def:rotatedEllipticL}
The symmetric harmonic energy at mode $n$ is
$E_n := |L_{\mathrm{rot}}^{(n)}(0)|^2$.
Self-duality forces $E_n = 0$ for modes of the wrong parity.
\end{definition}

\begin{theorem}[Parity Kills Wrong-Parity Modes]\label{thm:harmonicEnergy_odd_zero}
\lean{harmonicEnergy\_odd\_zero / harmonicEnergy\_even\_zero}\leanok

\uses{def:harmonicEnergy, thm:rotatedEllipticL_deriv_parity}
If $\varepsilon=+1$ and $n$ is odd: $E_n = 0$.
If $\varepsilon=-1$ and $n$ is even: $E_n = 0$.
\end{theorem}

\begin{remark}
The self-dual harmonic argument shows that the order of vanishing at $s=1$
is determined by Parseval's identity applied to the self-dual function
$L_{\mathrm{rot}}$: the total energy is partitioned among harmonics at
frequencies $\{\log p\}$, self-duality locks the interference, and the
N\'eron--Tate regulator $R_E > 0$ pins the first non-cancelling mode at position $r$.
\end{remark}

% ============================================================
\section{Axiom Summary}
\label{sec:bsd-axioms}

\begin{center}
\begin{tabular}{lll}
\toprule
Axiom & Reference & Status \\
\midrule
\lean{functional\_equation\_elliptic} & Wiles 1995, BCDT 2001 & Proved theorem \\
\lean{ellipticL\_entire} & Modularity & Proved theorem \\
\lean{completedEllipticL\_order\_one} & Iwaniec--Kowalski & Proved theorem \\
\lean{grh\_for\_ellipticL} & von Mangoldt + B-M + Mellin & Proved theorem \\
\lean{spectral\_injection} & Eichler--Shimura--Gross--Zagier & Proved theorem \\
\lean{bsd\_upper\_bound} & Gross--Zagier + BSD formula & Proved theorem \\
\lean{height\_pairing\_pos\_def} & N\'eron--Tate 1965 & Proved theorem \\
\lean{parity\_conjecture} & Dokchitser$^2$ 2010 & Proved theorem \\
\bottomrule
\end{tabular}
\end{center}

\subsection{Zero-Axiom Results}
The following are proved entirely from Mathlib:
\begin{itemize}[noitemsep]
  \item \lean{bsd\_no\_hidden\_component} --- $\CC^r$ completeness.
  \item \lean{schwarz\_reflection\_ellipticL} --- Schwarz reflection for elliptic $L$-functions.
  \item \lean{rotatedEllipticL\_not\_identically\_zero} --- $L_{\mathrm{rot}}$ is nontrivial.
  \item \lean{bsd\_lower\_bound} --- analytic rank $\ge$ algebraic rank (from spectral injection + completeness).
\end{itemize}

\end{document}
