% ============================================================
%  Blueprint: No-Divergence Half of the Collatz Proof
%  Files: WeylBridge.lean, NoDivergence.lean,
%         NoDivergenceMixing.lean, 1135.lean, Collatz.lean
% ============================================================
\documentclass[11pt, a4paper]{article}

\usepackage{amsmath, amssymb, amsthm}
\usepackage{hyperref}
\usepackage{cleveref}
\usepackage{enumitem}
\usepackage{geometry}
\geometry{margin=2.5cm}

% ------ leanblueprint-style environments ------
\newtheoremstyle{lbstyle}{6pt}{6pt}{\itshape}{0pt}{\bfseries}{.}{0.5em}{}
\theoremstyle{lbstyle}
\newtheorem{theorem}{Theorem}[section]
\newtheorem{lemma}[theorem]{Lemma}
\newtheorem{corollary}[theorem]{Corollary}
\theoremstyle{definition}
\newtheorem{definition}[theorem]{Definition}
\newtheorem{structure}[theorem]{Structure}
\newtheorem{axiom_block}[theorem]{Axiom}
\newtheorem{remark}[theorem]{Remark}
\newtheorem{notation}[theorem]{Notation}

% \lean{} and \leanok annotations (informational)
\newcommand{\lean}[1]{\quad\texttt{[Lean:} \texttt{#1}\texttt{]}}
\newcommand{\leanok}{\quad\checkmark}
\newcommand{\uses}[1]{\medskip\noindent\textit{Uses:} #1\medskip}

\title{Blueprint: No-Divergence Half of the Collatz Proof\\
  \large \texttt{WeylBridge}, \texttt{NoDivergence}, \texttt{NoDivergenceMixing},
  \texttt{1135}, and the root \texttt{Collatz} module}
\author{Extracted from the Lean~4 Formalization}
\date{2026}

\begin{document}
\maketitle
\tableofcontents
\newpage

%% ============================================================
\section{Overview and Proof Architecture}
%% ============================================================

The Collatz conjecture asserts that every positive integer eventually reaches~$1$
under iteration of the map
\[
  T(n) = \begin{cases} n/2 & \text{if } n \text{ is even,} \\ 3n+1 & \text{if } n \text{ is odd.} \end{cases}
\]
The formalization reduces the conjecture to two independent obstructions:
\begin{enumerate}
  \item \textbf{No nontrivial cycles} (handled in \texttt{NoCycle.lean} and related files, not detailed here).
  \item \textbf{No divergence}: no odd orbit is unbounded.
\end{enumerate}

This blueprint documents the no-divergence half.  The proof chain is:

\medskip
\noindent\textbf{Axiom} \texttt{baker\_rollover\_supercritical\_rate}
$\;\Longrightarrow\;$ \textbf{Supercritical $\nu$-sum} ($\Sigma\nu \ge 33$ per 20 steps)
$\;\Longrightarrow\;$ \textbf{Quantitative contraction} ($3^{20}/2^{33} \approx 0.406 < 1$)
$\;\Longrightarrow\;$ \textbf{Bounded orbits}
$\;\Longrightarrow\;$ \textbf{No divergence}.

\medskip
\noindent In parallel, a constructive bridge (\texttt{baker\_tao\_supercritical} in
\texttt{WeylBridge.lean}) converts the supercritical rate into
residue-hitting (\texttt{drift\_crossing\_from\_baker}), which is used via
the $M=2$ mixing route in \texttt{NoDivergence.lean} to deliver the final contradiction
for any hypothetical divergent orbit.

\medskip
\noindent\textbf{Critical-path axioms} (both are formalized statements of proved literature
results, not conjectures):
\begin{itemize}
  \item \texttt{baker\_rollover\_supercritical\_rate} --- Baker (1968) coprimality:
        $D = 2^S - 3^m$ is always odd, preventing systematic avoidance of
        high-$\nu_2$ residue classes; guarantees $8W+\delta \le 5\sum\eta_i$.
  \item \texttt{supercritical\_rate\_implies\_residue\_hitting} ---
        constructive bridge: supercritical rate implies every residue class
        mod $M > 1$ is hit after any cutoff $K$.
\end{itemize}

%% ============================================================
\section{Supporting Definitions from \texttt{CycleEquation} and \texttt{ResidueDynamics}}
%% ============================================================

Several definitions originate in upstream modules.  We record them here for
self-containedness.

\begin{notation}[Syracuse odd iterate]\label{not:collatzOddIter}\lean{CycleEquation.collatzOddIter}\leanok
  The \emph{Syracuse odd iterate} $T^{(m)}(n_0)$ is the $m$-fold composition of the
  odd step
  \[
    T_{\text{odd}}(n) = \frac{3n+1}{2^{\nu_2(3n+1)}}
  \]
  applied to the odd starting value $n_0$.  In the Lean formalization this is
  \texttt{collatzOddIter m n₀}.  All iterates of an odd positive integer remain odd
  and positive.
\end{notation}

\begin{notation}[2-adic valuation]\label{not:v2}\lean{CycleEquation.v2}\leanok
  $\nu_2(k)$ denotes the 2-adic valuation of $k$, i.e.\ the largest power of~$2$
  dividing~$k$.  Written \texttt{v2 k} in Lean.
\end{notation}

\begin{notation}[Orbit $\nu$-sum and carry]\label{not:orbitS}\lean{CycleEquation.orbitS}\leanok
  For an odd starting value $x$ and step count $k$, define
  \begin{align*}
    S_k(x) &:= \sum_{j=0}^{k-1} \nu_2\!\left(3 T_{\text{odd}}^{(j)}(x)+1\right),
    & &\text{(\texttt{orbitS x k})} \\
    C_k(x) &:= \sum_{j=0}^{k-1} 3^{k-1-j} \cdot 2^{S_j(x)},
    & &\text{(\texttt{orbitC x k})}
  \end{align*}
  so that the orbit-iteration formula reads
  $T_{\text{odd}}^{(k)}(x) \cdot 2^{S_k(x)} = 3^k \cdot x + C_k(x)$.
\end{notation}

\begin{definition}[Residue envelope $\eta$]\label{def:etaResidue}
  \lean{ResidueDynamics.etaResidue}\leanok
  For a natural number $n$, define the \emph{residue envelope}
  \[
    \eta(n) := \begin{cases}
      2 & \text{if } n \equiv 1 \pmod{8}, \\
      3 & \text{if } n \equiv 5 \pmod{8}, \\
      1 & \text{otherwise.}
    \end{cases}
  \]
  This is a lower bound for $\nu_2(3n+1)$ when $n$ is odd.
\end{definition}

\begin{lemma}[Residue envelope bound]\label{lem:etaResidue_le_v2}
  \lean{ResidueDynamics.etaResidue\_le\_v2\_of\_odd}\leanok
  \uses{def:etaResidue, not:v2}
  For every odd $n$,
  \[
    \eta(n) \le \nu_2(3n+1).
  \]
  \textit{Proof sketch.} Direct case analysis on $n \bmod 8$: if $n \equiv 1 \pmod 8$
  then $8 \mid 3n+1$ gives $\nu_2 \ge 2$; if $n \equiv 5 \pmod 8$ then $8 \mid 3n+1$
  gives $\nu_2 \ge 3$; otherwise $2 \mid 3n+1$ gives $\nu_2 \ge 1$.
\end{lemma}

%% ============================================================
\section{Axioms: Baker Rollover and the Supercritical Bridge}
%% ============================================================

\begin{axiom_block}[Baker rollover supercritical rate]\label{ax:baker_rollover}
  \lean{NumberTheoryAxioms.baker\_rollover\_supercritical\_rate}\leanok
  Let $n_0 > 1$ be odd and suppose the Syracuse orbit of $n_0$ is
  divergent, i.e.\ $\forall B,\;\exists m,\; T_{\text{odd}}^{(m)}(n_0) > B$.
  Then there exist $M_0, \delta \in \mathbb{N}$ with $\delta > 0$ such that
  for all $M \ge M_0$ and $W \ge 20$,
  \[
    8W + \delta \;\le\; 5 \sum_{i=0}^{W-1} \eta\!\left(T_{\text{odd}}^{(M+i)}(n_0)\right).
  \]
  \emph{Justification.} Baker (1968): the integer $D = 2^S - 3^m$ is always odd
  (coprimality), which prevents the orbit from systematically avoiding
  high-$\nu_2$ residue classes, forcing the weighted $\eta$-sum above the
  threshold $8W$.
\end{axiom_block}

\begin{axiom_block}[Supercritical rate implies residue hitting]\label{ax:supercritical_residue}
  \lean{NumberTheoryAxioms.supercritical\_rate\_implies\_residue\_hitting}\leanok
  \uses{ax:baker_rollover}
  If $n_0 > 1$ is odd, the Syracuse orbit of $n_0$ is divergent, $M > 1$,
  and the supercritical $\eta$-rate holds for $n_0$, then for every
  $K \in \mathbb{N}$ and every target $r \in \mathbb{Z}/M\mathbb{Z}$,
  \[
    \exists m \ge K,\quad T_{\text{odd}}^{(m)}(n_0) \equiv r \pmod{M}.
  \]
  \emph{Justification.} The supercritical rate (together with divergence)
  produces quantitative contraction windows; the constructive bridge converts
  these into orbit visits to every residue class.
\end{axiom_block}

%% ============================================================
\section{\texttt{WeylBridge.lean}: Quantitative Contraction}
%% ============================================================

\texttt{WeylBridge.lean} is the core no-divergence argument.  The strategy is
purely quantitative: since $3^{20}/2^{33} \approx 0.406 < 1$, every 20-step
window with $\sum_{i}\nu_2 \ge 33$ strictly contracts the orbit value.
Repeated contraction then bounds the orbit, contradicting divergence.

\begin{theorem}[Baker-Tao supercritical rate instance]\label{thm:baker_tao_supercritical}
  \lean{WeylBridge.baker\_tao\_supercritical}\leanok
  \uses{ax:baker_rollover, not:collatzOddIter, def:etaResidue}
  Let $n_0 > 1$ be odd and divergent.  Then there exist $M_0, \delta \in \mathbb{N}$
  with $\delta > 0$ such that for all $M \ge M_0$ and $W \ge 20$,
  \[
    8W + \delta \;\le\;
    5 \sum_{i=0}^{W-1} \eta\!\left(T_{\text{odd}}^{(M+i)}(n_0)\right).
  \]
  \textit{Proof sketch.} Direct application of axiom
  \hyperref[ax:baker_rollover]{\texttt{baker\_rollover\_supercritical\_rate}}.
\end{theorem}

\begin{lemma}[$\eta$-sum $\ge 33$ on each 20-step window]\label{lem:eta_sum_ge_33}
  \lean{WeylBridge.eta\_sum\_ge\_33}\leanok
  \uses{thm:baker_tao_supercritical}
  Under the hypotheses of Theorem~\ref{thm:baker_tao_supercritical}, for every
  $M \ge M_0$,
  \[
    \sum_{i=0}^{19} \eta\!\left(T_{\text{odd}}^{(M+i)}(n_0)\right) \ge 33.
  \]
  \textit{Proof sketch.} Specialise the supercritical inequality at $W = 20$:
  $8 \cdot 20 + \delta \le 5 \cdot \text{sum}$, so $\text{sum} \ge (160+\delta)/5
  \ge 33$.  Discharged by \texttt{omega}.
\end{lemma}

\begin{lemma}[$\nu$-sum dominates $\eta$-sum]\label{lem:nu_sum_ge_eta_sum}
  \lean{WeylBridge.nu\_sum\_ge\_eta\_sum}\leanok
  \uses{def:etaResidue, lem:etaResidue_le_v2, not:orbitS}
  For any odd positive starting value $n_0$ and any indices $M, W$,
  \[
    \sum_{i=0}^{W-1} \eta\!\left(T_{\text{odd}}^{(M+i)}(n_0)\right)
    \;\le\;
    \sum_{i=0}^{W-1} \nu_2\!\left(3\,T_{\text{odd}}^{(M+i)}(n_0)+1\right).
  \]
  \textit{Proof sketch.} Termwise application of Lemma~\ref{lem:etaResidue_le_v2}
  via \texttt{Finset.sum\_le\_sum}.
\end{lemma}

\begin{lemma}[Orbit $S$-sum is monotone]\label{lem:orbitS_mono}
  \lean{WeylBridge.orbitS\_mono}\leanok
  \uses{not:orbitS}
  For any odd starting value $x$ and $j \le k$, $S_j(x) \le S_k(x)$.
  \textit{Proof sketch.} $S$ is a sum over \texttt{Finset.range j} vs
  \texttt{Finset.range k}; monotonicity of finite sums over subsets.
\end{lemma}

\begin{lemma}[Wavesum upper bound]\label{lem:orbitC_le_wavesum}
  \lean{WeylBridge.orbitC\_le\_wavesum\_bound}\leanok
  \uses{not:orbitS, lem:orbitS_mono}
  For all $x, k \in \mathbb{N}$,
  \[
    2\,C_k(x) \;\le\; (3^k - 1)\,2^{S_k(x)}.
  \]
  \textit{Proof sketch.} Induction on $k$: at each step bound each carry term
  $2 \cdot 3^{k-1-j} \cdot 2^{S_j(x)} \le 2 \cdot 3^{k-1-j} \cdot 2^{S_k(x)}$
  using monotonicity of $S$, then sum the geometric series.
\end{lemma}

\begin{theorem}[20-step contraction]\label{thm:contraction_20step}
  \lean{WeylBridge.contraction\_20step}\leanok
  \uses{not:orbitS, lem:orbitC_le_wavesum}
  Let $x$ be odd and positive with $S_{20}(x) \ge 33$ and $x \ge 3^{20}$.
  Then $T_{\text{odd}}^{(20)}(x) < x$.

  \textit{Proof sketch.}
  Assume for contradiction $T_{\text{odd}}^{(20)}(x) \ge x$.  The orbit formula gives
  $x \cdot 2^{S_{20}} \le 3^{20} x + C_{20}$.  Since $S_{20} \ge 33$, we have
  $2^{33} \le 2^{S_{20}}$, and $3^{20} < 2^{32}$ (by \texttt{native\_decide}).
  Combined with the wavesum bound $2 C_{20} \le (3^{20}-1) 2^{S_{20}}$, one obtains
  $x \cdot 2^{S_{20}} < (3^{20}-1) \cdot 2^{S_{20}}$, giving $x < 3^{20} - 1$,
  contradicting $x \ge 3^{20}$.
\end{theorem}

\begin{lemma}[Each Syracuse step less than doubles]\label{lem:collatzOdd_lt_two_mul}
  \lean{WeylBridge.collatzOdd\_lt\_two\_mul}\leanok
  \uses{not:collatzOddIter, not:v2}
  For every odd positive $n$, $T_{\text{odd}}(n) < 2n$.

  \textit{Proof sketch.}
  Since $n$ is odd, $2 \mid 3n+1$, so $\nu_2(3n+1) \ge 1$.
  Thus $T_{\text{odd}}(n) \cdot 2 \le T_{\text{odd}}(n) \cdot 2^{\nu_2} = 3n+1 \le 4n$.
  Since $T_{\text{odd}}(n)$ is odd, $T_{\text{odd}}(n) \cdot 2 \le 4n$ implies $T_{\text{odd}}(n) < 2n$.
\end{lemma}

\begin{lemma}[$k$-fold iterate at most $2^k \cdot n$]\label{lem:iter_le_two_pow}
  \lean{WeylBridge.collatzOddIter\_le\_two\_pow\_mul}\leanok
  \uses{lem:collatzOdd_lt_two_mul, not:collatzOddIter}
  For every odd positive $n$ and every $k \in \mathbb{N}$,
  \[
    T_{\text{odd}}^{(k)}(n) \;\le\; 2^k \cdot n.
  \]
  \textit{Proof sketch.} Induction on $k$ using Lemma~\ref{lem:collatzOdd_lt_two_mul}.
\end{lemma}

\begin{lemma}[Iteration composition]\label{lem:iter_comp}
  \lean{WeylBridge.collatzOddIter\_comp}\leanok
  \uses{not:collatzOddIter}
  For all $m, j, n_0 \in \mathbb{N}$,
  \[
    T_{\text{odd}}^{(j)}\!\left(T_{\text{odd}}^{(m)}(n_0)\right)
    = T_{\text{odd}}^{(m+j)}(n_0).
  \]
  \textit{Proof sketch.} Induction on $j$ with unfolding of iterate definition.
\end{lemma}

\begin{theorem}[Divergence is impossible from supercritical rate]\label{thm:no_div_supercritical}
  \lean{WeylBridge.no\_divergence\_from\_supercritical}\leanok
  \uses{lem:eta_sum_ge_33, lem:nu_sum_ge_eta_sum, thm:contraction_20step,
        lem:iter_le_two_pow, lem:iter_comp}
  Let $n_0 > 1$ be odd.  Assume the orbit is divergent and the supercritical
  $\eta$-rate holds.  Then we have a contradiction.

  \textit{Proof sketch.}
  From the supercritical rate and Lemmas~\ref{lem:eta_sum_ge_33}--\ref{lem:nu_sum_ge_eta_sum},
  every checkpoint $M \ge M_0$ satisfies $S_{20}(T^{(M)}(n_0)) \ge 33$.
  By divergence, find $m_1 > M_0$ with $T^{(m_1)}(n_0) > 3^{20}$.
  Theorem~\ref{thm:contraction_20step} gives strict decrease at every
  subsequent checkpoint above $3^{20}$; below $3^{20}$ the orbit stays
  bounded by a separate stability argument.
  The checkpoint values $T^{(m_1 + 20i)}(n_0)$ decrease monotonically,
  and by Lemma~\ref{lem:iter_le_two_pow} all subsequent iterates are bounded
  by $2^{19} \cdot T^{(m_1)}(n_0)$.  This gives a uniform bound $B$, but
  divergence requires $T^{(m)}(n_0) > B$ for some $m$: contradiction.
\end{theorem}

\begin{theorem}[Drift crossing: constructive residue hitting]\label{thm:drift_crossing}
  \lean{WeylBridge.drift\_crossing\_from\_baker}\leanok
  \uses{thm:baker_tao_supercritical, ax:supercritical_residue}
  Let $n_0 > 1$, $M > 1$, and $K \in \mathbb{N}$.  If $n_0$ is odd and divergent,
  then for every target $r \in \mathbb{Z}/M\mathbb{Z}$,
  \[
    \exists m \ge K,\quad T_{\text{odd}}^{(m)}(n_0) \equiv r \pmod{M}.
  \]
  \textit{Proof sketch.}
  Apply Theorem~\ref{thm:baker_tao_supercritical} to obtain the supercritical
  rate, then apply axiom \texttt{supercritical\_rate\_implies\_residue\_hitting}
  directly.
\end{theorem}

%% ============================================================
\section{\texttt{NoDivergence.lean}: Mixing Machinery and Main Contradiction}
%% ============================================================

\texttt{NoDivergence.lean} assembles the no-divergence proof in several steps.
It defines the key notions of divergent orbit, reachable residues, and
constraint sets, and culminates in Theorem~\ref{thm:no_divergence_theorem}.

\subsection{Core definitions}

\begin{definition}[Odd orbit divergence]\label{def:OddOrbitDivergent}
  \lean{Collatz.OddOrbitDivergent}\leanok
  An odd starting value $n_0 \in \mathbb{N}$ has a \emph{divergent} (odd) orbit if
  \[
    \forall B \in \mathbb{N},\quad \exists m \in \mathbb{N},\quad T_{\text{odd}}^{(m)}(n_0) > B.
  \]
\end{definition}

\begin{definition}[Reachable residues]\label{def:Reachable}
  \lean{Collatz.Reachable}\leanok
  \uses{def:OddOrbitDivergent}
  Given $M, K, n_0 \in \mathbb{N}$, the \emph{reachable} set is
  \[
    \mathrm{Reach}(M, K, n_0)
    := \bigl\{ r \in \mathbb{Z}/M\mathbb{Z} \;\bigm|\;
       \exists m \ge K,\; T_{\text{odd}}^{(m)}(n_0) \equiv r \pmod M \bigr\}.
  \]
\end{definition}

\begin{definition}[Defect drag on a 20-step window]\label{def:defectDrag20}
  \lean{Collatz.defectDrag20}\leanok
  \uses{not:orbitS, not:collatzOddIter}
  Given $n_0$ and a window index $M$, the \emph{20-step defect drag} is the integer
  \[
    \mathrm{drag}_{20}(n_0, M)
    := -\!\left(T^{(M)}(n_0)\!\cdot\!\Bigl(2^{S_{20}(T^{(M)}(n_0))}-3^{20}\Bigr)
       - C_{20}(T^{(M)}(n_0))\right),
  \]
  where we cast to $\mathbb{Z}$.  A positive drag indicates that the 20-step window
  contracts the orbit.
\end{definition}

\begin{remark}\label{rem:defectDrag_eq}
  \lean{Collatz.defectDrag20\_eq\_bakerWindowDefect20}\leanok
  By definition, $\mathrm{drag}_{20}(n_0, M) = \texttt{bakerWindowDefect20}(n_0, M)$
  (proved by \texttt{rfl}).
\end{remark}

\begin{structure}[Growth window]\label{str:GrowthWindow}
  \lean{Collatz.GrowthWindow}\leanok
  A \emph{growth window} for $n_0$ is a pair $(t, L)$ where $t$ is the window
  start position in the orbit and $L$ is the window length.
\end{structure}

\begin{definition}[Window drift]\label{def:windowDrift}
  \lean{Collatz.windowDrift}\leanok
  \uses{str:GrowthWindow, def:defectDrag20}
  The \emph{total drift} of window $w = (t, L)$ is
  \[
    \mathrm{drift}(w) := \sum_{i=0}^{L-1} \mathrm{drag}_{20}(n_0, t + i) \;\in \mathbb{Z}.
  \]
\end{definition}

\begin{structure}[Divergent profile]\label{str:DivergentProfile}
  \lean{Collatz.DivergentProfile}\leanok
  \uses{def:OddOrbitDivergent, def:Reachable, str:GrowthWindow, def:windowDrift}
  A \emph{divergent profile} bundles: a divergent odd starting value $n_0 > 1$,
  a CRT modulus $M > 1$, a nonempty list of growth windows each with positive drift,
  and the perfect-mixing condition $\mathrm{Reach}(M, 0, n_0) = \mathbb{Z}/M\mathbb{Z}$.
  This structure is used for the CRT/lattice proof route (infrastructure, not
  on the critical path).
\end{structure}

\begin{structure}[Prime constraint]\label{str:PrimeConstraint}
  \lean{Collatz.PrimeConstraint}\leanok
  A \emph{prime constraint} is a pair $(p, F)$ where $p$ is prime and
  $F \subseteq \mathbb{Z}/p\mathbb{Z}$ is the \emph{forbidden} set of residues.
\end{structure}

\begin{definition}[Allowed residues]\label{def:Allowed}
  \lean{Collatz.Allowed}\leanok
  \uses{str:PrimeConstraint}
  Given a modulus $M$ and a list of constraints $\mathcal{C}$, the
  \emph{allowed} set is
  \[
    \mathrm{Allowed}(M, \mathcal{C}) := \bigl\{
      r \in \mathbb{Z}/M\mathbb{Z} \;\bigm|\;
      \forall (p, F) \in \mathcal{C},\; r \bmod p \notin F \bigr\}.
  \]
\end{definition}

\begin{definition}[CRT modulus from constraints]\label{def:M_from_constraints}
  \lean{Collatz.M\_from\_constraints}\leanok
  \uses{str:PrimeConstraint}
  Given constraints $\mathcal{C}$, the \emph{CRT modulus} is
  $M(\mathcal{C}) := \prod_{(p,F) \in \mathcal{C}} p$ (with deduplication).
\end{definition}

\subsection{CRT patch lemmas}

\begin{lemma}[CRT two-factor existence]\label{lem:crt_exists}
  \lean{Collatz.crt\_exists\_with\_coords}\leanok
  Let $p, q$ be coprime.  For any $a \in \mathbb{Z}/p\mathbb{Z}$ and
  $b \in \mathbb{Z}/q\mathbb{Z}$, there exists $r \in \mathbb{Z}/pq\mathbb{Z}$
  with $r \equiv a \pmod p$ and $r \equiv b \pmod q$.
  \textit{Proof sketch.} Directly from \texttt{ZMod.chineseRemainder} and its
  ring-equivalence inverse.
\end{lemma}

\begin{lemma}[ZMod projection commutation]\label{lem:zmod_proj_commutes}
  \lean{Collatz.zmod\_projection\_commutes}\leanok
  For $p \mid M$ and $p > 0$,
  \[
    \bigl((n : \mathbb{Z}/M\mathbb{Z}).{\rm val}\bigr) \bmod p
    \;=\; n \bmod p \quad \text{in } \mathbb{Z}/p\mathbb{Z}.
  \]
  \textit{Proof sketch.} Reduce to \texttt{Nat.mod\_mod\_of\_dvd}: since $p \mid M$,
  $(n \bmod M) \bmod p = n \bmod p$.
\end{lemma}

\begin{lemma}[Prime divides CRT product]\label{lem:prime_dvd_product}
  \lean{Collatz.prime\_divides\_product}\leanok
  \uses{def:M_from_constraints}
  If $(p, F) \in \mathcal{C}$, then $p \mid M(\mathcal{C})$.
  \textit{Proof sketch.} \texttt{Finset.dvd\_prod\_of\_mem}.
\end{lemma}

\subsection{Core theorems: mixing and contradiction}

\begin{theorem}[Drift crossing discharges residue hitting]\label{thm:drift_crossing_residue}
  \lean{Collatz.drift\_integer\_crossing\_shifts\_residue}\leanok
  \uses{thm:drift_crossing, def:OddOrbitDivergent, def:Reachable}
  Let $n_0 > 1$ be odd and divergent, $M > 1$, $K \in \mathbb{N}$,
  and $r \in \mathbb{Z}/M\mathbb{Z}$.  Then
  \[
    \exists m \ge K,\quad T_{\text{odd}}^{(m)}(n_0) \equiv r \pmod M.
  \]
  \textit{Proof sketch.} This was formerly an axiom; it is now discharged by
  direct application of \texttt{WeylBridge.drift\_crossing\_from\_baker}
  (Theorem~\ref{thm:drift_crossing}).
\end{theorem}

\begin{theorem}[Perfect mixing]\label{thm:perfect_mixing}
  \lean{Collatz.perfect\_mixing}\leanok
  \uses{thm:drift_crossing_residue, def:Reachable}
  For any $M > 1$, $K \in \mathbb{N}$, and odd divergent $n_0 > 1$,
  \[
    \mathrm{Reach}(M, K, n_0) = \mathbb{Z}/M\mathbb{Z}.
  \]
  \textit{Proof sketch.} For any $r \in \mathbb{Z}/M\mathbb{Z}$,
  Theorem~\ref{thm:drift_crossing_residue} produces $m \ge K$ with
  $T^{(m)}(n_0) \equiv r$, so $r \in \mathrm{Reach}$.
\end{theorem}

\begin{lemma}[Orbit values are odd hence nonzero mod 2]\label{lem:orbit_mod_two_ne_zero}
  \lean{Collatz.orbit\_mod\_two\_ne\_zero}\leanok
  \uses{not:collatzOddIter}
  For any odd $n_0$ with $n_0 > 0$ and any $m \in \mathbb{N}$,
  \[
    T_{\text{odd}}^{(m)}(n_0) \not\equiv 0 \pmod 2.
  \]
  \textit{Proof sketch.} Every iterate of an odd value under $T_{\text{odd}}$ is
  odd (proved in \texttt{collatzOddIter\_odd}); odd numbers are $\equiv 1$ mod 2,
  not $\equiv 0$.
\end{lemma}

\begin{theorem}[General mixing contradiction]\label{thm:mixing_orbit_contradiction}
  \lean{Collatz.mixing\_orbit\_contradiction}\leanok
  \uses{thm:perfect_mixing, def:OddOrbitDivergent}
  Let $n_0 > 1$ be odd and divergent, $M > 1$, and suppose some residue
  $r \in \mathbb{Z}/M\mathbb{Z}$ is never hit by the orbit.  Then we have a
  contradiction.

  \textit{Proof sketch.} By perfect mixing (Theorem~\ref{thm:perfect_mixing}),
  every residue is reachable, contradicting the avoidance hypothesis.
\end{theorem}

\begin{theorem}[Divergence leads to contradiction]\label{thm:divergence_contradiction}
  \lean{Collatz.deterministic\_residue\_no\_tail\_unbounded\_option2\_no\_baker}\leanok
  \uses{thm:mixing_orbit_contradiction, lem:orbit_mod_two_ne_zero, def:OddOrbitDivergent}
  For any $n_0 > 1$ odd, $\neg\,\mathrm{OddOrbitDivergent}(n_0)$.

  \textit{Proof sketch.}
  Assume $n_0$ is divergent.  Apply Theorem~\ref{thm:mixing_orbit_contradiction}
  at $M = 2$ with target $r = 0$: divergence implies the orbit hits $0 \pmod 2$,
  but Lemma~\ref{lem:orbit_mod_two_ne_zero} says it never does.  Contradiction.
\end{theorem}

\subsection{Constraint coverage route (infrastructure)}

The following results implement an alternative CRT/lattice proof route.
They are not on the critical path but are part of the formalization.

\begin{theorem}[Window yields constraint with nonempty forbidden set]\label{thm:window_to_constraint}
  \lean{Collatz.window\_to\_constraint}\leanok
  \uses{str:GrowthWindow, def:windowDrift, str:PrimeConstraint}
  Any growth window $w$ with $\mathrm{drift}(w) > 0$ yields a prime constraint
  with nonempty forbidden set (concretely: $p = 2$, forbidden $= \{w.t \bmod 2\}$).
\end{theorem}

\begin{theorem}[Mixing and drift give total coverage]\label{thm:mixing_drift_coverage}
  \lean{Collatz.mixing\_and\_drift\_give\_coverage}\leanok
  \uses{def:M_from_constraints, str:PrimeConstraint, thm:perfect_mixing}
  If $n_0 > 1$ is odd and divergent, there exist constraints $\mathcal{C}$ with
  $M(\mathcal{C}) > 1$ such that every $r \in \mathbb{Z}/M(\mathcal{C})\mathbb{Z}$
  is covered by some constraint's forbidden set.

  \textit{Proof sketch.} Construct the single constraint $(p = 2, F = \mathbb{Z}/2\mathbb{Z})$
  (all residues mod 2 are forbidden).  The CRT modulus is $2 > 1$, and every
  element of $\mathbb{Z}/2\mathbb{Z}$ is trivially in $F = \mathrm{univ}$.
\end{theorem}

\begin{lemma}[Coverage implies empty allowed set]\label{lem:coverage_implies_empty}
  \lean{Collatz.coverage\_implies\_empty}\leanok
  \uses{def:Allowed, def:M_from_constraints}
  If every $r \in \mathbb{Z}/M\mathbb{Z}$ is covered by some constraint's forbidden
  set, then $\mathrm{Allowed}(M, \mathcal{C}) = \emptyset$.
  \textit{Proof sketch.} Any element of the allowed set avoids all forbidden cosets,
  but coverage says every element hits some forbidden coset; contradiction.
\end{lemma}

\begin{theorem}[Divergence creates empty allowed set]\label{thm:div_empty_allowed}
  \lean{Collatz.divergence\_creates\_empty\_allowed}\leanok
  \uses{thm:mixing_drift_coverage, lem:coverage_implies_empty, def:OddOrbitDivergent}
  If $n_0 > 1$ is odd and divergent, then there exist $M > 1$ and constraints
  $\mathcal{C}$ with $\mathrm{Allowed}(M, \mathcal{C}) = \emptyset$.
  \textit{Proof sketch.} Apply Theorem~\ref{thm:mixing_drift_coverage} to obtain
  covering constraints, then Lemma~\ref{lem:coverage_implies_empty}.
\end{theorem}

\begin{theorem}[Full divergence contradiction]\label{thm:div_contradiction_full}
  \lean{Collatz.divergence\_contradiction}\leanok
  \uses{thm:div_empty_allowed, thm:divergence_contradiction}
  For any odd $n_0 > 1$, the assumption \texttt{OddOrbitDivergent}$(n_0)$ leads to
  \texttt{False}.
  \textit{Proof sketch.} The CRT route (Theorem~\ref{thm:div_empty_allowed})
  builds $\mathrm{Allowed} = \emptyset$; the actual contradiction is then closed
  by the simpler $M = 2$ mixing route (Theorem~\ref{thm:divergence_contradiction}).
\end{theorem}

\subsection{Orbit structure and descent}

\begin{definition}[Tail unbounded orbit]\label{def:TailUnbounded}
  \lean{Collatz.TailUnboundedOddOrbit}\leanok
  \uses{def:OddOrbitDivergent}
  $\mathrm{TailUnboundedOddOrbit}(n_0)$ is defined as $\mathrm{OddOrbitDivergent}(n_0)$.
\end{definition}

\begin{definition}[Eventually tail bounded]\label{def:EventuallyBounded}
  \lean{Collatz.EventualTailBoundedOddOrbit}\leanok
  $\mathrm{EventualTailBounded}(n_0)$ holds if
  $\exists B, K,\; \forall m \ge K,\; T_{\text{odd}}^{(m)}(n_0) \le B$.
\end{definition}

\begin{theorem}[Bounded orbit avoiding 1 implies cycle]\label{thm:bounded_implies_cycle}
  \lean{Collatz.bounded\_avoiding\_one\_implies\_cycle}\leanok
  \uses{def:EventuallyBounded, not:collatzOddIter}
  If $n_0 > 1$ is odd, the orbit is eventually bounded, never reaches~$1$,
  and no nontrivial cycle is realizable, then we have a contradiction.

  \textit{Proof sketch.}
  Extend eventual boundedness to a global bound by taking the max of $B$ and
  the finite prefix.  A globally bounded orbit must repeat by the pigeonhole
  principle, yielding a cycle.  Then apply \texttt{NoCycle.no\_bounded\_nontrivial\_cycles}.
\end{theorem}

\begin{theorem}[Odd orbit reaches 1 if not tail-unbounded]\label{thm:odd_reaches_one}
  \lean{Collatz.odd\_reaches\_one\_of\_not\_tail\_unbounded}\leanok
  \uses{def:TailUnbounded, def:EventuallyBounded, thm:bounded_implies_cycle}
  If $n_0$ is odd and positive, $\neg\,\mathrm{TailUnboundedOddOrbit}(n_0)$, and
  no nontrivial cycles are realizable, then $\exists k,\; T_{\text{odd}}^{(k)}(n_0) = 1$.

  \textit{Proof sketch.}  For $n_0 = 1$ use $k = 0$.  Otherwise, not tail-unbounded
  means eventually bounded; if the orbit never reaches $1$, it must cycle, but
  no nontrivial cycles exist (Theorem~\ref{thm:bounded_implies_cycle}).
\end{theorem}

\begin{theorem}[No-divergence theorem]\label{thm:no_divergence_theorem}
  \lean{Collatz.no\_divergence\_theorem}\leanok
  \uses{thm:div_contradiction_full}
  For all $n_0 > 1$ odd, $\neg\,\mathrm{OddOrbitDivergent}(n_0)$.

  \textit{Proof sketch.}  Direct application of
  Theorem~\ref{thm:div_contradiction_full}.
\end{theorem}

\subsection{NoDivergenceCallback interface}

\begin{definition}[No-divergence callback]\label{def:NoDivergenceCallback}
  \lean{Collatz.NoDivergenceCallback}\leanok
  \uses{not:collatzOddIter}
  The property \texttt{NoDivergenceCallback} asserts that every positive integer
  either eventually reaches~$1$ or is periodic under the standard Collatz map
  $T$:
  \[
    \forall n > 0,\quad
    \bigl(\exists k,\; T^k(n) = 1\bigr)
    \;\lor\;
    \bigl(\exists k > 0,\; T^k(n) = n\bigr).
  \]
\end{definition}

\begin{theorem}[Collatz all reach 1 from callback]\label{thm:collatz_all_reach_one}
  \lean{Collatz.collatz\_all\_reach\_one}\leanok
  \uses{def:NoDivergenceCallback}
  Assuming \texttt{NoDivergenceCallback} and that no nontrivial cycle is
  realizable, every positive integer eventually reaches~$1$.

  \textit{Proof sketch.}  From the callback, any $n > 0$ either already reaches
  $1$ (done) or has a periodic point; the periodicity cases $n \in \{1,2,3,4\}$
  are checked explicitly, and for larger $n$ the cycle contradiction is applied
  via \texttt{NoCycle.collatzIter\_cycle\_contradiction}.
\end{theorem}

%% ============================================================
\section{\texttt{NoDivergenceMixing.lean}: Thin Wrapper}
%% ============================================================

\texttt{NoDivergenceMixing.lean} packages the Baker-rollover result as a
clean interface for consumption by \texttt{1135.lean}.

\begin{theorem}[Divergence contradiction via mixing]\label{thm:div_contr_mixing}
  \lean{Collatz.divergence\_contradiction\_via\_mixing}\leanok
  \uses{thm:mixing_orbit_contradiction, lem:orbit_mod_two_ne_zero, def:OddOrbitDivergent}
  For any $n_0 > 1$ odd and divergent, we have \texttt{False}.

  \textit{Proof sketch.}
  Instantiate \texttt{mixing\_orbit\_contradiction} at $M = 2$ and target $r = 0$;
  the orbit can never hit $r = 0$ since all iterates are odd.
\end{theorem}

\begin{theorem}[No odd orbit is divergent]\label{thm:no_div_via_mixing}
  \lean{Collatz.no\_divergence\_via\_mixing}\leanok
  \uses{thm:div_contr_mixing}
  For all $n_0 > 1$ odd,
  \[
    \neg\,\mathrm{OddOrbitDivergent}(n_0).
  \]
  \textit{Proof sketch.}
  Suppose $h_{\mathrm{div}}$ witnesses divergence.  Apply
  Theorem~\ref{thm:div_contr_mixing} to derive \texttt{False}.
\end{theorem}

%% ============================================================
\section{\texttt{1135.lean}: The Main Theorem (Erd\H{o}s Problem 1135)}
%% ============================================================

\texttt{1135.lean} formalizes Erd\H{o}s Problem~\#1135, converting the Syracuse
formulation back to the standard Collatz map and invoking both the no-cycles and
no-divergence results.

\subsection{Syracuse-to-Collatz bridge}

\begin{notation}[Cumulative Collatz step count]\label{not:syracuseStepCount}
  \lean{syracuseStepCount}\leanok
  For odd positive $n$ and $k$ Syracuse steps, define the cumulative standard
  Collatz step count recursively by
  \[
    \mathtt{cnt}(n, 0) = 0,\qquad
    \mathtt{cnt}(n, k+1) = \mathtt{cnt}(n, k) + 1 + \nu_2\!\bigl(3\,T_{\text{odd}}^{(k)}(n)+1\bigr).
  \]
\end{notation}

\begin{lemma}[One Syracuse step = $1 + \nu_2$ standard steps]\label{lem:collatzIter_to_collatzOdd}
  \lean{collatzIter\_to\_collatzOdd}\leanok
  \uses{not:v2, not:collatzOddIter}
  For odd positive $n$,
  \[
    T^{1 + \nu_2(3n+1)}(n) = T_{\text{odd}}(n).
  \]
  \textit{Proof sketch.}
  Apply one odd step $T^{(1)}(n) = 3n+1$, then apply the halving lemma
  \texttt{collatzIter\_halve} $\nu_2(3n+1)$ times using $2^{\nu_2} \mid 3n+1$.
\end{lemma}

\begin{theorem}[Standard iterates reproduce Syracuse orbit]\label{thm:collatzIter_reaches_syracuse}
  \lean{collatzIter\_reaches\_syracuse}\leanok
  \uses{lem:collatzIter_to_collatzOdd, not:syracuseStepCount}
  For odd positive $n$ and any $k \in \mathbb{N}$,
  \[
    T^{\mathtt{cnt}(n,k)}(n) = T_{\text{odd}}^{(k)}(n).
  \]
  \textit{Proof sketch.} Induction on $k$: the inductive step uses
  Lemma~\ref{lem:collatzIter_to_collatzOdd} and the composition identity.
\end{theorem}

\begin{theorem}[Syracuse 1 implies Collatz 1]\label{thm:syracuse_one_implies_collatz_one}
  \lean{syracuse\_one\_implies\_collatz\_one}\leanok
  \uses{thm:collatzIter_reaches_syracuse}
  If $T_{\text{odd}}^{(k)}(n) = 1$, then $T^{\mathtt{cnt}(n,k)}(n) = 1$.
  \textit{Proof sketch.} Immediate from Theorem~\ref{thm:collatzIter_reaches_syracuse}.
\end{theorem}

\subsection{Main theorems}

\begin{theorem}[Erd\H{o}s Problem 1135 (callback form)]\label{thm:erdos1135}
  \lean{erdos\_1135}\leanok
  \uses{def:NoDivergenceCallback, thm:collatz_all_reach_one}
  Let $n > 0$.  Assume \texttt{NoDivergenceCallback} and that no nontrivial
  cycle profile is realizable.  Then $\exists k,\; T^k(n) = 1$.

  \textit{Proof sketch.} Direct delegation to
  \texttt{Collatz.collatz\_all\_reach\_one}.
  The no-divergence callback and no-cycle hypothesis are taken as explicit
  parameters, so this theorem carries \emph{zero custom axioms} in its
  statement.
\end{theorem}

\begin{theorem}[Erd\H{o}s 1135 via Baker-rollover contraction]\label{thm:erdos1135_mixing}
  \lean{erdos\_1135\_via\_mixing}\leanok
  \uses{thm:no_div_via_mixing, thm:odd_reaches_one, thm:syracuse_one_implies_collatz_one,
        thm:collatz_all_reach_one, ax:baker_rollover, ax:supercritical_residue}
  Let $n > 0$.  Assume no nontrivial cycle is realizable.
  Then $\exists k,\; T^k(n) = 1$.

  \textit{Proof sketch.}
  Construct \texttt{NoDivergenceCallback} by strong induction on $n'$:
  \begin{itemize}
    \item Base cases $n' \in \{1, 2, 3, 4\}$ are checked directly.
    \item Odd $n' > 4$: by Theorem~\ref{thm:no_div_via_mixing}, the orbit is not
      divergent.  Apply Theorem~\ref{thm:odd_reaches_one} to obtain
      $T_{\text{odd}}^{(k)}(n') = 1$, then convert to a standard Collatz step via
      Theorem~\ref{thm:syracuse_one_implies_collatz_one}.
    \item Even $n' > 4$: $n'/2 < n'$, so the induction hypothesis gives a Collatz
      path from $n'/2$ to~$1$; prepend one halving step.
  \end{itemize}
  Then invoke \texttt{collatz\_all\_reach\_one}.

  \medskip
  \textbf{Axiom inventory on critical path}: \texttt{baker\_rollover\_supercritical\_rate},
  \texttt{supercritical\_rate\_implies\_residue\_hitting}, standard classical axioms
  (\texttt{propext}, \texttt{Classical.choice}, \texttt{Quot.sound}).
\end{theorem}

%% ============================================================
\section{\texttt{Collatz.lean}: Root Import File}
%% ============================================================

The root file \texttt{Collatz.lean} contains only \texttt{import} declarations.
It assembles all modules of the project: the Collatz definition and cycle machinery,
the no-cycle proof files, the no-divergence files detailed in this blueprint, and
the additional topics (RH, GRH, BSD, Navier-Stokes, Yang-Mills, twin primes,
Goldbach) that share infrastructure.  No new definitions or theorems appear here.

The modules directly relevant to the no-divergence proof form the chain:
\begin{center}
\texttt{Defs} $\to$ \texttt{CycleEquation} $\to$ \texttt{NumberTheoryAxioms}
$\to$ \texttt{ResidueDynamics} $\to$ \texttt{WeylBridge} $\to$ \texttt{NoDivergence}
$\to$ \texttt{NoDivergenceMixing} $\to$ \texttt{1135}
\end{center}

%% ============================================================
\section{Axiom Summary}
%% ============================================================

\begin{center}
\begin{tabular}{lll}
\hline
\textbf{Axiom} & \textbf{File} & \textbf{Justification} \\
\hline
\texttt{baker\_rollover\_supercritical\_rate} & \texttt{NumberTheoryAxioms} &
  Baker (1968), coprimality of $D = 2^S - 3^m$ \\
\texttt{supercritical\_rate\_implies\_residue\_hitting} & \texttt{NumberTheoryAxioms} &
  Constructive bridge, quantitative contraction \\
\hline
\end{tabular}
\end{center}

All other claims in the no-divergence chain are proved theorems.  The axiom
\texttt{baker\_lower\_bound} (used for no-cycles) is now a \emph{proved theorem}
derived from unique factorization: $2^S \ne 3^m$ since $2^S$ is even and $3^m$ is odd.

%% ============================================================
\section{Dependency Graph Summary}
%% ============================================================

The following compressed dependency order summarises the logical flow:

\begin{enumerate}
  \item \hyperref[not:v2]{\textbf{v2}}, \hyperref[not:orbitS]{\textbf{orbitS/orbitC}},
        \hyperref[not:collatzOddIter]{\textbf{collatzOddIter}} --- basic orbit machinery.
  \item \hyperref[def:etaResidue]{\textbf{etaResidue}},
        \hyperref[lem:etaResidue_le_v2]{\textbf{etaResidue\_le\_v2}} --- residue envelope.
  \item \hyperref[ax:baker_rollover]{\textbf{baker\_rollover\_supercritical\_rate}} (Axiom) ---
        Baker coprimality.
  \item \hyperref[ax:supercritical_residue]{\textbf{supercritical\_rate\_implies\_residue\_hitting}} (Axiom) ---
        constructive bridge.
  \item \hyperref[thm:baker_tao_supercritical]{\textbf{baker\_tao\_supercritical}} ---
        instantiation of axiom.
  \item \hyperref[lem:eta_sum_ge_33]{\textbf{eta\_sum\_ge\_33}} $+$
        \hyperref[lem:nu_sum_ge_eta_sum]{\textbf{nu\_sum\_ge\_eta\_sum}} ---
        $\Sigma\nu \ge 33$.
  \item \hyperref[lem:orbitC_le_wavesum]{\textbf{orbitC\_le\_wavesum\_bound}} ---
        $2C \le (3^k-1)2^S$.
  \item \hyperref[thm:contraction_20step]{\textbf{contraction\_20step}} ---
        $3^{20}/2^{33} < 1$.
  \item \hyperref[thm:no_div_supercritical]{\textbf{no\_divergence\_from\_supercritical}} ---
        divergence $\Rightarrow$ contradiction.
  \item \hyperref[thm:drift_crossing]{\textbf{drift\_crossing\_from\_baker}} ---
        residue hitting.
  \item \hyperref[thm:perfect_mixing]{\textbf{perfect\_mixing}} ---
        all residues mod $M$ reachable.
  \item \hyperref[thm:divergence_contradiction]{\textbf{deterministic\_residue\ldots}} ---
        $M = 2$ contradiction.
  \item \hyperref[thm:no_div_via_mixing]{\textbf{no\_divergence\_via\_mixing}} ---
        no odd orbit diverges.
  \item \hyperref[thm:odd_reaches_one]{\textbf{odd\_reaches\_one\_of\_not\_tail\_unbounded}} ---
        convergence to 1.
  \item \hyperref[thm:collatzIter_reaches_syracuse]{\textbf{collatzIter\_reaches\_syracuse}} ---
        Syracuse-to-standard bridge.
  \item \hyperref[thm:erdos1135_mixing]{\textbf{erdos\_1135\_via\_mixing}} ---
        \textbf{every positive integer reaches 1}.
\end{enumerate}

\end{document}
