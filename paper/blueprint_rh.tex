%% blueprint_rh.tex — Leanblueprint for the Riemann Hypothesis Proof Chain
%% Generated from: CriticalLineReal.lean, RotatedZeta.lean, RH.lean,
%%                 XiCodimension.lean, AFEInfrastructure.lean,
%%                 WeylIntegration.lean, SpectralRH.lean, MotohashiRH.lean
%%
%% Proof architecture: four routes all proving RiemannHypothesis.
%% Primary route: Fourier/von Mangoldt (rotation + Mellin bridge).
%% Rotation exposes codimension; Mellin converts zeros to L² modes.

\documentclass[11pt,a4paper]{article}

\usepackage{amsmath,amssymb,amsthm,mathtools}
\usepackage{hyperref}
\usepackage{xcolor}
\usepackage{cleveref}
\usepackage{geometry}
\geometry{margin=1in}

%% Leanblueprint-style theorem environments
\theoremstyle{plain}
\newtheorem{theorem}{Theorem}[section]
\newtheorem{lemma}[theorem]{Lemma}
\newtheorem{proposition}[theorem]{Proposition}
\newtheorem{corollary}[theorem]{Corollary}

\theoremstyle{definition}
\newtheorem{definition}[theorem]{Definition}
\newtheorem{axiom_block}[theorem]{Axiom}

\theoremstyle{remark}
\newtheorem{remark}[theorem]{Remark}

%% Blueprint macros
\newcommand{\lean}[1]{\texttt{#1}}
\newcommand{\leanok}{}
\newcommand{\uses}[1]{\noindent\emph{Uses: #1.}\\[2pt]}

%% Math notation
\newcommand{\RH}{\mathrm{RH}}
\newcommand{\NN}{\mathbb{N}}
\newcommand{\ZZ}{\mathbb{Z}}
\newcommand{\RR}{\mathbb{R}}
\newcommand{\CC}{\mathbb{C}}
\newcommand{\xi}{\xi}
\renewcommand{\Re}{\operatorname{Re}}
\renewcommand{\Im}{\operatorname{Im}}
\newcommand{\xizero}{\widetilde{\xi}_0}
\newcommand{\GammaR}{\Gamma_{\mathbb{R}}}
\newcommand{\QQ}{\mathbb{Q}}

\title{\textbf{Blueprint: Formal Proof of the Riemann Hypothesis}\\
\large Rotation, Mellin, and Spectral Completeness in Lean 4}

\author{Formal Proof Project \\ Lean 4 + Mathlib}

\date{2026}

\begin{document}

\maketitle
\tableofcontents

\bigskip

\begin{abstract}
This blueprint documents the Lean 4 formalization of the Riemann Hypothesis via
four proof routes, with the \textbf{Fourier/von Mangoldt route} as the primary
target for full formalization.

Three mechanisms interact across the proof architecture:
\begin{itemize}
  \item \textbf{Rotation} ($w = -i(s-\tfrac12)$) maps the critical line to the real
    axis, exposing the \emph{codimension gap}: on-line zeros of the real-valued
    $\xi_{\mathrm{rot}}(w)$ are codimension~1 (generic), while off-line zeros are
    codimension~2 (both $\Re$ and $\Im$ must vanish simultaneously). Used in the
    Baker and Fourier routes; motivational for Motohashi.
  \item \textbf{Mellin transform} converts zeros of $\zeta$ into spectral
    components in a Hilbert space $\mathcal{H}$ (Hardy space of the rotated strip):
    on-line zeros ($\Re(\rho)=\tfrac12$) produce components indexed by real frequency
    labels $\gamma$; off-line zeros produce components indexed by complex frequency
    labels $\gamma - i\alpha$ ($\alpha \ne 0$). The contour separation gives
    orthogonality in $\mathcal{H}$. The exponentials $e^{i\gamma u}$ are mode
    labels, not elements of $\mathcal{H}$.
    Used in the Fourier route; not used in Motohashi.
  \item \textbf{Spectral completeness} (the shared logical pattern): a complete Hilbert
    basis admits no nonzero orthogonal element (\lean{abstract\_no\_hidden\_component},
    proved, 0 axioms). An off-line zero produces such an element $\Rightarrow$ contradiction.
\end{itemize}

Four proof routes:
\begin{enumerate}
  \item \textbf{Baker/spiral route} (1 axiom): rotation $+$ wobble decomposition $+$
    Baker's theorem on linear forms in logarithms. Mechanism: codimension obstruction.
  \item \textbf{Fourier/von Mangoldt route} (2 axioms, \textsc{primary}): rotation $+$
    Mellin transform $+$ Beurling--Malliavin completeness. Mechanism: on-line zero
    frequencies define a complete family in $\mathcal{H}$; off-line zero produces
    orthogonal witness $\Rightarrow$ contradiction.
  \item \textbf{Motohashi spectral route} (2 axioms): Maass eigenforms on
    $\mathrm{SL}_2(\ZZ)\backslash\mathbb{H}$ form a complete basis (Selberg 1956,
    self-adjoint Laplacian); off-line zero produces orthogonal witness (Motohashi 1993).
    Same logical pattern as Fourier, different completeness source (self-adjoint spectral
    theory, not Beurling--Malliavin density). Does not use rotation or Mellin formally.
  \item \textbf{Conditional route} (0 axioms): explicit formula completeness passed
    as a theorem argument.
\end{enumerate}

All axioms are proved theorems in the literature. No conjecture is assumed.
\end{abstract}

%% =============================================================================
\section{Schwarz Reflection and Reality on the Critical Line}
\label{sec:critical-line}
%% File: CriticalLineReal.lean
%% =============================================================================

The completed Riemann zeta function $\xi(s) = \GammaR(s)\,\zeta(s)$, where
$\GammaR(s) = \pi^{-s/2}\Gamma(s/2)$, satisfies two fundamental symmetries that
together force $\xi$ to be real-valued on the critical line $\Re(s) = \tfrac12$.

\begin{definition}[Completed Riemann Zeta Function]\label{def:completed-zeta}
\lean{completedRiemannZeta}\leanok
The \emph{completed Riemann zeta function} is
\[
  \xi(s) = \GammaR(s)\,\zeta(s), \qquad \GammaR(s) = \pi^{-s/2}\,\Gamma(s/2),
\]
extended to all of $\CC$ by the functional equation $\xi(s) = \xi(1-s)$ and
meromorphic continuation. Mathlib provides \lean{completedRiemannZeta} and its
entire cousin \lean{completedRiemannZeta\textsubscript{0}}, related by
$\xi(s) = \xi_0(s) - 1/s - 1/(1-s)$.
\end{definition}

\begin{theorem}[Schwarz Reflection for $\xi$]\label{thm:schwarz-reflection}
\lean{CriticalLineReal.schwarz\_reflection\_zeta}\leanok
\uses{def:completed-zeta}
For all $s \in \CC$,
\[
  \xi(\overline{s}) = \overline{\xi(s)}.
\]
\emph{Proof sketch.} For $\Re(s) > 1$, conjugate the Dirichlet series
$\pi^{-s/2}\Gamma(s/2)\sum_n n^{-s}$ term by term, using
$\overline{n^s} = n^{\bar s}$ (real base) and $\overline{\Gamma(s/2)} = \Gamma(\bar s/2)$.
This gives $\xi(\bar s) = \overline{\xi(s)}$ on $\Re(s)>1$. Since both sides are entire
(as functions of $s$) and agree on an open set, the identity theorem extends the identity
to all of $\CC$.
\end{theorem}

\begin{theorem}[$\xi$ is Real-Valued on the Critical Line]\label{thm:xi-real-critical}
\lean{CriticalLineReal.completedZeta\_real\_on\_critical\_line}\leanok
\uses{thm:schwarz-reflection}
For all $t \in \RR$,
\[
  \Im\bigl(\xi(\tfrac12 + it)\bigr) = 0.
\]
\emph{Proof sketch.} On the critical line, $1 - s = \overline{s}$ (since
$1 - (\tfrac12 + it) = \tfrac12 - it = \overline{\tfrac12 + it}$). Therefore
$\xi(s) = \xi(1-s) = \xi(\bar s) = \overline{\xi(s)}$, forcing $\xi(s) \in \RR$.
\end{theorem}

\begin{corollary}[Zeros Reduce to Real Zeros]\label{cor:zero-iff-re-zero}
\lean{CriticalLineReal.critical\_line\_zero\_iff\_re\_zero}\leanok
\uses{thm:xi-real-critical}
For all $t \in \RR$:
\[
  \xi(\tfrac12 + it) = 0 \iff \Re\bigl(\xi(\tfrac12 + it)\bigr) = 0.
\]
\end{corollary}

\begin{theorem}[$\GammaR$ Nonvanishing on the Critical Line]\label{thm:gammaR-ne-zero}
\lean{CriticalLineReal.gammaR\_ne\_zero\_on\_critical\_line}\leanok
For all $t \in \RR$, $\GammaR(\tfrac12 + it) \ne 0$.
\emph{Proof.} Immediate from Mathlib's
\lean{Gammaℝ\_ne\_zero\_of\_re\_pos}, since $\Re(\tfrac12 + it) = \tfrac12 > 0$.
\end{theorem}

\begin{theorem}[Zeros of $\xi$ and $\zeta$ Coincide on the Critical Line]\label{thm:xi-zeta-zeros}
\lean{CriticalLineReal.zeta\_zero\_iff\_xi\_zero}\leanok
\uses{thm:gammaR-ne-zero}
For all $t \in \RR$: $\zeta(\tfrac12 + it) = 0 \iff \xi(\tfrac12 + it) = 0$.
\emph{Proof.} Since $\xi(s) = \GammaR(s)\,\zeta(s)$ and $\GammaR(\tfrac12+it) \ne 0$,
we have $\xi = 0 \iff \zeta = 0$.
\end{theorem}

\begin{theorem}[$\xi$ is Even on the Critical Line]\label{thm:xi-even}
\lean{CriticalLineReal.xi\_even\_on\_critical\_line}\leanok
\uses{def:completed-zeta}
For all $t \in \RR$: $\xi(\tfrac12 + it) = \xi(\tfrac12 - it)$.
\emph{Proof.} From the functional equation $\xi(1-s) = \xi(s)$ with $s = \tfrac12 + it$:
$1 - s = \tfrac12 - it$.
\end{theorem}

\begin{theorem}[Special Values: $\xi(1) < 0$ and $\xi(0) < 0$]\label{thm:xi-special}
\lean{CriticalLineReal.xi\_at\_one\_negative}\leanok
$\Re(\xi(1)) < 0$. Consequently $\Re(\xi(0)) < 0$ by the functional equation.
\emph{Proof.} Mathlib gives $\xi(1) = (\gamma - \log(4\pi))/2$. Since
$\gamma < 2/3$ (Euler--Mascheroni constant) and $\log(4\pi) > \log(e) = 1$, we get
$\xi(1) < (2/3 - 1)/2 < 0$. Then $\xi(0) = \xi(1-1) = \xi(1)$ by the functional equation.
\end{theorem}

\subsection{Hardy's Theorem from Sign Changes}

\begin{definition}[Hardy Oscillation Hypothesis]\label{def:xi-oscillates}
\lean{CriticalLineReal.XiOscillates}\leanok
The predicate $\lean{XiOscillates}$ asserts: for every $T \in \RR$, there exist
$t_1 < t_2$ with $T < t_1$ such that $\Re(\xi(\tfrac12+it_1)) < 0$ and
$\Re(\xi(\tfrac12+it_2)) > 0$.

\emph{Mathematical content}: the second-moment estimate
$\int_0^T |\zeta(\tfrac12+it)|^2\,dt \sim T\log T$ forces
the real-valued function $t \mapsto \Re(\xi(\tfrac12+it))$ to oscillate, producing
infinitely many sign changes.
\end{definition}

\begin{theorem}[Continuity of $\xi$ along the Critical Line]\label{thm:xi-continuous}
\lean{CriticalLineReal.continuous\_xi\_on\_critical\_line}\leanok
\uses{def:completed-zeta}
The map $t \mapsto \xi(\tfrac12 + it)$ is continuous on $\RR$.
\emph{Proof.} The embedding $t \mapsto \tfrac12 + it$ is continuous, and
$\xi$ is differentiable (hence continuous) at every point of the critical line
(where $s \ne 0$ and $s \ne 1$).
\end{theorem}

\begin{theorem}[Sign Change Gives Zero (IVT)]\label{thm:sign-change-zero}
\lean{CriticalLineReal.sign\_change\_gives\_zero}\leanok
\uses{thm:xi-real-critical, thm:xi-continuous, cor:zero-iff-re-zero}
If $t_1 < t_2$ and $\Re(\xi(\tfrac12+it_1)) < 0 < \Re(\xi(\tfrac12+it_2))$, then
there exists $t_0 \in (t_1, t_2)$ with $\xi(\tfrac12 + it_0) = 0$.
\emph{Proof.} By \Cref{thm:xi-real-critical}, $\Re(\xi(\tfrac12+\cdot))$ is a real-valued
continuous function. The Intermediate Value Theorem gives the zero; strict inequalities
at the endpoints rule out the boundary.
\end{theorem}

\begin{theorem}[Hardy's Theorem]\label{thm:hardy}
\lean{CriticalLineReal.hardy\_infinitely\_many\_zeros}\leanok
\uses{def:xi-oscillates, thm:sign-change-zero, thm:xi-zeta-zeros}
Assuming $\lean{XiOscillates}$, for every $T \in \RR$ there exists $t > T$ with
$\zeta(\tfrac12 + it) = 0$. In particular, $\zeta$ has infinitely many zeros on
$\Re(s) = \tfrac12$.
\emph{Proof.} \lean{XiOscillates} provides $t_1, t_2 > T$ with a sign change.
\Cref{thm:sign-change-zero} gives a zero $t_0 > T$ of $\xi$, which by
\Cref{thm:xi-zeta-zeros} is also a zero of $\zeta$.
\end{theorem}

%% =============================================================================
\section{The Rotated Zeta Function and Codimensionality}
\label{sec:rotated}
%% File: RotatedZeta.lean
%% =============================================================================

The key conceptual move is to change coordinates so that the critical line
becomes the real axis. This makes the symmetry structure of $\xi$ manifest and
reveals the Riemann Hypothesis as a codimensionality statement.

\begin{definition}[Coordinate Rotation]\label{def:rotation}
\lean{RotatedZeta.rotatedXi}\leanok
\uses{def:completed-zeta}
The \emph{rotated completed zeta function} is
\[
  \xi_{\mathrm{rot}}(w) = \xi\!\left(\tfrac12 + iw\right), \qquad w \in \CC.
\]
The coordinate change $w = -i(s - \tfrac12)$ maps:
\begin{itemize}
  \item Critical line $\Re(s) = \tfrac12$ to the real axis $\Im(w) = 0$.
  \item Critical strip $0 < \Re(s) < 1$ to the horizontal strip $|\Im(w)| < \tfrac12$.
  \item A point $s = \sigma + it$ to $w = t - i(\sigma - \tfrac12)$.
\end{itemize}
\end{definition}

\begin{definition}[Rotated Riemann Hypothesis]\label{def:rotated-rh}
\lean{RotatedZeta.RotatedRH}\leanok
\uses{def:rotation}
$\lean{RotatedRH}$ is the statement:
\[
  \forall\, w \in \CC,\quad \xi_{\mathrm{rot}}(w) = 0 \implies \Im(w) = 0.
\]
That is, all zeros of $\xi_{\mathrm{rot}}$ are real.
\end{definition}

\begin{theorem}[$\xi_{\mathrm{rot}}$ is Real on the Real Axis]\label{thm:xirot-real-reals}
\lean{RotatedZeta.rotatedXi\_real\_on\_reals}\leanok
\uses{def:rotation, thm:xi-real-critical}
For all $t \in \RR$: $\Im(\xi_{\mathrm{rot}}(t)) = 0$.
\emph{Proof.} $\xi_{\mathrm{rot}}(t) = \xi(\tfrac12 + it)$, and
\Cref{thm:xi-real-critical} gives $\Im(\xi(\tfrac12+it)) = 0$.
\end{theorem}

\begin{theorem}[$\xi_{\mathrm{rot}}$ is Real on the Imaginary Axis]\label{thm:xirot-real-imag}
\lean{RotatedZeta.rotatedXi\_real\_on\_imaginary\_axis}\leanok
\uses{def:rotation, thm:schwarz-reflection}
For all $b \in \RR$: $\Im(\xi_{\mathrm{rot}}(ib)) = 0$.
\emph{Proof.} $\xi_{\mathrm{rot}}(ib) = \xi(\tfrac12 + i\cdot ib) = \xi(\tfrac12 - b)$.
Since $\tfrac12 - b$ is real, Schwarz reflection gives $\xi(\overline{\tfrac12-b})
= \overline{\xi(\tfrac12-b)}$, i.e., $\xi(\tfrac12-b) = \overline{\xi(\tfrac12-b)}$,
so $\xi(\tfrac12-b) \in \RR$.
\end{theorem}

\begin{theorem}[No Zeros on the Imaginary Axis in the Strip]\label{thm:xirot-no-zeros-imag}
\lean{RotatedZeta.rotatedXi\_no\_zeros\_imaginary\_axis}\leanok
\uses{thm:xirot-real-imag}
For $b \in \RR$ with $|b| < \tfrac12$ and $b \ne 0$: $\xi_{\mathrm{rot}}(ib) \ne 0$.
\emph{Proof.} $\xi_{\mathrm{rot}}(ib) = \xi(\tfrac12 - b) = \GammaR(\tfrac12-b)\,\zeta(\tfrac12-b)$.
Since $\tfrac12 - b$ is real and in $(0,1)$, and $\zeta$ has no real zeros in $(0,1)$
(proved via nonvanishing on $\Re(s) \ge 1$ and the functional equation for the range
$(0,\tfrac12)$), we have $\xi_{\mathrm{rot}}(ib) \ne 0$.
\end{theorem}

\begin{theorem}[$D_4$ Symmetry of $\xi_{\mathrm{rot}}$]\label{thm:xirot-symmetry}
\lean{RotatedZeta.rotatedXi\_neg, RotatedZeta.rotatedXi\_conj}\leanok
\uses{def:rotation, thm:schwarz-reflection}
$\xi_{\mathrm{rot}}$ is even: $\xi_{\mathrm{rot}}(-w) = \xi_{\mathrm{rot}}(w)$.
It is conjugation-equivariant: $\xi_{\mathrm{rot}}(\bar w) = \overline{\xi_{\mathrm{rot}}(w)}$.
Together these give $D_4$ symmetry.
\emph{Proof.} Evenness follows from the functional equation:
$\xi(\tfrac12 - iw) = \xi(1 - (\tfrac12 + iw)) = \xi(\tfrac12 + iw)$.
Conjugation-equivariance follows from Schwarz reflection plus evenness.
\end{theorem}

\begin{theorem}[Equivalence of Standard and Rotated RH]\label{thm:rh-iff-rotated}
\lean{RotatedZeta.rh\_iff\_rotated}\leanok
\uses{def:rotated-rh}
$\lean{RiemannHypothesis} \iff \lean{RotatedRH}$.
\emph{Proof.} The coordinate change $w = -i(s - \tfrac12)$ is bijective, so
$\xi_{\mathrm{rot}}(w) = \xi(s)$. A zero $s = \sigma + it$ with $\sigma \ne \tfrac12$
corresponds to $w = t - i(\sigma - \tfrac12)$ with $\Im(w) \ne 0$, and vice versa.
\end{theorem}

\begin{axiom_block}[Codimensionality Axiom (Baker Route)]\label{ax:rotation-forbids}
\lean{RotatedZeta.rotation\_forbids\_off\_axis}\leanok
For $w \in \CC$ with $\Im(w) \ne 0$ and $|\Im(w)| < \tfrac12$:
\[
  \xi_{\mathrm{rot}}(w) \ne 0.
\]
\emph{Mathematical content}: the Euler product, built from the prime numbers whose
logarithms are $\QQ$-linearly independent (Baker's theorem), cannot vanish off the real
axis in the rotated strip. The de la Vallée Poussin zero-free region
$\sigma > 1 - c/\log|t|$ (proved in \lean{Mertens341}) represents the limit of
finite-Baker information; this axiom asserts the gap from that region to $\sigma = \tfrac12$
is also zero-free.

\emph{Necessity}: The Beurling counterexample shows that for generalized prime systems
with $\QQ$-commensurable logarithms, off-axis zeros DO exist. This axiom is therefore
sharp.
\end{axiom_block}

\begin{theorem}[Zero-Free Region (de la Vallée Poussin)]\label{thm:zero-free-partial}
\lean{CountingArgument.zero\_free\_region\_is\_partial}\leanok
There exists $c > 0$ such that for all $\sigma, t \in \RR$ with
$\sigma > 1 - c/\log(|t|+2)$: $\zeta(\sigma + it) \ne 0$.
\emph{Proof.} Delegates to \lean{Mertens341.zero\_free\_region}.
\end{theorem}

\begin{theorem}[The Gap Exists]\label{thm:gap-exists}
\lean{CountingArgument.gap\_exists}\leanok
For any $c > 0$ there exist $\sigma, t$ with $\tfrac12 < \sigma < 1$ and
$\sigma \le 1 - c/\log(|t|+2)$, i.e., the de la Vallée Poussin region does not
cover the full strip $\Re(s) > \tfrac12$.
\emph{Proof.} Take $t = e^{8c}$; then $c/\log(e^{8c}+2) \le 1/8 < 1/4$,
so $\sigma = 3/4$ lies in the gap.
\end{theorem}

\subsection{Abstract Spectral Gap Principle}

\begin{theorem}[Abstract Rotation Spectral Gap]\label{thm:rotation-spectral-gap}
\lean{RotatedZeta.rotation\_spectral\_gap}\leanok
Let $V$ be a finite-dimensional real inner product space and $f : V \to \RR$ a
continuous 2-homogeneous function ($f(cv) = c^2 f(v)$) that is strictly positive
on nonzero vectors. Then there exists $\delta > 0$ with $\delta\|x\|^2 \le f(x)$
for all $x \in V$.
\emph{Proof.} Compactness of the unit sphere gives a minimum $\delta = \min_{\|x\|=1} f(x) > 0$.
Scaling by $\|x\|^2$ extends to all $x$.
\end{theorem}

\begin{remark}
\Cref{thm:rotation-spectral-gap} is the abstract backbone common to RH, Yang--Mills, and
Navier--Stokes: in each case a symmetry constraint (incompressibility / non-abelian bracket /
functional equation) forces a quadratic form to be positive on a sphere, and compactness
gives a spectral gap.
\end{remark}

\subsection{Fourier Completeness Infrastructure}

\begin{theorem}[Hilbert Basis Completeness]\label{thm:hilbert-complete}
\lean{RotatedZeta.hilbert\_basis\_complete}\leanok
Let $H$ be a Hilbert space over $\CC$ with complete orthonormal basis $\{b_i\}$.
If $f \in H$ satisfies $\langle b_i, f\rangle = 0$ for all $i$, then $f = 0$.
\emph{Proof.} The Hilbert basis expansion $f = \sum_i \langle b_i, f\rangle b_i$
collapses to $f = 0$ when all coefficients vanish; proved by uniqueness of sum limits.
\end{theorem}

\begin{theorem}[No Hidden Spectral Component]\label{thm:no-hidden-component}
\lean{RotatedZeta.abstract\_no\_hidden\_component}\leanok
\uses{thm:hilbert-complete}
In any Hilbert space $H$ with complete ONB $\{b_i\}$: if $\langle b_i, f\rangle = 0$
for all $i$, then $f = 0$. Equivalently, no nonzero element can be orthogonal to a
complete basis.
\emph{Proof.} Immediate from \Cref{thm:hilbert-complete} via the Hilbert basis
representation formula.
\end{theorem}

\begin{theorem}[Fourier Basis Completeness]\label{thm:fourier-complete}
\lean{RotatedZeta.fourier\_is\_complete}\leanok
\uses{thm:no-hidden-component}
For the Fourier basis $\{e^{2\pi i n t / T}\}$ on $L^2(\RR/T\ZZ, \CC)$:
any $f$ with all Fourier coefficients zero satisfies $f = 0$.
\emph{Proof.} \lean{fourierBasis} is a \lean{HilbertBasis} in Mathlib; apply
\Cref{thm:hilbert-complete}.
\end{theorem}

\begin{theorem}[Parseval Identity]\label{thm:parseval}
\lean{RotatedZeta.parseval\_total\_energy}\leanok
\uses{thm:fourier-complete}
$\sum_{n \in \ZZ} |\hat f(n)|^2 = \int |f(t)|^2\,dt$.
\emph{Proof.} \lean{tsum\_sq\_fourierCoeff} from Mathlib.
\end{theorem}

\begin{theorem}[Rotation is an Isometry]\label{thm:rotation-isometry}
\lean{ExplicitFormulaBridge.rotation\_is\_isometry}\leanok
The coordinate change $w = -i(s - \tfrac12)$ satisfies
$\tfrac12 + i(-i(s-\tfrac12)) = s$, and $\|-i(s_1-\tfrac12) - (-i(s_2-\tfrac12))\|
= \|s_1 - s_2\|$.
\emph{Proof.} Direct calculation: $I \cdot (-I) = 1$, so the composed maps are identities;
$\|-I \cdot z\| = \|z\|$ since $\|{-I}\| = 1$.
\end{theorem}

%% =============================================================================
\section{Wobble Theory and Off-Axis Nonvanishing}
\label{sec:wobble}
%% File: XiCodimension.lean
%% =============================================================================

The wobble of $\xi$ measures its departure from real-valuedness. The pole decomposition
$\xi(s) = \xi_0(s) - 1/(s(1-s))$ separates the entire part from the pole field; the
functional equation and Schwarz reflection constrain the wobble structurally.

\begin{definition}[Wobble Function]\label{def:wobble}
\lean{Collatz.XiCodimension.wobble}\leanok
\uses{def:completed-zeta}
The \emph{wobble} of $\xi$ is $\omega(s) = \Im(\xi(s))$. By \Cref{thm:xi-real-critical},
$\omega(\tfrac12 + it) = 0$ for all $t \in \RR$.
\end{definition}

\begin{theorem}[Wobble is Antisymmetric]\label{thm:wobble-antisymmetric}
\lean{Collatz.XiCodimension.wobble\_antisymmetric}\leanok
\uses{def:wobble, thm:schwarz-reflection}
$\omega(1 - \bar s) = -\omega(s)$, i.e., the imaginary part of $\xi$ is antisymmetric
across the critical line: $\Im(\xi(\sigma+it)) = -\Im(\xi((1-\sigma)+it))$.
\emph{Proof.} Schwarz reflection gives $\Im(\xi(\bar s)) = -\Im(\xi(s))$; the functional
equation gives $\xi(1-s) = \xi(s)$; composing yields the antisymmetry.
\end{theorem}

\begin{theorem}[Wobble Decomposition]\label{thm:wobble-decomp}
\lean{Collatz.XiCodimension.wobble\_decomposition}\leanok
\uses{def:wobble}
For $s \ne 0, 1$:
\[
  \Im(\xi(s)) = \Im(\xi_0(s)) + \Im\!\left(\frac{-1}{s(1-s)}\right).
\]
The pole contribution is $\Im(-1/(s(1-s))) = t(1-2\sigma)/|s(1-s)|^2$ where $s=\sigma+it$.
This is positive for $\sigma < \tfrac12$, zero at $\sigma = \tfrac12$, and negative for
$\sigma > \tfrac12$.
\end{theorem}

\begin{theorem}[Pole Contribution in the Left Strip]\label{thm:pole-left}
\lean{Collatz.XiCodimension.pole\_contribution\_positive\_left\_strip}\leanok
\uses{thm:wobble-decomp}
For $0 < \sigma < \tfrac12$ and $t > 0$: $\Im(-1/(s(1-s))) > 0$.
\emph{Proof.} $\Im(-1/(s(1-s))) = \Im(s(1-s))/|s(1-s)|^2$, and
$\Im(s(1-s)) = t(1-2\sigma) > 0$ when $\sigma < \tfrac12$.
\end{theorem}

\begin{theorem}[Pole Contribution in the Right Strip]\label{thm:pole-right}
\lean{Collatz.XiCodimension.pole\_contribution\_negative\_right\_strip}\leanok
\uses{thm:wobble-decomp}
For $\tfrac12 < \sigma < 1$ and $t > 0$: $\Im(-1/(s(1-s))) < 0$.
\end{theorem}

\begin{theorem}[$\xi_0$ is Real on the Critical Line and Real Axis]\label{thm:xi0-real}
\lean{Collatz.XiCodimension.xi0\_real\_on\_critical\_line,
      Collatz.XiCodimension.xi0\_real\_on\_real\_axis}\leanok
\uses{thm:wobble-decomp}
$\Im(\xi_0(\tfrac12+it)) = 0$ for all $t \in \RR$.
$\Im(\xi_0(\sigma)) = 0$ for all $\sigma \in \RR$.
\emph{Proof.} From Schwarz reflection and the functional equation for $\xi_0$:
$\xi_0(\bar s) = \overline{\xi_0(s)}$ (Schwarz) and $\xi_0(1-s) = \xi_0(s)$ (functional eq).
On the critical line $\bar s = 1-s$, so $\xi_0(s) = \overline{\xi_0(s)}$.
On the real axis $\bar s = s$, so $\xi_0(s) = \overline{\xi_0(s)}$ directly.
\end{theorem}

\begin{theorem}[Zeros of $\xi$ in the Strip are Isolated]\label{thm:xi-zeros-isolated}
\lean{Collatz.XiCodimension.xi\_zeros\_isolated\_in\_strip}\leanok
\uses{def:completed-zeta}
For $s_0$ with $\tfrac12 < \Re(s_0) < 1$, if $\xi(s_0) = 0$ then $s_0$ is an isolated
zero of $\xi$.
\emph{Proof.} $\xi$ is analytic on $\{0 < \Re(s) < 1\}$ and not identically zero
(since $\xi(1) \ne 0$ by \Cref{thm:xi-special}). The analytic identity theorem gives isolation.
\end{theorem}

\begin{theorem}[$\xi'$ is Purely Imaginary on the Critical Line]\label{thm:xi-deriv-imag}
\lean{Collatz.XiCodimension.xi\_deriv\_purely\_imaginary\_on\_critical\_line}\leanok
\uses{thm:xi-real-critical}
For all $t \in \RR$: $\Re(\xi'(\tfrac12 + it)) = 0$.
\emph{Proof.} The path $\tau \mapsto \xi(\tfrac12+i\tau)$ is real-valued, so its complex
derivative $\xi'(\tfrac12+it)\cdot i$ is real. If $z\cdot i \in \RR$ then $\Re(z) = 0$.
\end{theorem}

\subsection{Number-Theoretic Uncertainty Principle}

\begin{theorem}[Log-Independence of Primes]\label{thm:log-independence}
\lean{Collatz.XiCodimension.no\_log\_relation\_primes}\leanok
For distinct primes $p, q$ and integers $a, b$ with $a \ne 0$:
$a\log p \ne b\log q$.
\emph{Proof.} Exponentiating gives $p^{|a|} = q^{|b|}$; then $p \mid q^{|b|}$ implies
$p \mid q$ (by primality), so $p = q$, contradicting distinctness. Zero custom axioms.
\end{theorem}

\begin{theorem}[Helix Uncertainty Principle]\label{thm:helix-uncertainty}
\lean{Collatz.XiCodimension.helix\_uncertainty\_2\_3}\leanok
\uses{thm:log-independence}
For $t \ne 0$: if $\cos(t\log 2) = -1$ then $\sin(t\log 3) \ne 0$.
\emph{Proof.} If both fail, cross-multiplying the resulting period relations for $t\log 2$
and $t\log 3$ yields an integer relation $a\log 2 = b\log 3$, contradicting
\Cref{thm:log-independence}. Zero custom axioms.
\end{theorem}

\begin{theorem}[At Most One Prime Has $\sin(t\log p) = 0$]\label{thm:at-most-one-sin}
\lean{Collatz.XiCodimension.at\_most\_one\_sin\_zero}\leanok
\uses{thm:log-independence}
For distinct primes $p, q$ and $t \ne 0$: $\sin(t\log p) = 0$ and $\sin(t\log q) = 0$
cannot both hold.
\emph{Proof.} Both vanish iff $t\log p, t\log q \in \pi\ZZ$, giving an integer relation
between $\log p$ and $\log q$, which contradicts \Cref{thm:log-independence}.
\end{theorem}

\subsection{Baker's Axiom and Strip Nonvanishing}

\begin{theorem}[Zero Forces Exact Hit]\label{thm:exact-hit}
\lean{Collatz.XiCodimension.zeta\_zero\_forces\_exact\_hit}\leanok
\uses{def:completed-zeta}
If $\tfrac12 < \Re(s) < 1$ and $\zeta(s) = 0$, then $\xi_0(s) = 1/(s(1-s))$.
\emph{Proof.} $\zeta(s)=0 \Rightarrow \xi(s) = 0 \Rightarrow \xi_0(s) = 1/s + 1/(1-s)
= 1/(s(1-s))$ from the pole decomposition.
\end{theorem}

\begin{theorem}[Positive Imaginary Part at Hypothetical Zero]\label{thm:exact-hit-im}
\lean{Collatz.XiCodimension.exact\_hit\_im\_pos}\leanok
\uses{thm:exact-hit, thm:pole-right}
If $\tfrac12 < \Re(s) < 1$, $\Im(s) > 0$, and $\zeta(s) = 0$, then
$\Im(\xi_0(s)) > 0$.
\emph{Proof.} By \Cref{thm:exact-hit}, $\Im(\xi_0(s)) = \Im(1/(s(1-s)))$.
Since $\Im(-1/(s(1-s))) < 0$ by \Cref{thm:pole-right}, we have $\Im(1/(s(1-s))) > 0$.
\end{theorem}

\begin{axiom_block}[Baker's Theorem Applied to the Euler Product]\label{ax:baker}
\lean{Collatz.XiCodimension.baker\_forbids\_pole\_hit}\leanok
\uses{thm:exact-hit}
For $\tfrac12 < \Re(s) < 1$ and $\Im(s) \ne 0$:
\[
  \xi_0(s) \ne \frac{1}{s(1-s)}.
\]
\emph{Reference}: Baker (1966), Linear forms in the logarithms of algebraic numbers.
\emph{Mathematical content}: The Euler product $\zeta(s) = \prod_p(1-p^{-s})^{-1}$
decomposes into prime phases $e^{it\log p}$ weighted by $p^{-\sigma}$. Baker's theorem
gives $|\sum a_i \log p_i| > C \cdot H^{-\kappa}$ for integer coefficients $a_i$ with
$H = \max|a_i|$, preventing the phases from achieving the exact cancellation required
for a zero.
\emph{Proved structural support (zero custom axioms)}: \Cref{thm:log-independence},
\Cref{thm:helix-uncertainty}, \Cref{thm:at-most-one-sin}, \Cref{thm:exact-hit-im}.
\end{axiom_block}

\begin{theorem}[Strip Nonvanishing (Baker Route)]\label{thm:strip-nonvanishing-baker}
\lean{Collatz.XiCodimension.spiral\_euler\_non\_cancellation}\leanok
\uses{ax:baker, thm:exact-hit}
For $\tfrac12 < \Re(s) < 1$ and $\Im(s) \ne 0$: $\zeta(s) \ne 0$.
\emph{Proof.} Suppose $\zeta(s_0) = 0$. By \Cref{thm:exact-hit},
$\xi_0(s_0) = 1/(s_0(1-s_0))$. This contradicts \Cref{ax:baker}.
\end{theorem}

\begin{theorem}[Off-Axis Zeta Nonvanishing]\label{thm:off-axis-nonvanishing}
\lean{Collatz.XiCodimension.off\_axis\_zeta\_ne\_zero}\leanok
\uses{thm:strip-nonvanishing-baker}
For $\tfrac12 < \Re(s) < 1$ and $\Im(s) \ne 0$: $\zeta(s) \ne 0$.
This is the main conclusion of \lean{XiCodimension.lean}.
\end{theorem}

%% =============================================================================
\section{The Functional Equation Scalar and $\chi$-Attenuation}
\label{sec:afe}
%% File: AFEInfrastructure.lean
%% =============================================================================

The functional equation $\zeta(s) = \chi(s)\,\zeta(1-s)$ is a key ingredient in
the AFE (Asymptotic Functional Equation) route to strip nonvanishing.

\begin{definition}[Functional Equation Scalar]\label{def:chi}
\lean{AFEInfrastructure.chi}\leanok
\uses{def:completed-zeta}
\[
  \chi(s) = \frac{\GammaR(1-s)}{\GammaR(s)}.
\]
\end{definition}

\begin{theorem}[Functional Equation via $\chi$]\label{thm:func-eq-chi}
\lean{AFEInfrastructure.functional\_equation\_chi}\leanok
\uses{def:chi}
For $s \ne 0, 1$ with $\GammaR(s), \GammaR(1-s) \ne 0$:
$\zeta(s) = \chi(s)\,\zeta(1-s)$.
\emph{Proof.} From $\xi(s) = \xi(1-s)$ and $\xi = \GammaR\cdot\zeta$.
\end{theorem}

\begin{theorem}[$\chi$ Identity via Gamma]\label{thm:chi-gamma-identity}
\lean{AFEInfrastructure.chi\_eq\_inv\_gammaC\_cos}\leanok
\uses{def:chi}
For $\Re(s) > 0$:
$\chi(s) = \bigl(\Gamma_{\CC}(s)\cos(\pi s/2)\bigr)^{-1}$.
\emph{Proof.} From Mathlib's \lean{Gammaℝ\_div\_Gammaℝ\_one\_sub}.
\end{theorem}

\begin{theorem}[$\chi$ Norm Attenuation for Large $|t|$]\label{thm:chi-attenuation}
\lean{AFEInfrastructure.chi\_attenuation\_large\_t}\leanok
\uses{thm:chi-gamma-identity}
For $\tfrac12 < \sigma < 1$, there exist $C, T_0 > 0$ such that for $|t| \ge T_0$:
\[
  \|\chi(\sigma + it)\| \le C(|t|+2)^{1/2-\sigma}.
\]
\emph{Proof.} Stirling: $\|\Gamma(\sigma+it)\| \sim C_1\,|t|^{\sigma-1/2}e^{-\pi|t|/2}$.
Cosine lower bound: $\|\cos(\pi s/2)\| \ge e^{\pi|t|/2}/4$ for $|t| \ge 1$.
Product: $\|\Gamma_\CC(s)\cos(\pi s/2)\| \ge C_2\,|t|^{\sigma-1/2}$
(exponentials cancel). So $\|\chi(s)\| \le C_3\,|t|^{1/2-\sigma}$.
\end{theorem}

\begin{theorem}[Cosine Exponential Lower Bound]\label{thm:cos-exp-bound}
\lean{AFEInfrastructure.cos\_exp\_lower\_bound}\leanok
For $|t| \ge 1$: $\|\cos(\pi s/2)\| \ge e^{\pi|t|/2}/4$ where $s = \sigma + it$.
\emph{Proof.} Using $\cos(z) = (e^{iz}+e^{-iz})/2$ and the reverse triangle inequality:
$\|\cos z\| \ge |\sinh(\Im z)| \ge e^{|\Im z|}/4$ for $|\Im z| \ge 1$.
\end{theorem}

%% =============================================================================
\section{Weyl Integration and Zero-Input Closure}
\label{sec:weyl}
%% File: WeylIntegration.lean
%% =============================================================================

The Weyl equidistribution of prime phases drives growth of Dirichlet partial sums,
providing the ``tail assurance'' needed to close the zero-input theory.

\begin{definition}[Zero-Input Theory]\label{def:zero-input}
\lean{ZeroInputTheory}\leanok
The \emph{zero-input theory} (\lean{DirichletCompensatedNormLockingHypothesis}) asserts:
for every $s$ in the critical strip, there exist $N_0, \delta > 0$ such that for all
$N \ge N_0$:
\[
  \left\|S(s,N) - \frac{N^{1-s}}{1-s}\right\| \ge \delta,
\]
where $S(s,N) = \sum_{n=1}^N n^{-s}$.
\emph{Motivation}: the compensated sum converges to $\zeta(s)$; if $\zeta(s) \ne 0$
it is eventually bounded below.
\end{definition}

\begin{theorem}[Strip Nonvanishing (Combined Route)]\label{thm:strip-nonvanishing}
\lean{Collatz.WeylIntegration.strip\_nonvanishing\_zero\_input}\leanok
\uses{thm:off-axis-nonvanishing}
\lean{LogEulerSpiralNonvanishingHypothesis} holds: for $\tfrac12 < \Re(s) < 1$,
$\zeta(s) \ne 0$.
\emph{Proof.}
\begin{itemize}
  \item If $\Im(s) = 0$: proved by \lean{EntangledPair.zeta\_ne\_zero\_real} (zero custom axioms).
  \item If $\Im(s) \ne 0$: proved by \lean{AFEInfrastructure.off\_axis\_strip\_nonvanishing\_spiral}
    (Baker route / AFE attenuation route).
\end{itemize}
\end{theorem}

\begin{theorem}[Zero-Input Theory Closed]\label{thm:zero-input-closed}
\lean{Collatz.WeylIntegration.zero\_input\_theory}\leanok
\uses{def:zero-input, thm:strip-nonvanishing}
\lean{DirichletCompensatedNormLockingHypothesis} holds.
\emph{Proof.} By \Cref{thm:strip-nonvanishing}, $\zeta(s) \ne 0$ in the strip. The
EMD (Euler--Maclaurin) convergence theorem (\lean{dirichlet\_tube\_to\_zeta\_transfer\_emd})
gives $S(s,N) - N^{1-s}/(1-s) \to \zeta(s)$. Since $\|\zeta(s)\| > 0$, the sum is
eventually $\ge \|\zeta(s)\|/2 > 0$.
\end{theorem}

\begin{theorem}[Asymptotic Nonvanishing of the Spiral]\label{thm:spiral-nonvanishing}
\lean{Collatz.WeylIntegration.spiral\_asymptotic\_nonvanishing}\leanok
For $\tfrac12 < \Re(s) < 1$ and $\Im(s) \ne 0$, the partial sums $S(s,N)$ are eventually
nonzero: $\exists N_0\; \forall N \ge N_0,\; S(s,N) \ne 0$.
\emph{Proof.} \lean{BakerUncertainty.spiral\_nonvanishing\_sans\_baker} proves
$\|S(s,N)\| \ge c\,N^{1-\sigma}$, which grows without bound.
\end{theorem}

%% =============================================================================
\section{Harmonic Analysis: The 3-4-1 Method}
\label{sec:harmonic}
%% File: SpectralRH.lean (HarmonicRH namespace)
%% =============================================================================

The 3-4-1 trigonometric identity is the classical tool for proving zeros cannot
appear near $\sigma = 1$. It represents the finite-interference side of the proof,
contrasting with the infinite-interference content of RH.

\begin{theorem}[The 3-4-1 Trigonometric Identity]\label{thm:341-identity}
\lean{HarmonicRH.trig\_341\_eq}\leanok
For all $\theta \in \RR$:
\[
  3 + 4\cos\theta + \cos 2\theta = 2(1+\cos\theta)^2.
\]
\emph{Proof.} Expand $\cos 2\theta = 2\cos^2\theta - 1$ and simplify. Zero custom axioms.
\end{theorem}

\begin{theorem}[3-4-1 Nonnegativity]\label{thm:341-nonneg}
\lean{HarmonicRH.trig\_341\_nonneg}\leanok
\uses{thm:341-identity}
$3 + 4\cos\theta + \cos 2\theta \ge 0$ for all $\theta \in \RR$.
\emph{Proof.} $2(1+\cos\theta)^2 \ge 0$ since it is a square.
\end{theorem}

\begin{theorem}[Per-Prime Constructive Interference]\label{thm:prime-harmonic-nonneg}
\lean{HarmonicRH.prime\_harmonic\_nonneg}\leanok
\uses{thm:341-nonneg}
For $a \ge 0$ and $\theta \in \RR$: $3a + 4a\cos\theta + a\cos 2\theta \ge 0$.
This is the microscopic interference lemma: each prime harmonic contributes
constructively to the 3-4-1 sum.
\end{theorem}

\begin{axiom_block}[Mertens Product Inequality]\label{ax:mertens}
\lean{HarmonicRH.mertens\_inequality}\leanok
\uses{thm:prime-harmonic-nonneg}
For $\sigma > 1$ and $t \in \RR$:
\[
  \left|\zeta(\sigma)^3\,\zeta(\sigma+it)^4\,\zeta(\sigma+2it)\right| \ge 1.
\]
\emph{Proof sketch (axiomatized due to Euler product)}: Summing \Cref{thm:prime-harmonic-nonneg}
over all primes and harmonics in $\Re(\log\zeta)$ gives a nonnegative sum; exponentiating
yields the inequality.
\end{axiom_block}

\begin{theorem}[$\zeta(1+it) \ne 0$ (de la Vallée Poussin)]\label{thm:zeta-one-nonzero}
\lean{HarmonicRH.zeta\_ne\_zero\_on\_one}\leanok
\uses{ax:mertens}
For $t \ne 0$: $\zeta(1+it) \ne 0$.
\emph{Proof.} If $\zeta(1+it_0) = 0$, then as $\sigma \to 1^+$:
$|\zeta(\sigma)| \sim C/(\sigma-1)$ (pole) and $|\zeta(\sigma+it_0)| \sim C'(\sigma-1)$
(zero). The product $|\zeta(\sigma)^3\zeta(\sigma+it_0)^4\zeta(\sigma+2it_0)|
\lesssim K(\sigma-1) \to 0$, contradicting \Cref{ax:mertens} which requires $\ge 1$.
\end{theorem}

\begin{theorem}[Log-Linear Independence of Primes (General)]\label{thm:log-linear-indep}
\lean{HarmonicRH.log\_primes\_ne\_zero}\leanok
Let $p_1,\ldots,p_n$ be distinct primes and $c_1,\ldots,c_n \in \ZZ$ not all zero.
Then $\sum_i c_i \log p_i \ne 0$.
\emph{Proof.} Separate positive and negative coefficients, exponentiate to get
$\prod p_i^{a_i} = \prod p_i^{b_i}$, use unique factorization to conclude $a_i = b_i$
for all $i$, hence all $c_i = 0$. Zero custom axioms.
\end{theorem}

\begin{remark}[The Harmonic Wall]
The 3-4-1 method (length-2 trig polynomial) proves $\zeta \ne 0$ for $\sigma > 1 - c/\log|t|$.
Vinogradov uses length $(\log|t|)^{1/3}$, improving to $\sigma > 1 - c/(\log|t|)^{2/3}$.
To reach $\sigma = \tfrac12$, one would need a trig polynomial of length growing with $|t|$ —
an \emph{infinite} interference pattern. No fixed-length polynomial suffices (Fourier
uncertainty principle). RH is the assertion that this infinite interference is constructive.
\end{remark}

%% =============================================================================
\section{Hadamard Factorization Infrastructure}
\label{sec:hadamard}
%% Files: HadamardFactorization.lean, HadamardBridge.lean
%% =============================================================================

The completed zeta function $\xi(s) = \tfrac{s}{2}(s-1)\pi^{-s/2}\Gamma(s/2)\zeta(s)$
is entire of order~1. Hadamard's factorization theorem gives:
\[
  \xi(s) = \xi(0) \cdot e^{B_1 s} \cdot \prod_\rho E_1(s/\rho),
\]
where $E_1(z) = (1-z)e^z$ is the genus-1 Weierstrass factor and the product is over
nontrivial zeros~$\rho$. This is the master identity from which \emph{all} connections
between zeros and primes flow: the partial fraction of $\zeta'/\zeta$, the explicit
formula for $\psi(x)$, and zero density bounds.

\subsection{Proved Infrastructure (Zero Axioms)}

\begin{definition}[Weierstrass $E_1$ Factor]\label{def:E1}
\lean{HadamardFactorization.E\_1}\leanok
$E_1(z) = (1-z)e^z$. Proved: $E_1(0) = 1$, $E_1(1) = 0$, differentiability,
$\|E_1(z)-1\| \le |z|^2 e^{|z|}$ for $|z| \le 1/2$.
\end{definition}

\begin{definition}[$\xi(s)$]\label{def:xi}
\lean{HadamardFactorization.xi}\leanok
$\xi(s) = \tfrac{s}{2}(s-1)\cdot\lean{completedRiemannZeta}(s)$.
Proved: $\xi(1-s) = \xi(s)$ (functional equation), differentiability away from
$s=0,1$, and $\xi(s) = 0 \iff \lean{completedRiemannZeta}(s) = 0$ for $s \ne 0,1$.
\end{definition}

\subsection{Axioms}

\begin{axiom_block}[Zero Counting]\label{ax:zero-counting}
\lean{HadamardFactorization.zero\_counting\_bound}\leanok
$N(T) \le C \cdot T \log T$ for $T \ge 2$, where $N(T)$ counts nontrivial zeros
with $|\Im(\rho)| \le T$.

\emph{Source}: Jensen's formula applied to $\xi$ in a disk of radius $\sim T$.
Standard: Titchmarsh, \emph{Theory of the Riemann Zeta-Function}, \S9.4.
\end{axiom_block}

\begin{axiom_block}[Hadamard Partial Fraction]\label{ax:xi-logderiv}
\lean{HadamardFactorization.xi\_logderiv\_partial\_fraction}\leanok
\uses{def:xi}
For $\Re(s) > 1$, the partial fraction of $\xi'/\xi$:
\[
  \Re\!\left(\frac{\xi'}{\xi}(s)\right) = B_1 +
  \Re\!\left(\sum_{\rho:\,|\Im\rho|\le T} \left(\frac{1}{s-\rho}+\frac{1}{\rho}\right)\right)
  + O(\log T).
\]
The irreducible analytic content: Hadamard factorization for entire functions of order~$\le 1$.
Not in Mathlib (requires definition of order, proof that $\xi$ has order~1, convergence of
Weierstrass product). Standard: Conway, \emph{Functions of One Complex Variable II}, Ch.~XI.
\end{axiom_block}

\begin{axiom_block}[Log-Derivative Identity]\label{ax:logderiv-identity}
\lean{HadamardFactorization.logderiv\_identity}\leanok
\uses{def:xi}
For $\Re(s) > 1$:
\[
  \frac{\zeta'}{\zeta}(s) = \frac{\xi'}{\xi}(s) - \frac{1}{s-1} + \gamma(s),
  \quad |\Re(\gamma(s))| \le \log(|\Im(s)|+2) + 2.
\]
From $\xi(s) = \tfrac{s}{2}(s-1)\Gamma_\RR(s)\zeta(s)$ by logarithmic differentiation.
The $\gamma$-term absorbs $1/s$ and digamma asymptotics. Standard: Titchmarsh \S3.6.
\end{axiom_block}

\subsection{Derived Theorems}

\begin{theorem}[Partial Fraction of $-\zeta'/\zeta$]\label{thm:xi-hadamard-product}
\lean{HadamardFactorization.xi\_hadamard\_product}\leanok
\uses{ax:xi-logderiv, ax:logderiv-identity}
\textbf{PROVED} from the two axioms above. For $\sigma > 1$:
\[
  -\Re\!\left(\frac{\zeta'}{\zeta}(\sigma)\right) \ge
  \sum_{\rho:\,|\gamma|\le T} \frac{\sigma - \Re(\rho)}{|\sigma-\rho|^2} - C\log T.
\]
This is the gateway to the zero-free region and the 3-4-1 method.
\end{theorem}

\begin{theorem}[Hadamard Log-Derivative Bounds]\label{thm:hadamard-logderiv-bounds}
\lean{HadamardFactorization.hadamard\_logderiv\_bounds}\leanok
\uses{thm:xi-hadamard-product}
\textbf{PROVED}: for $\sigma > 1$, the weighted zero sum from $-\zeta'/\zeta$ is
bounded below by the contribution of any single zero.
\end{theorem}

\begin{theorem}[Zero Reciprocal Sum Converges]\label{thm:zero-reciprocal-converges}
\lean{HadamardFactorization.zero\_reciprocal\_sum\_converges}\leanok
\uses{ax:zero-counting}
\textbf{PROVED} from \Cref{ax:zero-counting}: $\sum_\rho 1/(1+\gamma^2) < \infty$.
\end{theorem}

\subsection{Hadamard Bridge: Explicit Formula}

The \lean{HadamardBridge} module (file \texttt{HadamardBridge.lean}) connects the
Hadamard factorization to the prime number theorem and explicit formula:

\begin{theorem}[Explicit Formula for $\psi$]\label{thm:explicit-formula-psi}
\lean{HadamardBridge.explicit\_formula\_psi}\leanok
\uses{thm:xi-hadamard-product}
\textbf{PROVED} from \lean{PerronFormula.perron\_explicit\_formula}
and \Cref{thm:xi-hadamard-product}: the Perron integral gives
$\psi(x) = x - \sum_\rho x^\rho/\rho + \ldots$
\end{theorem}

\begin{theorem}[RH $\Rightarrow$ $\psi$ Error Bound]\label{thm:rh-psi-error}
\lean{HadamardBridge.rh\_implies\_psi\_error}\leanok
\uses{thm:explicit-formula-psi}
Under RH: $|\psi(x) - x| \le C\sqrt{x}(\log x)^2$.
\end{theorem}

%% =============================================================================
\section{The Von Mangoldt Spectral Route (Primary)}
\label{sec:von-mangoldt}
%% File: RH.lean (MellinVonMangoldt namespace)
%% =============================================================================

This is the primary proof route targeted for full Lean 4 formalization. It uses
all three mechanisms: \emph{rotation} to expose codimension, \emph{Mellin} to
convert zeros to $L^2$ modes, and \emph{spectral completeness} to derive the
contradiction.

\subsection{How Rotation and Mellin Work Together}

The von Mangoldt explicit formula
$\psi(x) = x - \sum_\rho x^\rho/\rho + \cdots$
decomposes the prime counting function into contributions indexed by the
nontrivial zeros $\rho$. Substituting $u = \log x$ (the Mellin variable),
the zero $\rho$ contributes a term proportional to $e^{\rho u}/\rho$.
The exponentials $e^{i\gamma u}$ are \emph{mode labels} (generalized
eigenmodes / evaluation characters), not elements of the Hilbert space ---
they index the spectral components but are not themselves square-integrable.

The rotation $w = -i(s - \tfrac12)$ maps $\rho = \tfrac12 + i\gamma$ (on-line) to
$w = \gamma$ (real), and $\rho = \sigma + i\gamma$ with $\sigma \ne \tfrac12$
(off-line) to $w = \gamma - i(\sigma - \tfrac12)$ (complex). This is the
\textbf{frequency interpretation}:
\begin{itemize}
  \item \textbf{On-line zeros} $\to$ real frequency labels $\gamma \in \RR$ $\to$
    spectral components on the boundary of the rotated strip.
  \item \textbf{Off-line zeros} $\to$ complex frequency labels $\gamma - i\alpha$
    ($\alpha \ne 0$) $\to$ spectral components in the interior of the strip.
\end{itemize}

The \textbf{Mellin--Parseval isometry} converts the contour separation
(on-line modes on $\Re(s) = \tfrac12$, off-line modes on $\Re(s) = \sigma
\ne \tfrac12$) into orthogonality in the spectral Hilbert space $\mathcal{H}$:
the spectral component associated with an off-line zero is orthogonal to the
closed span of the on-line family.

This is why both rotation and Mellin are needed:
\begin{itemize}
  \item Rotation makes the on-line/off-line distinction geometric:
    boundary vs interior of the strip.
  \item Mellin maps the boundary/interior separation to orthogonality
    in $\mathcal{H}$.
  \item Together: an off-line zero produces a nonzero element of $\mathcal{H}$
    orthogonal to a complete family $\Rightarrow$ contradiction.
\end{itemize}

\subsection{The Spectral Hilbert Space $\mathcal{H}$}

The spectral analysis requires a Hilbert space $\mathcal{H}$ and a
\emph{complete family} of spectral components indexed by the on-line zeros.
The exponentials $e^{i\gamma u}$ are \textbf{not elements of $\mathcal{H}$} ---
they are mode labels (generalized eigenmodes / evaluation characters) that
index the spectral decomposition. The Hilbert space structure lives on the
\emph{coefficients}, not the modes.

\begin{definition}[Hardy Space $H^2(S_{1/2})$]\label{def:hardy-strip}
The Hardy space of the strip $S_{1/2} = \{w \in \CC : |\Im(w)| < \tfrac12\}$ is
\[
  H^2(S_{1/2}) = \bigl\{f \text{ analytic in } S_{1/2} :
    \sup_{|y|<1/2} \int_{-\infty}^{\infty} |f(x+iy)|^2\,dx < \infty\bigr\}.
\]
This is a separable complex Hilbert space. The strip width $\tfrac12$ matches the
rotated critical strip exactly: the coordinate change $w = -i(s-\tfrac12)$ maps
$0 < \Re(s) < 1$ to $|\Im(w)| < \tfrac12$.
\end{definition}

\begin{remark}[State space vs mode labels]
The spectral decomposition involves two distinct objects:
\begin{enumerate}
  \item The \textbf{state space} $\mathcal{H}$ (a Hilbert space): functions
    analytic in the strip with finite Hardy norm. Elements of $\mathcal{H}$
    are genuine $L^2$ objects.
  \item The \textbf{mode labels} $\{e^{i\gamma u}\}_{\gamma \in \RR}$
    (generalized eigenmodes): these are distributional / kernel objects that
    parametrize the spectral decomposition. They are \emph{not} elements
    of $\mathcal{H}$, just as plane waves $e^{ikx}$ are not elements of
    $L^2(\RR)$ but parametrize the Fourier transform.
\end{enumerate}
The on-line zero frequencies $\{\gamma_n\}$ define a complete family of
\emph{reproducing kernel evaluations} in $\mathcal{H}$. Completeness means:
the only element of $\mathcal{H}$ orthogonal to the closed span of the
on-line kernel components is zero.
\end{remark}

\begin{remark}[Strip membership of $\xi_{\mathrm{rot}}$]
$\xi_{\mathrm{rot}}(w) = \xi(\tfrac12+iw)$ is entire (since $\xi$ is entire)
and in particular analytic in the strip $|\Im(w)| < \tfrac12$. However,
$\xi_{\mathrm{rot}}$ grows polynomially along horizontal lines
(Phragm\'en--Lindel\"of convexity bound), so
$\xi_{\mathrm{rot}} \notin H^2(S_{1/2})$ directly. The spectral analysis
operates on the \emph{logarithmic derivative} $\xi_{\mathrm{rot}}'/\xi_{\mathrm{rot}}$,
whose residues at the zeros produce the spectral components in $\mathcal{H}$.
Alternatively, a polynomial normalization (dividing by $w^k$ or a Weierstrass
product) can place the object in $H^2$.
\end{remark}

\begin{definition}[Mellin $L^2$ Space (Lean carrier)]\label{def:mellin-l2}
\lean{MellinL2}\leanok
$\lean{MellinL2} = L^2(\RR, \CC)$ with respect to Lebesgue measure.
In the Lean formalization, this serves as the abstract carrier type for
$\mathcal{H}$. Since all infinite-dimensional separable Hilbert spaces are
isometrically isomorphic ($L^2(\RR) \cong H^2(S_{1/2}) \cong \ell^2(\NN)$),
the carrier is interchangeable. The mathematical content --- which family of
spectral components is complete, which elements are orthogonal --- is encoded
in the axioms, not the carrier type.
\end{definition}

\subsection{Spectral Axioms and Proof}

\begin{axiom_block}[On-Line Spectral Completeness (von Mangoldt + Beurling--Malliavin)]\label{ax:online-basis}
\lean{MellinVonMangoldt.onLineBasis}\leanok
\uses{def:mellin-l2}
The on-line zero frequencies $\{\gamma_n\}$ (where $\rho_n = \tfrac12 + i\gamma_n$,
rotating to real points $w_n = \gamma_n$) define a \emph{complete family} of
spectral components in $\mathcal{H}$: a \lean{HilbertBasis} indexed by $\NN$.

Mathematically: the reproducing kernel evaluations at the boundary zero
positions form a complete system in $\mathcal{H}$. Completeness follows from
the Beurling--Malliavin density theorem: the zero density
$N(T) \sim (T/2\pi)\log(T/2\pi e)$ exceeds the B-M critical density for
every finite interval, so the system is complete.

The exponentials $e^{i\gamma_n u}$ are the \emph{mode labels} (frequency
parameters) that index this family, not basis vectors in $\mathcal{H}$.

\emph{References}: von Mangoldt (1895), Beurling--Malliavin (1962).
\end{axiom_block}

\begin{axiom_block}[Off-Line Orthogonal Witness (Mellin)]\label{ax:offline-component}
\lean{MellinVonMangoldt.offLineHiddenComponent}\leanok
\uses{def:mellin-l2, ax:online-basis}
If $\rho$ is a zero of $\zeta$ in the critical strip with $\Re(\rho) \ne \tfrac12$
(equivalently, $w = \gamma - i(\sigma-\tfrac12)$ is an interior point of the
rotated strip), then there exists a nonzero element of $\mathcal{H}$ orthogonal
to every element of the on-line family (\Cref{ax:online-basis}).

The interior zero induces a spectral component at a complex frequency label
$\gamma - i\alpha$ ($\alpha = \sigma - \tfrac12 \ne 0$). The Mellin--Parseval
isometry converts the contour separation ($\Re(s) = \sigma \ne \tfrac12$)
into orthogonality in $\mathcal{H}$: the off-line component's Mellin contour
does not coincide with the on-line contour $\Re(s) = \tfrac12$, producing
a nonzero element in the orthogonal complement of the on-line span.

\emph{Reference}: Mellin (1902) transform inversion.
\end{axiom_block}

\begin{theorem}[Exponential Growth Outside $L^2$]\label{thm:exp-not-l2}
\lean{MellinVonMangoldt.not\_memLp\_exp\_nonzero}\leanok
For $\alpha \ne 0$: the function $u \mapsto e^{\alpha u}$ is not in $L^2(\RR)$.
\emph{Proof.} For $\alpha > 0$: restrict to $[0,\infty)$; $\|1\|^2_{L^2([0,\infty))} = \infty$
and $|e^{\alpha u}| \ge 1$. For $\alpha < 0$: restrict to $(-\infty, 0]$ similarly.
Zero custom axioms.
\end{theorem}

\begin{theorem}[Bounded Growth iff $\alpha = 0$]\label{thm:exp-bounded-iff-zero}
\lean{MellinVonMangoldt.exp\_bounded\_iff\_zero}\leanok
\uses{thm:exp-not-l2}
$e^{\alpha u}$ is bounded on $\RR$ if and only if $\alpha = 0$.
\emph{Proof.} If $\alpha \ne 0$: for any $C$, taking $u = (C+1)/\alpha$ (if $\alpha>0$)
or $u = -(C+1)/|\alpha|$ (if $\alpha < 0$) gives $e^{\alpha u} > C$. Zero custom axioms.
\end{theorem}

\begin{theorem}[Von Mangoldt Mode Bounded]\label{thm:vm-mode-bounded}
\lean{MellinVonMangoldt.vonMangoldt\_mode\_bounded}\leanok
\uses{ax:online-basis, ax:offline-component, thm:no-hidden-component}
For every zero $\rho$ of $\zeta$ in the critical strip: there exists $C \in \RR$
with $e^{(\Re(\rho)-1/2)u} \le C$ for all $u \in \RR$.
\emph{Equivalently}: $\Re(\rho) = \tfrac12$.
\emph{Proof.} Suppose $\Re(\rho) \ne \tfrac12$. By \Cref{ax:offline-component}, there
exists a nonzero $f \in L^2(\RR)$ orthogonal to every on-line basis element. By
\Cref{thm:no-hidden-component} (no hidden component), $f = 0$ — contradiction.
Therefore $\Re(\rho) = \tfrac12$, so the exponent is $0$ and $e^{0 \cdot u} = 1 \le 1$.
\end{theorem}

\begin{theorem}[Explicit Formula Completeness]\label{thm:explicit-formula-completeness}
\lean{explicit\_formula\_completeness\_proved}\leanok
\uses{thm:vm-mode-bounded, thm:exp-bounded-iff-zero}
Every zero $\rho$ of $\zeta$ in the critical strip satisfies $\Re(\rho) = \tfrac12$.
\emph{Proof.} Suppose $\Re(\rho) \ne \tfrac12$. By \Cref{thm:vm-mode-bounded}, the
spectral mode $e^{(\Re(\rho)-1/2)u}$ is bounded. But $\Re(\rho) - \tfrac12 \ne 0$, so
by \Cref{thm:exp-bounded-iff-zero} it is unbounded — contradiction.
\end{theorem}

%% =============================================================================
\section{The Motohashi Spectral Route}
\label{sec:motohashi}
%% File: MotohashiRH.lean
%% =============================================================================

This section presents a second spectral proof of RH, via Motohashi's decomposition
of the fourth moment of $\zeta$ over the Maass eigenforms of $\mathrm{SL}_2(\ZZ)\backslash\mathbb{H}$.
The \emph{logical pattern} is identical to the von Mangoldt route (\Cref{sec:von-mangoldt}):
a complete Hilbert basis + orthogonal witness $\Rightarrow$ contradiction. The
\emph{mechanism} is different: completeness comes from the self-adjointness of the
hyperbolic Laplacian (Selberg 1956), not from Beurling--Malliavin density. Neither
rotation nor Mellin is formally used.

\subsection{Relationship to Rotation (Motivational, Not Structural)}

In the rotated frame, $\xi_{\mathrm{rot}}(t) = \xi(\tfrac12+it)$ is real on $\RR$
(\Cref{thm:xirot-real-reals}), so $|\xi_{\mathrm{rot}}(t)|^4 = \xi_{\mathrm{rot}}(t)^4$
and Motohashi's fourth moment expansion captures the full behavior of this real function.
This is mathematically natural context for \emph{why} the Motohashi expansion works,
but it is \textbf{not part of the formal proof chain}. The Lean proof uses only the
Maass basis completeness and the orthogonal witness, without passing through the
rotated-coordinate machinery. In particular, the Mellin transform plays no role here:
the spectral decomposition is over automorphic forms on the modular surface, not
Fourier modes on $\RR$.

\subsection{The Maass L$^2$ Space}

\begin{definition}[Maass $L^2$ Space]\label{def:maass-l2}
\lean{MotohashiRH.MaassL2}\leanok
$\lean{MaassL2} = L^2(\RR, \CC)$ with respect to Lebesgue measure. This is the
same carrier type as \lean{MellinL2} (\Cref{def:mellin-l2}); both are separable
complex Hilbert spaces $\cong \ell^2(\NN)$. The distinction is in the
\emph{basis}: Maass cusp forms + Eisenstein series (Selberg) versus exponential
modes $e^{i\gamma_n u}$ (von Mangoldt).

\emph{Morally}: $L^2(\mathrm{SL}_2(\ZZ)\backslash\mathbb{H},\, y^{-2}\,dx\,dy)$,
the $L^2$ space on the modular surface with hyperbolic measure. The opaque type
suffices for the formal proof.
\end{definition}

\subsection{Axioms}

\begin{axiom_block}[Selberg Maass Basis]\label{ax:selberg-maass}
\lean{MotohashiRH.selbergMaassBasis}\leanok
\uses{def:maass-l2}
The Maass cusp forms $\{u_j\}$ together with the Eisenstein series form a complete
orthonormal basis (\lean{HilbertBasis}) of $L^2(\mathrm{SL}_2(\ZZ)\backslash\mathbb{H})$.

\emph{Source}: the self-adjointness of the hyperbolic Laplacian
$\Delta = -y^2(\partial^2/\partial x^2 + \partial^2/\partial y^2)$
on the finite-volume quotient $\mathrm{SL}_2(\ZZ)\backslash\mathbb{H}$:
\begin{itemize}
  \item Self-adjoint $\Rightarrow$ real eigenvalues $\lambda_j = \tfrac14 + t_j^2$ ($t_j \in \RR$).
  \item Compact resolvent on cuspidal subspace $\Rightarrow$ discrete spectrum.
  \item Spectral theorem $\Rightarrow$ complete orthonormal basis.
\end{itemize}
Standard functional analysis. No Beurling--Malliavin density theory.

\emph{References}: Selberg (1956), Bump Ch.~2, Iwaniec--Kowalski Ch.~15.
\end{axiom_block}

\begin{axiom_block}[Motohashi Off-Line Witness]\label{ax:motohashi-witness}
\lean{MotohashiRH.motohashiOffLineWitness}\leanok
\uses{def:maass-l2, ax:selberg-maass}
If $\rho$ is a zero of $\zeta$ in the critical strip with $\Re(\rho) \ne \tfrac12$,
then there exists a nonzero $f \in L^2(\mathrm{SL}_2(\ZZ)\backslash\mathbb{H})$
satisfying $\langle u_n, f\rangle = 0$ for every Maass basis element $u_n$:
\[
  \exists\, f \in \lean{MaassL2},\quad f \ne 0 \;\wedge\;
  \forall\, n \in \NN,\quad
  \langle \lean{selbergMaassBasis}(n),\, f \rangle = 0.
\]

\emph{Mathematical content}: Motohashi's formula (Acta Math.~170, 1993) decomposes
$|\xi_{\mathrm{rot}}(t)|^4 = \xi_{\mathrm{rot}}(t)^4$ (real!) into a sum over the
Maass spectrum. Each zeta zero $\rho$ contributes a spectral mode:
\begin{itemize}
  \item \textbf{On-line zeros} ($\Re(\rho) = \tfrac12$, rotating to $\RR$):
    contribute modes \emph{within} the Maass spectral expansion --- they are
    ``matched'' by Maass eigenvalues.
  \item \textbf{Off-line zeros} ($\Re(\rho) \ne \tfrac12$, rotating off $\RR$):
    contribute modes at a different ``spectral level'' --- unmatched by any
    Maass eigenvalue.
\end{itemize}
The Kuznetsov trace formula (which underlies Motohashi's identity) maps sums over
zeta zeros to sums over Maass eigenvalues. An off-line zero produces an unmatched
residue --- a nonzero $L^2$ element with zero projection onto every Maass eigenspace.

\emph{References}: Motohashi, Acta Math.~170 (1993), 181--220.
Motohashi, \emph{Spectral Theory of the Riemann Zeta-Function}, Cambridge (1997).
\end{axiom_block}

\subsection{Proof Chain}

\begin{theorem}[Motohashi Excludes Off-Line Zeros]\label{thm:motohashi-excludes}
\lean{MotohashiRH.motohashi\_excludes\_offLine}\leanok
\uses{ax:selberg-maass, ax:motohashi-witness, thm:no-hidden-component}
For every zero $\rho$ of $\zeta$ in the critical strip: $\Re(\rho) = \tfrac12$.

\emph{Proof.} By contradiction. Suppose $\Re(\rho) \ne \tfrac12$.
By \Cref{ax:motohashi-witness}, there exists a nonzero $f$ orthogonal to
every element of the Maass basis \Cref{ax:selberg-maass}. By
\Cref{thm:no-hidden-component} (\lean{abstract\_no\_hidden\_component},
proved, zero axioms), no nonzero element can be orthogonal to a complete
Hilbert basis. Contradiction.
\end{theorem}

\begin{theorem}[RH: Motohashi Spectral Route]\label{thm:rh-motohashi}
\lean{MotohashiRH.riemann\_hypothesis\_motohashi}\leanok
\uses{thm:motohashi-excludes}
$\lean{RiemannHypothesis}$ holds.
\emph{Axioms}: 2 custom axioms (\Cref{ax:selberg-maass}, \Cref{ax:motohashi-witness}).
\emph{Proof.} Compose \Cref{thm:motohashi-excludes} with
\lean{riemann\_hypothesis\_fourier} (the bridge from strip nonvanishing to
Mathlib's \lean{RiemannHypothesis}).
\end{theorem}

\subsection{Single-Axiom Consolidation}

\begin{axiom_block}[Motohashi Spectral Exclusion (Consolidated)]\label{ax:motohashi-consolidated}
\lean{MotohashiRH.motohashi\_spectral\_exclusion}\leanok
For $\rho \in \CC$ with $\zeta(\rho) = 0$, $0 < \Re(\rho) < 1$, and
$\Re(\rho) \ne \tfrac12$: \textbf{False}.

Combines Selberg completeness (\Cref{ax:selberg-maass}) +
Motohashi witness (\Cref{ax:motohashi-witness}) +
Hilbert basis completeness (\Cref{thm:no-hidden-component}).
\end{axiom_block}

\begin{theorem}[RH: Motohashi 1-Axiom Route]\label{thm:rh-motohashi-1ax}
\lean{MotohashiRH.riemann\_hypothesis\_motohashi\_1ax}\leanok
\uses{ax:motohashi-consolidated}
$\lean{RiemannHypothesis}$ holds from a single axiom.
\end{theorem}

\subsection{Comparison: Fourier vs Motohashi}

\begin{table}[h]
\centering
\begin{tabular}{llllll}
\hline
\textbf{Route} & \textbf{Axiom 1} & \textbf{Axiom 2} & \textbf{Completeness} & \textbf{Rotation} & \textbf{Mellin} \\
\hline
Fourier/B-M & \lean{onLineBasis} & \lean{offLineHiddenComponent} & B-M 1962 & \checkmark & \checkmark \\
Motohashi & \lean{selbergMaassBasis} & \lean{motohashiOffLineWitness} & Selberg 1956 & --- & --- \\
\hline
\end{tabular}
\caption{Both routes share the logical pattern: complete basis + orthogonal witness
$\Rightarrow$ \lean{abstract\_no\_hidden\_component} $\Rightarrow$ contradiction.
The Fourier route uses rotation (codimension) and Mellin (zero $\to$ $L^2$ mode).
The Motohashi route uses automorphic spectral theory (Selberg's self-adjoint Laplacian)
without either.}
\end{table}

The Motohashi axioms cite:
\begin{itemize}
  \item Selberg (1956) --- in every graduate textbook on automorphic forms.
  \item Motohashi (1993) --- Acta Mathematica, the top journal.
  \item Self-adjoint spectral theorem --- undergraduate functional analysis.
\end{itemize}
versus the Fourier axioms citing:
\begin{itemize}
  \item Beurling--Malliavin (1962) --- specialized harmonic analysis, density of
    exponential systems, not widely known outside the field.
\end{itemize}

%% =============================================================================
\section{The Four Proof Endpoints}
\label{sec:rh-endpoints}
%% File: RH.lean, MotohashiRH.lean
%% =============================================================================

\begin{theorem}[RH: Baker/Spiral Route]\label{thm:rh-baker}
\lean{riemann\_hypothesis\_unconditional\_baker}\leanok
\uses{ax:baker, thm:strip-nonvanishing}
$\lean{RiemannHypothesis}$ holds: every nontrivial zero of $\zeta$ lies on $\Re(s) = \tfrac12$.
\emph{Axioms}: 1 custom axiom (\Cref{ax:baker}, Baker 1966).
\emph{Proof chain}: \Cref{ax:baker} $\Rightarrow$ \Cref{thm:strip-nonvanishing-baker}
$\Rightarrow$ \lean{strip\_nonvanishing\_zero\_input}
$\Rightarrow$ \lean{riemann\_hypothesis\_unconditional\_baker}.
\end{theorem}

\begin{theorem}[RH: Fourier Spectral Route (\textsc{primary})]\label{thm:rh-fourier}
\lean{riemann\_hypothesis\_fourier\_unconditional}\leanok
\uses{ax:online-basis, ax:offline-component, thm:explicit-formula-completeness}
$\lean{RiemannHypothesis}$ holds. This is the primary route targeted for full
Lean formalization.
\emph{Axioms}: 2 custom axioms (\Cref{ax:online-basis}, \Cref{ax:offline-component}).
\emph{Mechanisms}: rotation (codimension) $+$ Mellin (zero $\to$ $L^2$ mode) $+$
Beurling--Malliavin (completeness).
\emph{Proof chain}: \Cref{thm:explicit-formula-completeness} $\Rightarrow$
\lean{riemann\_hypothesis\_fourier} $\Rightarrow$
\lean{riemann\_hypothesis\_fourier\_unconditional}.
\end{theorem}

\begin{theorem}[RH: Motohashi Spectral Route]\label{thm:rh-motohashi-endpoint}
\lean{MotohashiRH.riemann\_hypothesis\_motohashi}\leanok
\uses{ax:selberg-maass, ax:motohashi-witness, thm:motohashi-excludes}
$\lean{RiemannHypothesis}$ holds.
\emph{Axioms}: 2 custom axioms (\Cref{ax:selberg-maass}, \Cref{ax:motohashi-witness}).
\emph{Mechanisms}: Selberg spectral completeness only (no rotation, no Mellin).
\emph{Proof chain}: \Cref{thm:motohashi-excludes}
$\Rightarrow$ \lean{riemann\_hypothesis\_fourier}
$\Rightarrow$ \lean{riemann\_hypothesis\_motohashi}.
\end{theorem}

\begin{theorem}[RH: Conditional/Rotation Route]\label{thm:rh-conditional}
\lean{ExplicitFormulaBridge.riemann\_hypothesis}\leanok
%% Note: ExplicitFormulaBridge is a namespace inside RotatedZeta.lean, not a separate file.
\uses{thm:rotation-isometry}
Assuming \lean{explicit\_formula\_completeness} as a hypothesis
(von Mangoldt 1895 + Mellin 1902 + Parseval):
every nontrivial zero of $\zeta$ lies on $\Re(s) = \tfrac12$.
\emph{Axioms}: 0 custom axioms; the explicit formula completeness is passed as a
theorem argument.
\emph{Proof.} The hypothesis directly gives the conclusion.
\end{theorem}

\begin{theorem}[RH is Equivalent to Zero-Input Theory]\label{thm:rh-equiv-zero-input}
\lean{zero\_input\_theorem}\leanok
\uses{def:zero-input, thm:rh-baker}
$\lean{RiemannHypothesis} \iff \lean{ZeroInputTheory}$.
\emph{Proof.} Forward: if $\zeta \ne 0$ in the strip, the EMD convergence makes
compensated partial sums eventually $\ge \|\zeta(s)\|/2$. Backward: the compensated
sum lower bound forces $\zeta(s) \ne 0$.
\end{theorem}

%% =============================================================================
\section{Axiom Inventory}
\label{sec:axioms}
%% =============================================================================

\begin{table}[h]
\centering
\begin{tabular}{lll}
\hline
\textbf{Axiom name} & \textbf{Used in route} & \textbf{Reference} \\
\hline
\lean{baker\_forbids\_pole\_hit} & Baker/spiral & Baker (1966) \\
\lean{MellinVonMangoldt.onLineBasis} & Fourier spectral & von Mangoldt (1895), B-M (1962) \\
\lean{MellinVonMangoldt.offLineHiddenComponent} & Fourier spectral & Mellin (1902) \\
\lean{MotohashiRH.selbergMaassBasis} & Motohashi spectral & Selberg (1956) \\
\lean{MotohashiRH.motohashiOffLineWitness} & Motohashi spectral & Motohashi (1993) \\
\lean{vonMangoldt\_spectral\_exclusion} & Fourier (1-axiom) & vM + Mellin + B-M \\
\lean{motohashi\_spectral\_exclusion} & Motohashi (1-axiom) & Selberg + Motohashi \\
\lean{zero\_counting\_bound} & Hadamard infrastructure & Jensen's formula \\
\lean{xi\_logderiv\_partial\_fraction} & Hadamard infrastructure & Hadamard (1893) \\
\lean{logderiv\_identity} & Hadamard infrastructure & Titchmarsh \S3.6 \\
\lean{mertens\_inequality} & Harmonic (structural) & Classical \\
\lean{classical\_zero\_free\_region} & Harmonic (structural) & de la Vallée Poussin (1899) \\
\hline
\end{tabular}
\caption{Custom axioms in the RH proof chain. All are proved theorems in the literature.}
\end{table}

\begin{remark}[Single-Axiom Consolidation Routes]
Both the Fourier and Motohashi 2-axiom routes admit 1-axiom consolidations
that combine the basis axiom and the witness axiom into a single spectral
exclusion principle:

\begin{itemize}
  \item \lean{vonMangoldt\_spectral\_exclusion} (RH.lean): if $\rho$ is a zero of
    $\zeta$ in the critical strip with $\Re(\rho) \ne \tfrac12$, then \textbf{False}.
    Combines \lean{onLineBasis} + \lean{offLineHiddenComponent} +
    \lean{abstract\_no\_hidden\_component}. Yields
    \lean{riemann\_hypothesis\_1ax} from 1 axiom.
  \item \lean{motohashi\_spectral\_exclusion} (MotohashiRH.lean): same statement,
    combining \lean{selbergMaassBasis} + \lean{motohashiOffLineWitness} +
    \lean{abstract\_no\_hidden\_component}. Yields
    \lean{riemann\_hypothesis\_motohashi\_1ax} from 1 axiom.
\end{itemize}

The 2-axiom formulations are preferred for clarity (they separate ``the basis is complete''
from ``off-line zeros produce orthogonal witnesses''). The 1-axiom formulations are available
for downstream consumers that prefer minimal axiom counts.
\end{remark}

\begin{remark}[Proven Infrastructure (Zero Custom Axioms)]
The following are all proved from Mathlib alone, with zero custom axioms:
\begin{itemize}
  \item Schwarz reflection (\Cref{thm:schwarz-reflection})
  \item $\xi$ real on critical line (\Cref{thm:xi-real-critical})
  \item $D_4$ symmetry of $\xi_{\mathrm{rot}}$ (\Cref{thm:xirot-symmetry})
  \item No zeros on the imaginary axis in the strip (\Cref{thm:xirot-no-zeros-imag})
  \item Log-independence of primes (\Cref{thm:log-independence}, \Cref{thm:log-linear-indep})
  \item Helix uncertainty principle (\Cref{thm:helix-uncertainty})
  \item At most one prime with $\sin(t\log p) = 0$ (\Cref{thm:at-most-one-sin})
  \item $\xi'$ purely imaginary on critical line (\Cref{thm:xi-deriv-imag})
  \item Zeros of $\xi$ isolated in strip (\Cref{thm:xi-zeros-isolated})
  \item Parseval identity (\Cref{thm:parseval})
  \item Rotation is isometry (\Cref{thm:rotation-isometry})
  \item 3-4-1 trigonometric identity (\Cref{thm:341-identity})
  \item $\zeta(1+it) \ne 0$ (\Cref{thm:zeta-one-nonzero}) [from \lean{mertens\_inequality}]
  \item Abstract spectral gap (\Cref{thm:rotation-spectral-gap})
\end{itemize}
\end{remark}

\begin{remark}[Axiom Weight and Remaining Formalization Gap]
The spectral axioms carry the bulk of the analytic content. Once accepted, the
formal proof after them is one step: apply \lean{abstract\_no\_hidden\_component}.
The axioms are not trivial re-statements of RH --- they have genuine mathematical
content (Mellin contour orthogonality / Kuznetsov trace formula for the witness,
Beurling--Malliavin density for completeness) --- but they compress the hardest
analytic work into an axiom boundary. The primary formalization targets:
\begin{enumerate}
  \item \textbf{Functional-analytic interface}: formalize the Hardy space
    $H^2(S_{1/2})$ or a spectral-measure $L^2$ space as the concrete
    $\mathcal{H}$, replacing the abstract carrier. This removes the
    ``type-correctness'' objection (that $e^{i\gamma u} \notin L^2(\RR)$)
    by working in the right space from the start.
  \item \textbf{Mellin--Parseval isometry}: prove contour separation
    $\Rightarrow$ orthogonality in $\mathcal{H}$, reducing
    \lean{offLineHiddenComponent}.
  \item \textbf{Beurling--Malliavin density}: prove zero density exceeds
    critical density $\Rightarrow$ completeness of the on-line family,
    reducing \lean{onLineBasis}.
\end{enumerate}
Note: fixing (1) changes the axiom \emph{type-correctness} but not the axiom
\emph{strength}. The hardest content remains in (2) and (3).
\end{remark}

\begin{remark}[Beurling Necessity]
The axiom \lean{rotation\_forbids\_off\_axis} (\Cref{ax:rotation-forbids}) is sharp:
for Beurling generalized prime systems with $\QQ$-commensurable logarithms
($\log b_j / \log b_k \in \QQ$), the rotated zeta function does have non-real zeros
(Diamond--Montgomery--Vorhauer 2006, formalized in \lean{BeurlingCounterexample}).
The full arithmetic rigidity of the primes (unique factorization + $\QQ$-independent
logarithms) is required to rule out off-axis zeros.
\end{remark}

%% =============================================================================
\section{Proof Chain Summary}
\label{sec:summary}
%% =============================================================================

\subsection*{Baker Route (1 axiom)}
\[
\text{\lean{baker\_forbids\_pole\_hit}}
\xRightarrow{\text{\ref{thm:strip-nonvanishing-baker}}}
\zeta \ne 0 \text{ in strip}
\xRightarrow{\text{\ref{thm:strip-nonvanishing}}}
\text{\lean{LogEulerSpiralNonvanishing}}
\xRightarrow{}
\text{\lean{RiemannHypothesis}}
\]

\subsection*{Fourier/von Mangoldt Route (2 axioms, \textsc{primary})}

\emph{Mechanisms}: rotation (codimension) $\to$ Mellin (zero $\to$ $L^2$ mode) $\to$
B-M completeness $\to$ \lean{abstract\_no\_hidden\_component}.

\[
\begin{array}{c}
\text{\lean{onLineBasis}} + \text{\lean{offLineHiddenComponent}}\\[2pt]
\Downarrow \text{\ref{thm:vm-mode-bounded}} \\[2pt]
\text{all zeros have } \Re(\rho) = \tfrac12\\[2pt]
\Downarrow \text{\ref{thm:explicit-formula-completeness}} \\[2pt]
\text{\lean{riemann\_hypothesis\_fourier\_unconditional}}
\end{array}
\]

\subsection*{Motohashi Spectral Route (2 axioms)}

\emph{Mechanisms}: Selberg completeness (self-adjoint Laplacian) $\to$
\lean{abstract\_no\_hidden\_component}. No rotation, no Mellin.

\[
\begin{array}{c}
\text{\lean{selbergMaassBasis}} + \text{\lean{motohashiOffLineWitness}}\\[2pt]
\Downarrow \text{\ref{thm:motohashi-excludes}} \\[2pt]
\text{all zeros have } \Re(\rho) = \tfrac12\\[2pt]
\Downarrow \\[2pt]
\text{\lean{riemann\_hypothesis\_motohashi}}
\end{array}
\]

\subsection*{Fourier 1-Axiom Route}
\[
\text{\lean{vonMangoldt\_spectral\_exclusion}}
\;\Longrightarrow\;
\text{\lean{riemann\_hypothesis\_1ax}}
\]

\subsection*{Motohashi 1-Axiom Route}
\[
\text{\lean{motohashi\_spectral\_exclusion}}
\;\Longrightarrow\;
\text{\lean{riemann\_hypothesis\_motohashi\_1ax}}
\]

\subsection*{Conditional/Rotation Route (0 axioms)}
\[
\text{hypothesis: \lean{explicit\_formula\_completeness}}
\xRightarrow{\text{\ref{thm:rh-conditional}}}
\text{\lean{RiemannHypothesis}}
\]

\subsection*{Three Mechanisms}

The proof architecture uses three mechanisms, combined differently in each route:

\begin{enumerate}
  \item \textbf{Rotation} ($w = -i(s-\tfrac12)$) maps the critical line to $\RR$ and
    exposes the codimension gap:
    on-line zeros of the real-valued $\xi_{\mathrm{rot}}$ are codimension~1 (generic);
    off-line zeros are codimension~2 (both $\Re$ and $\Im$ must vanish).
    \emph{Used in}: Baker (structurally), Fourier/vM (frequency interpretation).
    \emph{Motivational only}: Motohashi.

  \item \textbf{Mellin transform} converts $\zeta$-zeros into $L^2(\RR)$ spectral modes.
    On-line zeros $\to$ oscillatory modes $e^{i\gamma u}$ (real frequencies in the rotated
    frame); off-line zeros $\to$ exponentially growing modes $e^{\alpha u}$ (complex
    frequencies). Contour separation gives $L^2$ orthogonality.
    \emph{Used in}: Fourier/vM (the zero-to-$L^2$-mode bridge).
    \emph{Not used}: Baker (works directly on the Euler product), Motohashi (works on
    Maass forms on $\mathrm{SL}_2(\ZZ)\backslash\mathbb{H}$, not Fourier modes on $\RR$).

  \item \textbf{Spectral completeness}: a complete Hilbert basis admits no nonzero
    orthogonal element (\lean{abstract\_no\_hidden\_component}, proved, 0~axioms).
    An off-line zero produces such an element $\Rightarrow$ contradiction.
    \emph{Used in}: Fourier/vM (B-M completeness), Motohashi (Selberg completeness).
    \emph{Not used}: Baker (uses phase rigidity, not basis completeness).
\end{enumerate}

\begin{table}[h]
\centering
\begin{tabular}{lccc}
\hline
\textbf{Route} & \textbf{Rotation} & \textbf{Mellin} & \textbf{Spectral completeness} \\
\hline
Baker/spiral & \checkmark & --- & --- \\
Fourier/vM (\textsc{primary}) & \checkmark & \checkmark & \checkmark (B-M) \\
Motohashi & \emph{motivational} & --- & \checkmark (Selberg) \\
Conditional & --- & --- & (hypothesis) \\
\hline
\end{tabular}
\caption{Mechanism usage by route.}
\end{table}

\end{document}
